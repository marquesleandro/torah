\subsubsection*{A criação dos céus e da terra}
\textbf{\large 1} 
No princípio, Deus criou os céus e a terra. 
\textit{\tiny 2} 
A   terra era sem forma e vazia, a
escuridão cobria as águas profundas, e o Espírito de Deus se movia sobre a superfície das águas. 

\bigskip
\textit{\tiny 3} 
Então Deus disse: “Haja luz”, e houve luz. 
\textit{\tiny 4} 
E Deus viu que a luz era boa, e separou a luz da escuridão. 
\textit{\tiny 5} Deus chamou a luz de “dia” e a escuridão de “noite”. A noite passou e veio a manhã, encerrando o primeiro dia.

\bigskip
\textit{\tiny 6} Então Deus disse: “Haja um espaço entre as águas, para separar as águas dos céus das águas da terra”. 
\textit{\tiny 7} 
E assim aconteceu. Deus criou um espaço para separar as águas da terra das águas dos céus. 
\textit{\tiny 8} 
Deus chamou o espaço de “céu”. A noite passou e veio a manhã, encerrando o segundo dia.

\bigskip
\textit{\tiny 9} 
Então     Deus disse: “Juntem-se as águas que estão debaixo do céu num só lugar, para que apareça uma parte seca”. E assim aconteceu. 
\textit{\tiny 10} 
Deus chamou a parte seca de “terra” e as águas de “mares”. E Deus viu que isso era bom. 
\textit{\tiny 11} 
Então Deus disse: “Produza a terra vegetação: toda espécie de plantas com sementes e árvores que dão frutos com sementes. As sementes produzirão plantas e árvores, cada uma conforme a sua espécie”. E assim aconteceu.
\textit{\tiny 12} 
A terra produziu vegetação: toda espécie de plantas com sementes e árvores que dão frutos com sementes. As sementes produziram plantas e árvores, cada uma conforme a sua espécie. E Deus viu que isso era bom.      
\textit{\tiny 13}
A noite passou e veio a manhã, encerrando o terceiro dia.

\bigskip
\textit{\tiny 14}
Então  Deus disse: “Haja luzes no céu para separar o dia da noite e marcar as estações, os dias e os anos. 
\textit{\tiny 15}
Que essas luzes brilhem no céu para iluminar a terra”. E assim aconteceu. 
\textit{\tiny 16}
Deus criou duas grandes luzes: a maior para governar o dia e a menor para governar a noite, e criou também as estrelas.
\textit{\tiny 17}
Deus colocou essas luzes no céu para iluminar a terra,
\textit{\tiny 18} 
para governar o dia e a noite e para separar a luz da escuridão. E Deus viu que isso era bom.
\textit{\tiny 19}
A noite passou e veio a manhã, encerrando o quarto dia.

\bigskip
\textit{\tiny 20}
Então Deus disse: “Encham-se as águas de seres vivos, e voem as aves no céu acima da terra”. 
\textit{\tiny 21}
Assim, Deus criou os grandes animais marinhos e todos os seres vivos que se movem em grande número pelas águas, bem como uma grande variedade de aves, cada um conforme a sua espécie. E Deus viu que isso era bom. 
\textit{\tiny 22}
Então Deus os abençoou: “Sejam férteis e multipliquem-se. Que os seres encham os mares e as aves se multipliquem na terra”.
\textit{\tiny 23}
A noite passou e veio a manhã, encerrando o quinto dia.

\bigskip
\textit{\tiny 24}
Então Deus disse: “Produza a terra grande variedade de animais, cada um conforme a sua espécie: animais domésticos, animais que rastejam pelo chão e animais selvagens”. E assim aconteceu. 
\textit{\tiny 25}
Deus criou grande variedade de animais selvagens, animais domésticos e animais que rastejam pelo chão, cada um conforme a sua espécie. E Deus viu que isso era bom.
\textit{\tiny 26}
Então Deus disse: “Façamos o ser humano2 à nossa imagem; ele será semelhante a nós. Dominará sobre os peixes do mar, sobre as aves do céu, sobre os animais domésticos, sobre todos os animais selvagens da terra3 e sobre os animais que rastejam pelo chão”.
\textit{\tiny 27}
Assim, Deus criou os seres humanos à sua própria imagem, à imagem de Deus os criou; homem e mulher os criou.
\textit{\tiny 28}
Então  Deus os abençoou e disse: “Sejam férteis e multipliquem-se. Encham e governem a terra. Dominem sobre os peixes do mar, sobre as aves do céu e sobre todos os animais que rastejam pelo chão”.
\textit{\tiny 29}
Então Deus disse: “Vejam! Eu lhes dou todas as plantas com sementes em toda a tetra e todas as árvores frutíferas, para que lhes sirvam de alimento. 
\textit{\tiny 30}
E dou todas as plantas verdes como alimento a todos os seres vivos: aos animais selvagens, às aves do céu e aos animais que rastejam pelo chão”. E assim aconteceu.
\textit{\tiny 31}
Então Deus olhou para tudo que havia feito e viu que era muito bom. A noite passou e veio a manhã, encerrando o sexto dia.

\bigskip
\textbf{\large 2} 
Desse modo, completou-se a criação dos céus e da terra e de tudo que neles há. 
\textit{\tiny 2}
No sétimo dia, Deus havia terminado sua obra de criação e descansou de todo o seu trabalho. 
\textit{\tiny 3}
Deus abençoou o sétimo dia e o declarou santo, pois foi o dia em que ele descansou de toda a sua obra de criação.

\bigskip
\subsubsection*{O homem, a mulher e o jardim}
\textit{\tiny 4}
Esse é o relato da criação dos céus e da terra. Quando o SENHOR Deus criou a terra e os céus, 
\textit{\tiny 5}
nenhuma planta silvestre nem grãos haviam brotado na terra, pois o SENHOR Deus ainda não tinha mandado
chuva para regar a terra, e não havia quem a cultivasse. 
\textit{\tiny 6}
Mas do solo brotava água, que regava toda a terra. 

\bigskip
\textit{\tiny 7}
Então o SENHOR Deus formou o homem do pó da
terra. Soprou o fôlego da vida em suas narinas, e o homem se tornou ser vivo.
\textit{\tiny 8}
O SENHOR Deus plantou um jardim no Éden, para os lados do leste, e ali
colocou o homem que havia criado. 

\bigskip
\textit{\tiny 9}
O SENHOR Deus fez brotar do solo árvores de
todas as espécies, árvores lindas que produziam frutos deliciosos. No meio do
jardim, colocou a árvore da vida e a árvore do conhecimento do bem e do mal.

\bigskip
\textit{\tiny 10}
Da terra do Éden nascia um rio que regava o jardim e depois se dividia em
quatro braços.  
\textit{\tiny 11}O primeiro braço, chamado Pisom, rodeava toda a terra de
Havilá, onde existe ouro.  
\textit{\tiny 12}O ouro dessa terra é de grande pureza; lá também há
resina aromática e pedra de ônix.  
\textit{\tiny 13}O segundo braço, chamado Giom, rodeava
toda a terra de Cuxe. 
\textit{\tiny 14}O terceiro braço, chamado Tigre, corria para o leste da
terra da Assíria. O quarto braço era chamado de Eufrates.

\bigskip
\textit{\tiny 15}O SENHOR Deus colocou o homem no jardim do Éden para cultivá-lo e tomar
conta dele,  
\textit{\tiny 16}mas o SENHOR Deus lhe ordenou: “Coma à vontade dos frutos de
todas as árvores do jardim,  
\textit{\tiny 17}exceto da árvore do conhecimento do bem e do mal.
Se você comer desse fruto, com certeza morrerá”.

\bigskip
\textit{\tiny 18}O SENHOR Deus disse: “Não é bom que o homem esteja sozinho. Farei alguém
que o ajude e o complete”.  
\textit{\tiny 19}O SENHOR Deus formou da terra todos os animais
selvagens e todas as aves do céu. Trouxe-os ao homem para ver como os
chamaria, e o homem escolheu um nome para cada um deles.  
\textit{\tiny 20}Deu nome a todos
os animais domésticos, a todas as aves do céu e a todos os animais selvagens. O
homem, porém, continuava sem alguém que o ajudasse e o completasse.

\bigskip
\textit{\tiny 21}Então o SENHOR Deus o fez cair num sono profundo. Enquanto o homem
dormia, tirou dele uma das costelas e fechou o espaço que ela ocupava.  
\textit{\tiny 22}Dessa
costela o SENHOR Deus fez uma mulher e a trouxe ao homem.
\textit{\tiny 23}“Finalmente!”, exclamou o homem.
“Esta é osso dos meus ossos,
      e carne da minha carne!
    Será chamada ‘mulher’,
      porque foi tirada do ‘homem’”.
\textit{\tiny 24}Por   isso o homem deixa pai e mãe e se une à sua mulher, e os dois se tornam
um só.
\textit{\tiny 25}O homem e a mulher estavam nus, mas não sentiam vergonha.

\bigskip
\subsubsection*{O homem, a mulher e a serpente}
\textbf{\large 3}
A serpente era o mais astuto de todos os animais selvagens que o SENHOR
Deus havia criado. Certa vez, ela perguntou à mulher: “Deus realmente disse que
vocês não devem comer do fruto de nenhuma das árvores do jardim?”.
\textit{\tiny 2}
“Podemos comer do fruto das árvores do jardim”, respondeu a mulher. 
\textit{\tiny 3}
“É só
do fruto da árvore que está no meio do jardim que não podemos comer. Deus
disse: ‘Não comam e nem sequer toquem no fruto daquela árvore; se o fizerem,
morrerão’.”
\textit{\tiny 4}
“É claro que vocês não morrerão!”, a serpente respondeu à mulher. 
\textit{\tiny 5}
“Deus
sabe que, no momento em que comerem do fruto, seus olhos se abrirão e, como
Deus, conhecerão o bem e o mal.”

\bigskip
\textit{\tiny 6}
A mulher viu que a árvore era linda e que seu fruto parecia delicioso, e desejou
a sabedoria que ele lhe daria. Assim, tomou do fruto e o comeu. Depois, deu ao
marido, que estava com ela, e ele também comeu. 
\textit{\tiny 7}
Naquele momento, seus olhos
se abriram, e eles perceberam que estavam nus. Por isso, costuraram folhas de
figueira umas às outras para se cobrirem.

\bigskip
\textit{\tiny 8}
Quando soprava a brisa do entardecer, o homem e sua mulher ouviram o
SENHOR Deus caminhando pelo jardim e se esconderam dele entre as árvores.
\textit{\tiny 9}
Então o SENHOR Deus chamou o homem e perguntou: “Onde você está?”.
\textit{\tiny 10}
Ele respondeu: “Ouvi que estavas andando pelo jardim e me escondi. Tive
medo, pois eu estava nu”.
\textit{\tiny 11}
“Quem    lhe disse que você estava nu?”, perguntou Deus. “Você comeu do
fruto da árvore que eu lhe ordenei que não comesse?”
\textit{\tiny 12}
O homem respondeu: “Foi a mulher que me deste! Ela me ofereceu do fruto,
e eu comi”.
\textit{\tiny 13}
Então o SENHOR Deus perguntou à mulher: “O que foi que você fez?”.
   “A serpente me enganou”, respondeu a mulher. “Foi por isso que comi do
fruto.”


\bigskip
\textit{\tiny 14}
Então o SENHOR Deus disse à serpente:
  “Uma vez que fez isso, maldita é você
     entre todos os animais, domésticos e selvagens.
  Você se arrastará sobre o próprio ventre,
     rastejará no pó enquanto viver.
\textit{\tiny 15}
Farei que haja inimizade entre você e a mulher,
     e entre a sua descendência e o descendente dela.
  Ele lhe ferirá a cabeça,
     e você lhe ferirá o calcanhar”.
\textit{\tiny 16}
À mulher ele disse:
  “Farei mais intensas as dores de sua gravidez,
    e com dor você dará à luz.
  Seu desejo será para seu marido,
    e ele a dominará”.
\textit{\tiny 17}
E ao homem ele disse:
  “Uma vez que você deu ouvidos à sua mulher
    e comeu da árvore cujo fruto ordenei que não comesse,
  maldita é a terra por sua causa;
    por toda a vida, terá muito trabalho para tirar da terra seu sustento.
\textit{\tiny 18}
Ela produzirá espinhos e ervas daninhas,
    mas você comerá de seus frutos e grãos.
\textit{\tiny 19}
Com o suor do rosto você obterá alimento,
    até que volte à terra da qual foi formado.
  Pois você foi feito do pó,
    e ao pó voltará”.

\bigskip
\textit{\tiny 20}
O homem, Adão, deu à sua mulher o nome de Eva, pois ela seria a mãe de toda
a humanidade.  
\textit{\tiny 21}
E o SENHOR Deus fez roupas de peles de animais para Adão e sua
mulher.

\bigskip
\textit{\tiny 22}
Então o SENHOR Deus disse: “Vejam, agora os seres humanos se tornaram
semelhantes a nós, pois conhecem o bem e o mal. Se eles tomarem do fruto da
árvore da vida e dele comerem, viverão para sempre”.  
\textit{\tiny 23}
Para impedir que isso
acontecesse, o SENHOR Deus os expulsou do jardim do Éden, e Adão passou a
cultivar a terra da qual tinha sido formado.  
\textit{\tiny 24}
Depois de expulsá-los, colocou
querubins a leste do jardim do Éden e uma espada flamejante que se movia de um
lado para o outro, a fim de guardar o caminho até a árvore da vida.

\bigskip
\subsubsection*{Caim e Abel}
\textbf{\large 4} 
Adão teve relações com Eva, sua mulher, que engravidou. Quando deu à luz
Caim, ela disse: “Com a ajuda do SENHOR, tive um filho!”. 
\textit{\tiny 2}
Tempos depois, deu à
luz o irmão de Caim e o chamou de Abel.
   Quando os meninos cresceram, Abel se tornou pastor de ovelhas, e Caim
cultivava o solo. 

\bigskip
\textit{\tiny 3}
No tempo da colheita, Caim apresentou parte de sua produção
como oferta ao SENHOR. 
\textit{\tiny 4}
Abel, por sua vez, ofertou as melhores porções dos
cordeiros dentre as primeiras crias de seu rebanho. O SENHOR aceitou Abel e sua
oferta, 
\textit{\tiny 5}
mas não aceitou Caim e sua oferta. 

\bigskip
Caim se enfureceu e ficou
transtornado.
\textit{\tiny 6}
“Por que você está tão furioso?”, o SENHOR perguntou a Caim. “Por que está tão
transtornado? 
\textit{\tiny 7}
Se você fizer o que é certo, será aceito. Mas, se não o fizer, tome
cuidado! O pecado está à porta, à sua espera, e deseja controlá-lo, mas é você
quem deve dominá-lo.”
\textit{\tiny 8}
Caim sugeriu a seu irmão: “Vamos ao campo”. E, enquanto estavam lá, Caim
atacou seu irmão Abel e o matou.

\bigskip
\textit{\tiny 9}
Então o SENHOR perguntou a Caim: “Onde está seu irmão? Onde está Abel?”.
   “Não sei”, respondeu Caim. “Por acaso sou responsável por meu irmão?”
\textit{\tiny 10}
Então Deus disse: “O que você fez? Ouça! O sangue de seu irmão clama a mim
da terra!  
\textit{\tiny 11}
O próprio solo, que bebeu o sangue de seu irmão, sangue que você
derramou, amaldiçoa você.  
\textit{\tiny 12}
O solo não lhe dará boas colheitas, por mais que
você se esforce! E, de agora em diante, você não terá um lar e andará sem rumo
pela terra”.
\textit{\tiny 13}
Caim disse ao SENHOR: “Meu castigo é pesado demais. Não posso aguentá-lo!
\textit{\tiny 14}
Tu me expulsaste da terra e de tua presença e me transformaste num andarilho
sem lar. Qualquer um que me encontrar me matará!”.
\textit{\tiny 15}
O SENHOR respondeu: “Eu castigarei sete vezes mais quem matar você”. Então
o SENHOR pôs em Caim um sinal para alertar qualquer um que tentasse matá-lo.

\bigskip
\textit{\tiny 16}
Caim saiu da presença do SENHOR e se estabeleceu na terra de Node, a leste do
Éden.
\textit{\tiny 17}
Caim teve relações com sua mulher, que engravidou e deu à luz Enoque. Então
Caim fundou uma cidade, à qual deu o nome de Enoque, como seu filho.
\textit{\tiny 18}
Enoque teve um filho chamado Irade. Irade gerou Meujael; Meujael gerou
Metusael; Metusael gerou Lameque.
\textit{\tiny 19}
Lameque se casou com duas mulheres. A primeira se chamava Ada, e a
segunda, Zilá.  
\textit{\tiny 20}
Ada deu à luz Jabal; ele foi o precursor dos que criam rebanhos e
moram em tendas.  
\textit{\tiny 21}
Seu irmão se chamava Jubal, o precursor dos que tocam
harpa e flauta.  
\textit{\tiny 22}
Zilá, a outra mulher de Lameque, deu à luz um filho chamado
Tubalcaim, que se tornou mestre em criar ferramentas de bronze e ferro.
Tubalcaim teve uma irmã chamada Naamá.  
\textit{\tiny 23}
Certo dia, Lameque disse a suas
mulheres:
  “Ada e Zilá, ouçam minha voz;
    escutem o que vou dizer, mulheres de Lameque.
  Matei um homem que me atacou,
    um rapaz que me feriu.
\textit{\tiny 24}
Se aquele que matar Caim será castigado sete vezes,
    quem me matar será castigado setenta e sete vezes!”.

\bigskip
\subsubsection*{O nascimento de Sete}
\textit{\tiny 25}
Adão teve relações com sua mulher novamente, e ela deu à luz outro filho.
Chamou-o de Sete, pois disse: “Deus me concedeu outro filho no lugar de Abel, a
quem Caim matou”.  
\textit{\tiny 26}
Quando Sete chegou à idade adulta, teve um filho e o
chamou de Enos. Nessa época, as pessoas começaram a invocar o nome do
SENHOR.

\bigskip
\subsubsection*{Descendentes de Adão}
\textbf{\large 5} 
Este é o relato dos descendentes de Adão. Quando Deus criou os seres
humanos, formou-os semelhantes a ele. 
\textit{\tiny 2}
Criou-os homem e mulher; quando
foram criados, Deus os abençoou e os chamou de “humanidade”.

\bigskip
\textit{\tiny 3}
Aos   130 anos, Adão teve um filho chamado Sete, que era semelhante a ele, à sua
 imagem. 
\textit{\tiny 4}
Depois do nascimento de Sete, Adão viveu mais 800 anos e teve outros
 filhos e filhas. 
\textit{\tiny 5}
Adão viveu 930 anos e morreu.

\bigskip
\textit{\tiny 6}
Aos 105 anos, Sete gerou Enos. 
\textit{\tiny 7}
Depois do nascimento de Enos, Sete viveu mais 807 anos e teve outros filhos e filhas. 
\textit{\tiny 8}
Sete viveu 912 anos e morreu.

\bigskip
\textit{\tiny 9}
Aos 90 anos, Enos gerou Cainã. 
\textit{\tiny 10}
Depois do nascimento de Cainã, Enos viveu mais 815 anos e teve outros filhos e filhas.  
\textit{\tiny 11}
Enos viveu 905 anos e morreu.

\bigskip
\textit{\tiny 12}
Aos 70 anos, Cainã gerou Maalaleel.  
\textit{\tiny 13}
Depois do nascimento de Maalaleel,
 Cainã viveu mais 840 anos e teve outros filhos e filhas.  
\textit{\tiny 14}
Cainã viveu 910 anos e morreu.

\bigskip
\textit{\tiny 15}
Aos 65 anos, Maalaleel gerou Jarede.  
\textit{\tiny 16}
Depois do nascimento de Jarede,
 Maalaleel viveu mais 830 anos e teve outros filhos e filhas.  
\textit{\tiny 17}
Maalaleel viveu 895 anos e morreu.

\bigskip
\textit{\tiny 18}
Aos 162 anos, Jarede gerou Enoque.  
\textit{\tiny 19}
Depois do nascimento de Enoque, Jarede
 viveu mais 800 anos e teve outros filhos e filhas.  
\textit{\tiny 20}
Jarede viveu 962 anos e
 morreu.

\bigskip
\textit{\tiny 21}
Aos 65 anos, Enoque gerou Matusalém.  
\textit{\tiny 22}
Depois do nascimento de Matusalém,
 Enoque viveu em comunhão com Deus por mais 300 anos e teve outros filhos e
 filhas.  
\textit{\tiny 23}
Enoque viveu 365 anos,  
\textit{\tiny 24}
andando em comunhão com Deus até que,
 um dia, desapareceu, porque Deus o levou para junto de si.

\bigskip
\textit{\tiny 25}
Aos 187 anos, Matusalém gerou Lameque.  
\textit{\tiny 26}
Depois do nascimento de
 Lameque, Matusalém viveu mais 782 anos e teve outros filhos e filhas.
\textit{\tiny 27}
Matusalém viveu 969 anos e morreu.

\bigskip
\textit{\tiny 28}
Aos 182 anos, Lameque gerou um filho.  
\textit{\tiny 29}
Chamou-o de Noé, pois disse: “Que
 ele nos traga alívio de nossas tarefas e do trabalho doloroso de cultivar esta terra
 que o SENHOR amaldiçoou”.  
\textit{\tiny 30}
Depois do nascimento de Noé, Lameque viveu mais
 595 anos e teve outros filhos e filhas.  
\textit{\tiny 31}
Lameque viveu 777 anos e morreu.

\bigskip
\textit{\tiny 32}
Depois que completou 500 anos, Noé gerou três filhos: Sem, Cam e Jafé.

\bigskip
\subsubsection*{O Arrependimento da Criação}
\textbf{\large 6} 
Os seres humanos começaram a se multiplicar na terra e tiveram filhas. 
\textit{\tiny 2}
Os filhos de Deus perceberam que as filhas dos homens eram belas, tomaram para si
as que os agradaram e se casaram com elas. 
\textit{\tiny 3}
Então o SENHOR disse: “Meu Espírito
não tolerará os humanos por muito tempo, pois são apenas carne mortal. Seus
dias serão limitados a 120 anos”.
\textit{\tiny 4}
Naqueles dias, e por algum tempo depois, havia na terra gigantes, pois
quando os filhos de Deus tiveram relações com as filhas dos homens, elas deram à
luz filhos que se tornaram os guerreiros famosos da antiguidade.

\bigskip
\textit{\tiny 5}
O SENHOR observou quanto havia aumentado a perversidade dos seres
humanos na terra e viu que todos os seus pensamentos e seus propósitos eram
sempre inteiramente maus. 
\textit{\tiny 6}
E o SENHOR se arrependeu de tê-los criado e colocado
na terra. Isso lhe causou imensa tristeza. 
\textit{\tiny 7}
O SENHOR disse: “Eliminarei da face da
terra esta raça humana que criei. Sim, e também destruirei todos os seres vivos: as
pessoas, os grandes animais, os animais que rastejam pelo chão e até as aves do
céu. Arrependo-me de tê-los criado”. 

\bigskip
\textit{\tiny 8}
Noé, porém, encontrou favor diante do
SENHOR.

----------------------------------------------------------------------
