\section*{Parashat Shemini (9:1–11:47)}

\subsubsection*{Os sacerdotes e a consagração - parte 2}
\textbf{\large 9} No oitavo dia, depois da cerimônia de consagração, Moisés reuniu Arão, seus
filhos e os líderes de Israel 
\textit{\tiny 2} 
e disse a Arão: 

\smallskip
“Escolha um bezerro para a oferta pelo
pecado e um carneiro para o holocausto, ambos sem defeito, e apresente-os ao
SENHOR. 

\smallskip
\textit{\tiny 3} 
Depois, diga aos israelitas: ‘Escolham um bode para a oferta pelo pecado
e um bezerro e um cordeiro, ambos de um ano e sem defeito, para o holocausto. 
\textit{\tiny 4} 
Escolham também um boi e um carneiro para a oferta de paz, além de farinha
misturada com azeite para a oferta de cereal. Apresentem todas essas ofertas ao
SENHOR, pois hoje o SENHOR aparecerá a vocês’”. 

\smallskip
\textit{\tiny 5} 
O povo trouxe todas essas coisas à entrada da tenda do encontro, conforme
Moisés tinha ordenado. Assim, toda a comunidade se aproximou e permaneceu
em pé diante do SENHOR. 
\textit{\tiny 6} 
Então Moisés disse: “É isto que o SENHOR ordenou que
façam para que a glória do SENHOR lhes apareça”. 

\smallskip
\textit{\tiny 7} 
Em seguida, Moisés disse a Arão: “Venha até o altar e apresente sua oferta pelo
pecado e seu holocausto para fazer expiação por si mesmo e pelo povo. Apresente
as ofertas do povo para fazer expiação por eles, conforme o SENHOR ordenou”. 

\smallskip
\textit{\tiny 8} 
Arão foi até o altar e matou o bezerro como oferta pelo pecado por si mesmo. 
\textit{\tiny 9} 
Seus filhos lhe trouxeram o sangue, e Arão molhou o dedo nele e o colocou nas
pontas do altar. O restante do sangue ele derramou na base do altar. 
\textit{\tiny 10}
Queimou
no altar a gordura, os rins e o lóbulo do fígado da oferta pelo pecado, conforme o
SENHOR havia ordenado a Moisés. 
\textit{\tiny 11}
A carne e o couro, porém, queimou fora do
acampamento.

\smallskip
\textit{\tiny 12}
Então Arão matou o animal para o holocausto. Seus filhos lhe trouxeram o
sangue, e ele o derramou em todos os lados do altar. 
\textit{\tiny 13}
Entregaram-lhe cada um
dos pedaços do holocausto, incluindo a cabeça, e ele os queimou no altar. 
\textit{\tiny 14}
Lavou
os órgãos internos e as pernas e os queimou no altar junto com o restante do
holocausto.

\smallskip
\textit{\tiny 15}
Em seguida, Arão apresentou as ofertas do povo. Matou o bode do povo e o
apresentou como oferta pelo pecado deles, como havia feito com a oferta por seu
próprio pecado. 
\textit{\tiny 16}
Depois, apresentou o holocausto e o ofereceu de acordo com a
forma prescrita. 

\smallskip
\textit{\tiny 17}
Apresentou também a oferta de cereal e queimou no altar um
punhado dela, além do holocausto da manhã.

\smallskip 
\textit{\tiny 18}
Arão matou o boi e o carneiro para a oferta de paz do povo. Seus filhos lhe
trouxeram o sangue, e ele o derramou em todos os lados do altar. 
\textit{\tiny 19}
Depois, pegou
a gordura do boi e do carneiro, incluindo a gordura da parte gorda da cauda e a
gordura que envolve os órgãos internos, bem como os rins e o lóbulo do fígado de
cada animal, 
\textit{\tiny 20}
colocou as porções de gordura sobre o peito dos animais e as
queimou no altar. 

\smallskip 
\textit{\tiny 21}
Arão moveu o peito e a coxa direita dos animais para o alto
como oferta especial para o SENHOR, conforme Moisés havia ordenado.

\smallskip
\textit{\tiny 22}
Por fim, Arão ergueu as mãos na direção do povo e o abençoou. Depois de
apresentar a oferta pelo pecado, o holocausto e a oferta de paz, desceu do altar.
\textit{\tiny 23}
Então Moisés e Arão entraram na tenda do encontro e, quando voltaram,
abençoaram o povo novamente, e a glória do SENHOR apareceu a todo o povo.
\textit{\tiny 24}
Fogo saiu da presença do SENHOR e consumiu o holocausto e a gordura no altar.
Quando eles viram isso, gritaram de alegria e se prostraram com o rosto no chão.

\bigskip
\subsubsection*{Nadabe e Abiú}
\textbf{\large 10}
 Nadabe e Abiú, filhos de Arão, colocaram brasas em seus incensários e as
salpicaram com incenso. Com isso, trouxeram fogo estranho diante do SENHOR,
diferente do que ele havia ordenado. 
\textit{\tiny 2} 
Por isso, fogo saiu da presença do SENHOR e
os devorou, e eles morreram diante do SENHOR.

\smallskip 
\textit{\tiny 3} 
Então Moisés disse a Arão: “Foi isto que o SENHOR declarou:
    ‘Mostrarei minha santidade
      entre aqueles que se aproximarem de mim.
    Mostrarei minha glória
      diante de todo o povo’”.
E Arão ficou em silêncio. 
\textit{\tiny 4} 
Moisés chamou Misael e Elzafã, primos de Arão e filhos de Uziel, tio de Arão, e
lhes disse: “Venham cá e levem o corpo de seus parentes da frente do santuário
para um lugar fora do acampamento”. 
\textit{\tiny 5} 
Eles se aproximaram e os puxaram pelas
roupas para fora do acampamento, conforme Moisés havia ordenado. 
\textit{\tiny 6} 
Então Moisés disse a Arão e a seus filhos Eleazar e Itamar: “Não deixem o
cabelo despenteado
 nem rasguem suas roupas em sinal de luto. Se o fizerem,
morrerão, e a ira do SENHOR ferirá toda a comunidade de Israel. Mas outros
israelitas, seus parentes, poderão ficar de luto porque o SENHOR destruiu Nadabe e
Abiú com fogo. 
\textit{\tiny 7} 
Não saiam da entrada da tenda do encontro, ou morrerão, pois
foram ungidos com o óleo da unção do SENHOR”. E fizeram conforme Moisés
ordenou.

\smallskip 
\textit{\tiny 8} 
Então o SENHOR disse a Arão: 
\textit{\tiny 9} 
“Você e seus descendentes jamais deverão beber
vinho ou qualquer outra bebida fermentada antes de entrar na tenda do encontro.
Se o fizerem, morrerão. Essa é uma lei permanente para vocês e deve ser
cumprida de geração em geração. 
\textit{\tiny 10}
Façam distinção entre o que é santo e o que é
comum, entre o que é impuro e o que é puro, 
\textit{\tiny 11}
e ensinem aos israelitas todos os
decretos que o SENHOR lhes deu por meio de Moisés”.

\smallskip 
\textit{\tiny 12}
Moisés disse a Arão e aos filhos que lhe restaram, Eleazar e Itamar: “Peguem
o que sobrar da oferta de cereal depois que uma porção tiver sido apresentada
como oferta especial para o SENHOR e comam-na junto do altar. Não deverá conter
fermento, pois é santíssima. 
\textit{\tiny 13}
Comam-na num lugar sagrado, pois foi dada a
vocês e a seus descendentes como sua porção das ofertas especiais apresentadas
ao SENHOR. Foram essas as ordens que recebi. 

\smallskip 
\textit{\tiny 14}
Quanto ao peito e à coxa que
foram movidos para o alto como oferta especial, poderão comê-los em qualquer
lugar cerimonialmente puro. Essas são as partes que foram dadas a você e a seus
descendentes como sua porção das ofertas de paz apresentadas pelos israelitas.
\textit{\tiny 15}
Movam para o alto o peito e a coxa como oferta especial para o SENHOR, junto
com a gordura das ofertas especiais. Essas partes pertencerão a vocês e a seus
descendentes como direito permanente, conforme o SENHOR ordenou”.

\smallskip 
\textit{\tiny 16}
Depois, Moisés procurou cuidadosamente pelo bode da oferta pelo pecado.
Quando descobriu que tinha sido queimado, ficou furioso com Eleazar e Itamar,
os filhos que restaram a Arão, e lhes disse: 
\textit{\tiny 17}
“Por que não comeram a oferta pelo
pecado no lugar sagrado? É uma oferta santíssima! O SENHOR a deu a vocês para
remover a culpa da comunidade e fazer expiação por ela. 
\textit{\tiny 18}
Uma vez que o sangue
do animal não foi levado ao lugar santo, vocês tinham a obrigação de comer a
carne no lugar sagrado, conforme eu ordenei!”.

\smallskip 
\textit{\tiny 19}
Arão respondeu a Moisés: “Hoje meus filhos apresentaram ao SENHOR sua
oferta pelo pecado e seu holocausto. E, no entanto, esta tragédia aconteceu
comigo. Será que o SENHOR teria se agradado se eu tivesse comido a oferta pelo
pecado do povo num dia como este?”. 
\textit{\tiny 20}
Quando Moisés ouviu isso, deu-se por
satisfeito.
Animais cerimonialmente puros e impuros
   
\bigskip
\subsubsection*{Os animais e o alimento}
\textbf{\large 11}
 O SENHOR disse a Moisés e a Arão: 
\textit{\tiny 2} 
“Deem as seguintes instruções ao povo
de Israel.
  “De todos os animais que vivem em terra,
 estes são os que vocês poderão
consumir como alimento: 

\smallskip 
\textit{\tiny 3} 
qualquer animal que tenha os cascos divididos em
duas partes e que rumine. 
\textit{\tiny 4} 
Mas, se o animal não apresentar essas duas
características, não pode ser consumido. O camelo rumina, mas não tem os cascos
divididos, de modo que é impuro para vocês. 
\textit{\tiny 5} 
O coelho silvestre
 rumina, mas
não tem cascos divididos, por isso é impuro. 
\textit{\tiny 6} 
A lebre rumina, mas não tem cascos
divididos, de modo que é impura. 
\textit{\tiny 7} 
O porco, embora tenha os cascos divididos,
não rumina e, portanto, também é impuro. 
\textit{\tiny 8} 
Não comerão a carne desses animais
nem tocarão em seu cadáver. São cerimonialmente impuros para vocês. 

\smallskip 
\textit{\tiny 9} 
“De todos os animais que vivem nas águas, estes são os que vocês poderão
consumir como alimento: qualquer animal aquático que tenha barbatanas e
escamas, seja de água salgada ou de rios. 
\textit{\tiny 10}
Contudo, jamais comerão animais de
mar ou de rio que não tenham barbatanas e escamas. São detestáveis para vocês.
Isso se aplica tanto às criaturas pequenas que vivem em águas rasas como a todas
as criaturas que vivem em águas profundas. 
\textit{\tiny 11}
Serão sempre detestáveis para
vocês. Não comerão a carne delas nem tocarão em seu cadáver. 
\textit{\tiny 12}
Qualquer
animal aquático que não tem barbatanas e escamas é detestável para vocês.

\smallskip 
\textit{\tiny 13}
“Estes são os animais voadores que vocês considerarão detestáveis e não
comerão: o abutre-fouveiro, o abutre-barbudo, o abutre-fusco, 
\textit{\tiny 14}
o milhafre e
todas as espécies de falcão, 
\textit{\tiny 15}
todas as espécies de corvos, 
\textit{\tiny 16}
a coruja-de-chifres, a
coruja-do-campo, a gaivota, todas as espécies de gaviões, 
\textit{\tiny 17}
o mocho-galego, o
cormorão, o corujão, 
\textit{\tiny 18}
a coruja-das-torres, a coruja-do-deserto, o abutre-do-egito,
\textit{\tiny 19}
a cegonha, todas as espécies de garças, a poupa e o morcego.

\smallskip 
\textit{\tiny 20}
“Não comerão insetos voadores que rastejam pelo chão, pois são detestáveis
para vocês. 
\textit{\tiny 21}
Contudo, poderão comer insetos voadores que andam pelo chão e têm
pernas articuladas para saltar. 
\textit{\tiny 22}
Os insetos que vocês poderão comer incluem
todas as espécies de gafanhotos, gafanhotos migradores, grilos e gafanhotos
devoradores. 
\textit{\tiny 23}
Todos os outros insetos voadores que andam pelo chão são
detestáveis para vocês.
\textit{\tiny 24}
“Por causa dessas criaturas vocês se tornarão cerimonialmente impuros.
Quem tocar em seus cadáveres ficará contaminado até o entardecer. 
\textit{\tiny 25}
Quem
carregar o cadáver delas deverá lavar as roupas e ficará contaminado até o
entardecer.
\textit{\tiny 26}
“Todo animal com cascos divididos de forma desigual ou que não rumina é
impuro para vocês. Quem tocar em algum desses animais ficará contaminado.
\textit{\tiny 27}
Dentre os quadrúpedes, aqueles que andam sobre a planta dos pés são impuros.
Se alguém tocar no cadáver de algum desses animais, ficará impuro até o
entardecer. 
\textit{\tiny 28}
Se carregar o cadáver deles, deverá lavar as roupas e ficará
contaminado até o entardecer. Esses animais são impuros para vocês.
   
\smallskip
\textit{\tiny 29}
“Dos animais pequenos que rastejam pelo chão, estes são impuros para vocês:
a doninha, o rato, todas as espécies de lagartos grandes, 
\textit{\tiny 30}
a lagartixa, o lagarto
pintado, o lagarto comum, o lagarto da areia e o camaleão. 
\textit{\tiny 31}
Todos esses animais
pequenos são impuros para vocês. Se alguém tocar no cadáver de um deles, ficará
contaminado até o entardecer. 
\textit{\tiny 32}
Se um deles morrer e cair sobre algo, tornará
impuro esse objeto, seja de madeira, tecido, couro ou pano de saco. Qualquer que
seja seu uso, deverá ser colocado de molho em água e ficará impuro até o
entardecer. Depois disso, estará cerimonialmente puro e poderá ser usado
novamente.
\textit{\tiny 33}
“Se um desses animais cair numa vasilha de barro, tudo que estiver dentro da
vasilha ficará contaminado, e a vasilha deverá ser despedaçada. 
\textit{\tiny 34}
Se a água dessa
vasilha cair sobre algum alimento, ele ficará contaminado. Qualquer bebida que
estiver dentro da vasilha ficará contaminada. 
\textit{\tiny 35}
Qualquer objeto no qual o cadáver
de um desses animais cair ficará contaminado. Se o objeto for um fogão ou um
forno de barro, deverá ser destruído, pois está contaminado e deverá ser tratado
como tal.
\textit{\tiny 36}
“Se o cadáver de um desses animais cair numa fonte ou cisterna, a água
continuará pura. Quem tocar no cadáver, porém, ficará cerimonialmente impuro.
\textit{\tiny 37}
Se o cadáver cair sobre sementes a serem plantadas no campo, ainda assim as
sementes serão consideradas puras. 
\textit{\tiny 38}
Mas, se já tiverem sido regadas quando o
cadáver cair sobre elas, as sementes serão impuras.
\textit{\tiny 39}
“Se morrer um animal que vocês têm permissão de comer e alguém tocar no
cadáver, ficará impuro até o entardecer. 
\textit{\tiny 40}
Se alguém comer da carne do animal
ou carregar o cadáver, deverá lavar as roupas e ficará impuro até o entardecer.
\textit{\tiny 41}
“Todos os animais pequenos que rastejam pelo chão são detestáveis, e vocês
jamais devem comê-los. 
\textit{\tiny 42}
Isso inclui todos os animais que se arrastam sobre o
ventre, bem como os que têm quatro pernas e os que têm muitas patas. Todos
esses animais que rastejam pelo chão são detestáveis, e vocês jamais devem comê-
los. 
\textit{\tiny 43}
Não se contaminem com eles. Não se tornem cerimonialmente impuros por
causa deles, 
\textit{\tiny 44}
pois eu sou o SENHOR, seu Deus. Consagrem-se e sejam santos, pois
eu sou santo. Não se contaminem com nenhum desses animais pequenos que
rastejam pelo chão. 
\textit{\tiny 45}
Eu, o SENHOR, sou aquele que os tirou da terra do Egito para
ser o seu Deus. Por isso, sejam santos, pois eu sou santo.

\smallskip
\textit{\tiny 46}
“Essas são as instruções acerca dos animais que vivem em terra, dos animais
voadores, das criaturas aquáticas e dos animais que rastejam pelo chão. 
\textit{\tiny 47}
Com
essas instruções, vocês saberão o que é impuro e o que é puro, os animais que
vocês podem comer e os que não podem”.



----------------------------------------------------------------------
