\section*{Parashat Toledot 25:19 - 28:9}

\subsubsection*{Isaque e os gêmeos}

\textit{\tiny 19}
Este é o relato da família de Isaque, filho de Abraão. 

\bigskip
\textit{\tiny 20}
Quando Isaque tinha 40 anos, casou-se com Rebeca, filha de Betuel, o arameu de Padã-Arã, e irmã de
Labão, o arameu.
\textit{\tiny 21}
Isaque orou ao SENHOR em favor de sua mulher, pois ela não podia ter filhos.
O SENHOR ouviu a oração de Isaque, e Rebeca ficou grávida de gêmeos. 
\textit{\tiny 22}
Os dois
bebês lutavam um com o outro no ventre da mãe, de modo que ela consultou o
SENHOR a esse respeito. “Por que isso está acontecendo comigo?”, perguntou ela.
\textit{\tiny 23}
O SENHOR respondeu: “Os filhos em seu ventre se tornarão duas nações.
Desde o começo, elas serão rivais. Uma nação será mais forte que a outra, e seu
filho mais velho servirá a seu filho mais novo”.
   
\bigskip
\textit{\tiny 24}
Quando chegou a hora de dar à luz, Rebeca descobriu que, de fato, eram
gêmeos. 
\textit{\tiny 25}
O primeiro a nascer era ruivo e coberto de pelos; por isso o chamaram
de Esaú. 
\textit{\tiny 26}
Depois, nasceu o outro gêmeo, com a mão agarrada ao calcanhar de
Esaú; por isso o chamaram de Jacó. 

\bigskip
Isaque tinha 60 anos quando os gêmeos
nasceram.



\subsubsection*{Jacó, Esaú e a Primogenitura}

\textit{\tiny 27}
Os meninos cresceram. Esaú se tornou um caçador habilidoso que vivia ao ar
livre, enquanto Jacó era mais pacato e preferia ficar em casa. 
\textit{\tiny 28}
Isaque amava Esaú
porque gostava de comer a carne de caça que ele trazia, mas Rebeca amava Jacó.

\bigskip   
\textit{\tiny 29}
Certo dia, quando Jacó preparava um ensopado, Esaú chegou do deserto,
exausto e faminto. 
\textit{\tiny 30}
“Estou faminto!”, disse ele a Jacó. “Dê-me um pouco desse
ensopado vermelho!” (Por isso Esaú também ficou conhecido como Edom.)
\textit{\tiny 31}
“Está bem”, respondeu Jacó. “Mas, em troca, dê-me seus direitos de filho mais
velho.”
   
\bigskip   
\textit{\tiny 32}
“Estou morrendo de fome!”, disse Esaú. “De que me servem meus direitos de
filho mais velho?”
\textit{\tiny 33}
Mas Jacó disse: “Primeiro, jure que seus direitos de filho mais velho agora são
meus”. Esaú fez um juramento e, desse modo, vendeu todos os seus direitos de
filho mais velho a seu irmão, Jacó.
\textit{\tiny 34}
Então Jacó deu a Esaú um pedaço de pão e o ensopado de lentilhas. Esaú
comeu, levantou-se e foi embora. 

\bigskip   
Assim, ele desprezou seu direito de filho mais
velho.


\subsubsection*{Isaque e os filisteus em Gerar}
   
\textbf{\large 26}
 Uma fome terrível atingiu a região, como havia acontecido antes no tempo
de Abraão. Por isso, Isaque se mudou para Gerar, onde vivia Abimeleque, rei dos
filisteus.

\bigskip   
\textit{\tiny 2}
O SENHOR apareceu a Isaque e disse: “Não desça ao Egito. Faça o que eu
mandar. 
\textit{\tiny 3}
Habite aqui como estrangeiro, e eu estarei com você e o abençoarei.
Com isso, confirmo que darei todas estas terras a você e a seus descendentes,
conforme prometi solenemente a Abraão, seu pai. 
\textit{\tiny 4}
Farei que seus descendentes
sejam tão numerosos quanto as estrelas do céu e darei a eles todas estas terras.
Por meio de sua descendência, todas as nações da terra serão abençoadas. 
\textit{\tiny 5}
Farei
isso porque Abraão me deu ouvidos e obedeceu ao que lhe ordenei: meus
mandamentos, decretos e instruções”. 

\bigskip   
\textit{\tiny 6}
Portanto, Isaque ficou em Gerar.

\bigskip   
\textit{\tiny 7}
Quando os homens que viviam na região perguntaram a Isaque sobre Rebeca,
sua mulher, ele disse: “É minha irmã”. Teve medo de dizer “É minha mulher”, pois
pensou: “Ela é tão bonita que os homens vão me matar por causa dela”. 
\textit{\tiny 8}
Algum
tempo depois, porém, Abimeleque, rei dos filisteus, olhou pela janela e viu Isaque
acariciar Rebeca.
\textit{\tiny 9}
No mesmo instante, Abimeleque mandou chamar Isaque e exclamou: “É
evidente que ela é sua mulher! Por que você disse que era sua irmã?”.
   “Porque tive medo que alguém me matasse por causa dela”, respondeu Isaque.
\textit{\tiny 10}
“Como você pôde fazer uma coisa dessas conosco?”, exclamou Abimeleque.
“Um dos meus homens poderia ter tomado sua mulher e dormido com ela e, por
sua causa, seríamos culpados de grande pecado!”
\textit{\tiny 11}
Então Abimeleque declarou a todo o povo: “Quem tocar neste homem ou em
sua mulher será executado!”.



\subsubsection*{Isaque e os filisteus em Berseba}
\textit{\tiny 12}
Naquele ano, quando Isaque plantou lavouras, colheu cem vezes mais cereais do
que havia semeado, pois o SENHOR o abençoou. 
\textit{\tiny 13}
Isaque prosperou e se tornou
rico e influente. 
\textit{\tiny 14}
Adquiriu tantos rebanhos de ovelhas e bois e tantos servos que
os filisteus o invejaram. 
\textit{\tiny 15}
Por isso, os filisteus fecharam com terra todos os poços
de Isaque, que tinham sido cavados pelos servos de seu pai, Abraão.
\textit{\tiny 16}
Por fim, Abimeleque ordenou que Isaque deixasse aquela terra. “Vá para
outro lugar”, disse ele. “Você se tornou poderoso demais para nós.”
   
\bigskip   
\textit{\tiny 17}
Então Isaque partiu de onde estava e se estabeleceu no vale de Gerar, onde
armou suas tendas. 
\textit{\tiny 18}
Reabriu os poços que seu pai havia cavado e que os filisteus
haviam fechado depois da morte de Abraão e lhes deu os mesmos nomes que
Abraão tinha dado.
\textit{\tiny 19}
Os servos de Isaque também cavaram no vale de Gerar e encontraram uma
fonte de água corrente. 
\textit{\tiny 20}
Contudo, os pastores de Gerar entraram em conflito
com os pastores de Isaque. “Esta água é nossa!”, diziam eles. Por isso, Isaque
chamou o poço de Eseque. 
\textit{\tiny 21}
Em seguida, os homens de Isaque cavaram outro
poço, mas, novamente, houve conflito por causa dele. Por isso, Isaque o chamou
de Sitna. 
\textit{\tiny 22}
Isaque abandonou esse poço e mandou cavar outro mais adiante.
Dessa vez, ninguém discutiu por causa dele, de modo que Isaque o chamou de
Reobote, pois disse: “Finalmente o SENHOR criou espaço suficiente para
prosperarmos nesta terra!”.
   
\bigskip   
\textit{\tiny 23}
Dali, Isaque se mudou para Berseba, 
\textit{\tiny 24}
onde o SENHOR lhe apareceu na noite
de sua chegada e disse: “Eu sou o Deus de seu pai, Abraão. Não tenha medo, pois
estou com você e o abençoarei. Multiplicarei seus descendentes, e eles se tornarão
uma grande nação. Farei isso por causa da minha promessa ao meu servo,
Abraão”. 
\textit{\tiny 25}
Isaque construiu ali um altar e invocou o nome do SENHOR. Armou
acampamento naquele local, e seus servos cavaram outro poço.

\bigskip   
\textit{\tiny 26}
Certo dia, o rei Abimeleque veio de Gerar com Auzate, seu conselheiro, e com
Ficol, comandante do seu exército. 
\textit{\tiny 27}
“Por que vocês vieram?”, perguntou Isaque.
“É evidente que me odeiam, já que me expulsaram de sua terra.”
\textit{\tiny 28}
Eles responderam: “Podemos ver claramente que o SENHOR está com você. Por
isso, queremos fazer com você um acordo sob juramento, uma aliança. 
\textit{\tiny 29}
Jure que
não nos fará mal, assim como nós nunca lhe fizemos mal. Sempre o tratamos bem
e o despedimos em paz. E agora, veja como o SENHOR o abençoou!”.
\textit{\tiny 30}
Então Isaque lhes preparou um banquete, e eles comeram e beberam juntos.
\textit{\tiny 31}
Logo cedo, na manhã seguinte, cada um fez o juramento solene de não
interferir com o outro. Isaque se despediu deles, e partiram em paz.
\textit{\tiny 32}
Naquele mesmo dia, os servos de Isaque vieram lhe falar de um novo poço
que tinham cavado. “Encontramos água!”, exclamaram. 
\textit{\tiny 33}
Por isso, Isaque chamou
o poço de Seba. E, até hoje, a cidade que se formou ali é conhecida como
Berseba.


\subsubsection*{Isaque, os gêmeos e a benção}
\textit{\tiny 34}
Quando Esaú tinha 
\textit{\tiny 40}
 anos, casou-se com duas mulheres hititas: Judite, filha
de Beeri, e Basemate, filha de Elom. 
\textit{\tiny 35}
Essas duas mulheres causaram grande
desgosto a Isaque e Rebeca.
Jacó rouba a bênção de Esaú
   
\bigskip  
\textbf{\large 27}
 Certo dia, quando Isaque era velho e estava ficando cego, chamou Esaú,
seu filho mais velho: “Meu filho!”.
   Esaú respondeu: “Aqui estou!”.
\textit{\tiny 2}
Isaque disse: “Estou velho e não sei quando vou morrer. 
\textit{\tiny 3}
Pegue suas armas, o
arco e as flechas, e vá ao campo caçar um animal para mim. 
\textit{\tiny 4}
Depois, prepare meu
prato favorito e traga-o aqui para eu comer. Então pronunciarei a bênção que
pertence a você, meu filho mais velho, antes de eu morrer”.

\bigskip  
\textit{\tiny 5}
Rebeca, porém, ouviu o que Isaque tinha dito a seu filho Esaú. Quando Esaú
saiu para caçar, 
\textit{\tiny 6}
ela disse a seu filho Jacó: “Ouvi seu pai dizer a Esaú: 
\textit{\tiny 7}
‘Traga-me
uma carne de caça e prepare-me uma refeição saborosa. Então abençoarei você
na presença do SENHOR antes de eu morrer’. 
\textit{\tiny 8}
Agora, meu filho, preste atenção e
faça exatamente o que lhe direi. 
\textit{\tiny 9}
Vá ao rebanho e traga-me dois dos melhores
cabritos. Eu os usarei para preparar o prato favorito de seu pai. 
\textit{\tiny 10}
Depois, leve a
comida para seu pai, para que ele a coma e o abençoe antes de morrer”.
\textit{\tiny 11}
Jacó respondeu a Rebeca: “Mas meu irmão Esaú é peludo, enquanto eu tenho
pele lisa. 
\textit{\tiny 12}
E se meu pai me tocar? Perceberá que estou tentando enganá-lo e, em
vez de me abençoar, me amaldiçoará!”.
\textit{\tiny 13}
Sua mãe, porém, respondeu: “Que caia sobre mim essa maldição, meu filho!
Apenas faça o que lhe digo. Vá e traga-me os cabritos”.

\bigskip   
\textit{\tiny 14}
Jacó foi e trouxe os cabritos para sua mãe. Rebeca os usou para preparar uma
refeição saborosa, do jeito que Isaque gostava. 
\textit{\tiny 15}
Em seguida, pegou as roupas
prediletas de Esaú que estavam na casa dela e as entregou a Jacó, seu filho mais
novo. 
\textit{\tiny 16}
Com a pele dos cabritos, cobriu-lhe os braços e a parte lisa do pescoço.
\textit{\tiny 17}
Depois, entregou-lhe a refeição saborosa, acompanhada do pão que havia
acabado de assar.
   
\bigskip   
\textit{\tiny 18}
Jacó levou a comida para o pai e disse: “Meu pai?”.
“Sim, meu filho”, respondeu Isaque. “Quem é você, Esaú ou Jacó?”
\textit{\tiny 19}
Jacó disse: “Sou Esaú, seu filho mais velho. Fiz o que o senhor mandou. Aqui
está a carne de caça. Sente-se e coma, para que me dê sua bênção”.
\textit{\tiny 20}
Isaque perguntou: “Como encontrou a caça tão depressa, meu filho?”.
   Jacó respondeu: “O SENHOR, seu Deus, a colocou no meu caminho”.
\textit{\tiny 21}
Então Isaque disse a Jacó: “Chegue mais perto, para que eu possa tocá-lo e ter
certeza de que você é mesmo Esaú”. 
\textit{\tiny 22}
Jacó se aproximou do pai, e Isaque o tocou
e disse: “A voz é de Jacó, mas as mãos são de Esaú”. 
\textit{\tiny 23}
Não o reconheceu, porém,
pois as mãos de Jacó estavam peludas, como as de Esaú. Assim, Isaque se preparou
para abençoar Jacó. 
\textit{\tiny 24}
“Mas você é mesmo meu filho Esaú?”, perguntou ele.
   “Sim, eu sou”, respondeu Jacó.
\textit{\tiny 25}
Então Isaque disse: “Agora, meu filho, traga-me a carne de caça. Depois que
comer, eu lhe darei a minha bênção”. Jacó trouxe a comida para o pai, e Isaque
comeu. Também bebeu o vinho que Jacó lhe serviu. 
\textit{\tiny 26}
Por fim, Isaque disse a Jacó:
“Aproxime-se, por favor, e dê-me um beijo, meu filho”.
   
\bigskip   
\textit{\tiny 27}
Jacó se aproximou e o beijou. Quando Isaque sentiu o cheiro das roupas,
finalmente abençoou o filho. Disse: “Ah! O cheiro de meu filho é como o cheiro
do campo que o SENHOR abençoou!
\textit{\tiny 28}
“Do orvalho do céu
    e da riqueza da terra,
  Deus lhe conceda fartas colheitas de cereais
    e vinho novo de sobra.
\textit{\tiny 29}
Que muitas nações o sirvam
    e se curvem à sua frente.
  Que você seja senhor de seus irmãos
    e os filhos de sua mãe se curvem à sua frente.
  Todos que o amaldiçoarem serão amaldiçoados,
    e todos que o abençoarem serão abençoados”.

\bigskip
\textit{\tiny 30}
Assim que Isaque terminou de abençoar Jacó, e logo depois de Jacó ter saído
da presença de seu pai, Esaú voltou da caçada. 
\textit{\tiny 31}
Preparou uma refeição saborosa,
levou-a para seu pai e disse: “Sente-se, meu pai, e coma da minha caça, para me
abençoar”.
\textit{\tiny 32}
Isaque lhe perguntou: “Quem é você?”.
   Ele respondeu: “Sou Esaú, seu filho mais velho”. 
\textit{\tiny 33}
Isaque começou a tremer incontrolavelmente e disse: “Então quem me serviu
a carne de caça? Acabei de comê-la, pouco antes de você chegar, e abençoei quem
a trouxe. Essa bênção deve permanecer!”.
   
\bigskip
\textit{\tiny 34}
Quando Esaú ouviu as palavras do pai, soltou um forte grito amargurado e
suplicou: “Ah, meu pai, e eu? Abençoe-me também!”.
\textit{\tiny 35}
Mas Isaque disse: “Seu irmão esteve aqui e me enganou. Levou embora a
bênção que pertencia a você!”.
   
\bigskip
\textit{\tiny 36}
Esaú exclamou: “Não é de admirar que ele se chame Jacó, pois é a segunda
vez que me engana. Primeiro, tomou meus direitos de filho mais velho e, agora,
roubou minha bênção. O senhor não guardou uma bênção sequer para mim?”.
\textit{\tiny 37}
Isaque disse a Esaú: “Fiz de Jacó o seu senhor e declarei que todos os irmãos
dele o servirão. Garanti a ele fartura de cereais e vinho. O que me resta para dar a
você, meu filho?”.
\textit{\tiny 38}
Esaú suplicou: “Por acaso o senhor tem apenas uma bênção? Ah, meu pai,
abençoe-me também!”. Então Esaú chorou em alta voz.
   
\bigskip
\textit{\tiny 39}
Por fim, seu pai, Isaque, lhe disse:
  “Você viverá longe das riquezas da terra
    e longe do orvalho do alto céu.
\textit{\tiny 40}
Viverá por sua espada
    e servirá a seu irmão.
  Quando, porém, conseguir se libertar,
    sacudirá do pescoço esse jugo”.
Jacó foge para Padã-Arã

\bigskip
\textit{\tiny 41}
Daquele momento em diante, Esaú passou a odiar Jacó porque seu pai o havia
abençoado. Começou a tramar: “Em breve meu pai morrerá. Então, matarei meu
irmão Jacó”.
\textit{\tiny 42}
Quando Rebeca soube das intenções de Esaú, mandou chamar Jacó e lhe
disse: “Ouça, Esaú se consola com planos para matar você. 
\textit{\tiny 43}
Portanto, preste
atenção, meu filho. Apronte-se e fuja para a casa de meu irmão Labão, em Harã.
\textit{\tiny 44}
Fique lá até que diminua a fúria de seu irmão. 
\textit{\tiny 45}
Quando ele se acalmar e se
esquecer do que você lhe fez, mandarei buscá-lo. Por que eu perderia meus dois
filhos no mesmo dia?”.

\bigskip
\textit{\tiny 46}
Depois, Rebeca disse a Isaque: “Estou cansada dessas mulheres hititas que
vivem aqui! Prefiro morrer a ver Jacó se casar com uma delas!”.
\textbf{\large 28}
 Então Isaque mandou chamar Jacó, o abençoou e disse: “Não se case com
uma mulher cananita. 
\textit{\tiny 2}
Em vez disso, vá de imediato a Padã-Arã, à casa de seu avô
Betuel, e case-se com uma das filhas de seu tio Labão. 
\textit{\tiny 3}
Que o Deus Todo-
poderoso o abençoe e lhe dê muitos filhos, e que eles se multipliquem e venham
a ser muitas nações. 
\textit{\tiny 4}
Que Deus dê a você e a seus descendentes as bênçãos que
ele prometeu a Abraão. Que você venha a possuir esta terra na qual vive agora
como estrangeiro, pois Deus entregou esta terra a Abraão”.

\bigskip
\textit{\tiny 5}
Assim, Isaque se despediu de Jacó, que foi a Padã-Arã morar com seu tio
Labão, irmão de Rebeca, filho de Betuel, o arameu.

\bigskip
\textit{\tiny 6}
Esaú soube que seu pai, Isaque, havia abençoado Jacó e o enviado a Padã-Arã
para encontrar uma esposa e que, ao abençoá-lo, tinha advertido a seu irmão:
“Não se case com uma mulher cananita”. 
\textit{\tiny 7}
Também soube que Jacó havia
obedecido aos pais e ido a Padã-Arã. 
\textit{\tiny 8}
Quando ficou evidente que seu pai não
aprovava as mulheres cananitas, 
\textit{\tiny 9}
Esaú foi visitar a família de seu tio Ismael e,
além das duas mulheres cananitas com as quais havia se casado, tomou para si
uma das filhas de Ismael. O nome de sua nova mulher era Maalate, irmã de
Nebaiote e filha de Ismael, filho de Abraão.

----------------------------------------------------------------------
