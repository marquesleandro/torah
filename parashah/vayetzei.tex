\section*{Parashat Vayetzei 28:10 - 32:3}

\subsubsection*{Jacó em Betel}

\textit{\tiny 10}
Nesse meio-tempo, Jacó partiu de Berseba e rumou para Harã. 
\textit{\tiny 11}
Quando o sol
se pôs, chegou a um bom local para acampar e ali passou a noite. Encontrou uma
pedra para descansar a cabeça e se deitou para dormir. 
\textit{\tiny 12}
Enquanto dormia,
sonhou com uma escada que ia da terra ao céu e viu os anjos de Deus, que subiam
e desciam pela escada.
\textit{\tiny 13}
No topo da escada estava o SENHOR, que lhe disse: “Eu sou o SENHOR, o Deus de
seu avô, Abraão, e o Deus de seu pai, Isaque. A terra na qual você está deitado lhe
pertence. Eu a darei a você e a seus descendentes. 
\textit{\tiny 14}
Seus descendentes serão tão
numerosos quanto o pó da terra! Eles se espalharão por todas as direções: leste e
oeste, norte e sul. E todas as famílias da terra serão abençoadas por seu
intermédio e de sua descendência. 
\textit{\tiny 15}
Além disso, estarei com você e o protegerei
aonde quer que vá. Um dia, trarei você de volta a esta terra. Não o deixarei
enquanto não tiver terminado de lhe dar tudo que prometi”.
\textit{\tiny 16}
Então Jacó acordou e disse: “Certamente o SENHOR está neste lugar, e eu não
havia percebido!”. 

\bigskip
\textit{\tiny 17}
Contudo, também teve medo e disse: “Como é temível este
lugar! Não é outro, senão a casa de Deus; é a porta para os céus!”.

\bigskip
\textit{\tiny 18}
Na manhã seguinte, Jacó se levantou bem cedo. Pegou a pedra na qual havia
descansado a cabeça, colocou-a em pé, como coluna memorial, e derramou azeite
de oliva sobre ela. 
\textit{\tiny 19}
Chamou o lugar de Betel, embora anteriormente se
chamasse Luz.
\textit{\tiny 20}
Então Jacó fez o seguinte voto: “Se, de fato, Deus for comigo e me proteger
nesta jornada, se ele me providenciar alimento e roupa, 
\textit{\tiny 21}
e se eu voltar são e salvo
à casa de meu pai, então o SENHOR certamente será o meu Deus. 
\textit{\tiny 22}
E esta coluna
memorial que eu levantei será um lugar de adoração a Deus, e eu entregarei a
Deus a décima parte de tudo que ele me der”.

\bigskip   
\subsubsection*{Jacó em Harã}
\textbf{\large 29}
 Jacó seguiu viagem e, por fim, chegou à terra do leste. 
\textit{\tiny 2} 
Viu um poço ao
longe e, junto ao poço, no campo, três rebanhos de ovelhas, à espera de que lhes
dessem água. Uma pedra pesada cobria a boca do poço.
\textit{\tiny 3} 
Era costume naquele lugar esperar que todos os rebanhos chegassem para,
então, remover a pedra e dar água aos animais. Depois, a pedra era recolocada na
boca do poço. 

\bigskip   
\textit{\tiny 4} 
Jacó se aproximou dos pastores e perguntou: “De onde vocês são,
amigos?”.
   “Somos de Harã”, disseram eles.
\textit{\tiny 5} 
“Conhecem um homem chamado Labão, neto de Naor?”, perguntou Jacó.
   “Sim, conhecemos”, responderam eles.
\textit{\tiny 6} 
“Ele vai bem?”, perguntou Jacó.
   “Sim, vai bem”, disseram. “Olhe, ali vem Raquel, filha dele, com o rebanho.”
\textit{\tiny 7} 
Jacó disse: “Ainda é dia claro, cedo demais para recolher os animais. Por que
vocês não dão de beber às ovelhas, para que elas possam voltar a pastar?”.
\textit{\tiny 8} 
“Não podemos dar de beber aos animais enquanto não chegarem todos os
rebanhos”, responderam. “Só então os pastores removem a pedra da boca do poço
e damos de beber a todas as ovelhas.”

\bigskip   
\textit{\tiny 9} 
Jacó ainda conversava com eles quando Raquel chegou com o rebanho de seu
pai, pois era pastora. 
\textit{\tiny 10}
Uma vez que Raquel era sua prima, filha de Labão, irmão
de sua mãe, e as ovelhas pertenciam a seu tio Labão, Jacó foi até o poço, removeu a
pedra que o cobria e deu de beber ao rebanho de seu tio. 
\textit{\tiny 11}
Então Jacó beijou
Raquel e chorou em alta voz. 
\textit{\tiny 12}
Explicou para Raquel que era seu primo por parte
do pai dela e filho de Rebeca, tia dela. Raquel foi correndo contar a seu pai, Labão.
\textit{\tiny 13}
Assim   que Labão soube que seu sobrinho Jacó havia chegado, correu ao seu
encontro. Ele o abraçou, o beijou e o levou para casa. Depois que Jacó lhe contou
sua história, 
\textit{\tiny 14}
Labão exclamou: “Você é, de fato, sangue do meu sangue!”.

\bigskip   
\subsubsection*{Jacó, Lia e Raquel}
Quando Jacó estava na casa de Labão havia cerca de um mês, 
\textit{\tiny 15}
Labão lhe disse:
“Você não deve trabalhar de graça para mim só porque somos parentes. Diga-me
qual deve ser o seu salário”.

\bigskip   
\textit{\tiny 16}
Labão tinha duas filhas. A mais velha se chamava Lia, e a mais nova, Raquel.
\textit{\tiny 17}
Os olhos de Lia eram sem brilho, mas Raquel tinha bela aparência e rosto
atraente. 
\textit{\tiny 18}
Visto que Jacó estava apaixonado por Raquel, disse a Labão:
“Trabalharei para o senhor por sete anos se me der Raquel, sua filha mais nova,
para ser minha esposa”.
\textit{\tiny 19}
“Melhor entregá-la a você do que a qualquer outro”, respondeu Labão. “Fique
aqui e trabalhe comigo.” 
\textit{\tiny 20}
Então Jacó trabalhou sete anos por Raquel. Ele a amava
tanto que lhe pareceram apenas alguns dias.

\bigskip   
\textit{\tiny 21}
Chegada a hora, Jacó disse a Labão. “Cumpri minha parte do acordo. Agora,
dê-me minha esposa, para que eu me deite com ela.”
\textit{\tiny 22}
Labão convidou toda a vizinhança e preparou uma grande festa de
casamento. 
\textit{\tiny 23}
À noite, porém, quando estava escuro, Labão tomou Lia e a entregou
a Jacó, e Jacó se deitou com ela. 
\textit{\tiny 24}
(Labão deu sua serva Zilpa a Lia para servi-la.)

\bigskip   
\textit{\tiny 25}
Na manhã seguinte, quando Jacó acordou, viu que era Lia. Então Jacó
perguntou a Labão: “O que o senhor fez comigo? Trabalhei sete anos por Raquel!
Por que o senhor me enganou?”.
\textit{\tiny 26}
Labão respondeu: “Aqui não é costume casar a filha mais nova antes da mais
velha. 
\textit{\tiny 27}
Espere, contudo, até terminar a semana de núpcias, e eu também lhe
entregarei Raquel, desde que você prometa trabalhar mais sete anos para mim”.

\bigskip   
\textit{\tiny 28}
Jacó concordou em trabalhar mais sete anos. Uma semana depois de Jacó ter
se casado com Lia, Labão lhe entregou Raquel. 
\textit{\tiny 29}
(Labão deu sua serva Bila a
Raquel para servi-la.) 
\textit{\tiny 30}
Jacó se deitou também com Raquel, a quem ele amava
muito mais que a Lia. Então permaneceu ali e trabalhou mais sete anos para
Labão.

\bigskip   
\subsubsection*{Jacó e os doze filhos}
\textit{\tiny 31}
Quando o SENHOR viu que Lia não era amada, permitiu que ela tivesse filhos;
Raquel, porém, era estéril. 

\bigskip   
\textit{\tiny 32}
Lia engravidou e deu à luz um filho. Chamou-o de
Rúben, pois disse: “O SENHOR viu minha infelicidade, e agora meu marido me
amará”.

\bigskip   
\textit{\tiny 33}
Pouco tempo depois, Lia engravidou novamente e deu à luz outro filho.
Chamou-o de Simeão, pois disse: “O SENHOR ouviu que eu não era amada e me
deu outro filho”.

\bigskip   
\textit{\tiny 34}
Lia engravidou pela terceira vez e deu à luz outro filho. Chamou-o de Levi,
pois disse: “Certamente, desta vez meu marido terá afeição por mim, pois lhe dei
três filhos!”.

\bigskip   
\textit{\tiny 35}
Lia engravidou mais uma vez e deu à luz outro filho. Chamou-o de Judá,
pois disse: “Agora louvarei ao SENHOR!”. Então, parou de ter filhos.

\bigskip   
\textbf{\large 30}
 Quando Raquel viu que não dava filhos a Jacó, teve inveja da irmã e
implorou a Jacó: “Dê-me filhos, ou morrerei!”.
\textit{\tiny 2} 
Jacó se enfureceu com Raquel. “Por acaso sou Deus?”, perguntou ele. “Foi ele
que não permitiu que você tivesse filhos!”
\textit{\tiny 3} 
Raquel lhe disse: “Tome minha serva Bila e deite-se com ela. Ela dará à luz
filhos em meu lugar e, por meio dela, também terei uma família”. 
\textit{\tiny 4} 
Então Raquel
entregou sua serva Bila a Jacó por mulher, e Jacó se deitou com ela. 
\textit{\tiny 5} 
Bila
engravidou e deu um filho a Jacó. 
\textit{\tiny 6} 
Raquel o chamou de Dã, pois disse: “Deus
me fez justiça! Ouviu meu pedido e me deu um filho!”. 

\bigskip   
\textit{\tiny 7} 
Bila engravidou
novamente e deu a Jacó o segundo filho. 
\textit{\tiny 8} 
Raquel o chamou de Naftali,
pois
disse: “Tive uma luta intensa com minha irmã e venci!”.

\bigskip   
\textit{\tiny 9} 
Quando Lia percebeu que tinha parado de engravidar, tomou sua serva Zilpa e
a entregou a Jacó por mulher. 
\textit{\tiny 10}
Pouco tempo depois, Zilpa deu um filho a Jacó.
\textit{\tiny 11}
Lia o chamou de Gade,
pois disse: “Como sou afortunada!”. 

\bigskip   
\textit{\tiny 12}
Então Zilpa deu
a Jacó o segundo filho. 
\textit{\tiny 13}
Lia o chamou de Aser,
pois disse: “Como estou alegre!
Agora as outras mulheres celebrarão comigo”.

\bigskip   
\textit{\tiny 14}
Certo dia, durante a colheita do trigo, Rúben encontrou algumas
mandrágoras que cresciam no campo e as trouxe para Lia, sua mãe. Raquel
suplicou a Lia: “Por favor, dê-me algumas das mandrágoras de seu filho”.
\textit{\tiny 15}
Lia, porém, respondeu: “Não basta ter roubado meu marido? Agora também
quer roubar as mandrágoras de meu filho?”.
   Raquel propôs: “Em troca de algumas mandrágoras, deixarei que Jacó se deite
com você esta noite”.
\textit{\tiny 16}
Ao   entardecer, quando Jacó estava voltando do campo, Lia foi ao seu
encontro e disse: “Esta noite você deve se deitar comigo. Paguei por você com
algumas mandrágoras que meu filho encontrou”. Assim, naquela noite Jacó se
deitou com Lia. 
\textit{\tiny 17}
Deus respondeu às orações de Lia, que engravidou novamente e
deu a Jacó o quinto filho. 
\textit{\tiny 18}
Chamou-o de Issacar,
pois disse: “Deus me
recompensou porque entreguei minha serva por mulher a meu marido”. 

\bigskip   
\textit{\tiny 19}
Lia
engravidou outra vez e deu a Jacó o sexto filho. 
\textit{\tiny 20}
Chamou-o de Zebulom,
pois
disse: “Deus me deu uma boa recompensa. Agora meu marido me tratará com
respeito, porque lhe dei seis filhos”. 

\bigskip   
\textit{\tiny 21}
Depois, Lia deu à luz uma filha e a chamou
de Diná.

\bigskip   
\textit{\tiny 22}
Então Deus se lembrou de Raquel e, em resposta a suas orações, permitiu que
ela se tornasse fértil. 
\textit{\tiny 23}
Ela engravidou e deu à luz um filho. “Deus tirou a minha
humilhação”, declarou, 
\textit{\tiny 24}
e o chamou de José,
pois disse: “Que o SENHOR me
acrescente ainda outro filho!”.

\bigskip   
\subsubsection*{Jacó e Labão}
\textit{\tiny 25}
Logo depois que Raquel deu à luz José, Jacó disse a Labão: “Por favor, libere-me
para que eu volte à minha terra natal. 
\textit{\tiny 26}
Permita-me levar minhas mulheres e
meus filhos, pelos quais o servi, e deixe-me partir. O senhor sabe muito bem
como trabalhei arduamente a seu serviço”.

\bigskip   
\textit{\tiny 27}
Labão respondeu: “Se mereço seu favor, fique. Eu enriqueci, pois
o SENHOR
me abençoou por sua causa. 
\textit{\tiny 28}
Diga-me qual será seu salário e, qualquer que seja o
valor, eu lhe pagarei”.
\textit{\tiny 29}
Jacó respondeu: “O senhor sabe como trabalhei arduamente a seu serviço e
como seus rebanhos cresceram sob meus cuidados. 
\textit{\tiny 30}
De fato, o senhor tinha
pouco antes de eu chegar, mas sua riqueza aumentou consideravelmente. O
SENHOR o abençoou por meio de tudo que eu fiz. Mas e quanto a mim? Quando
começarei a cuidar de minha própria família?”.

\bigskip   
\textit{\tiny 31}
“Quanto quer receber de salário?”, perguntou Labão mais uma vez.
   Jacó respondeu: “Não me dê coisa alguma. Se o senhor fizer o que lhe direi,
continuarei a cuidar de seus rebanhos: 
\textit{\tiny 32}
deixe-me inspecionar seus rebanhos
hoje e remover todas as ovelhas e cabras salpicadas e malhadas, além de todas as
ovelhas pretas. Elas serão o meu salário. 
\textit{\tiny 33}
No futuro, quando o senhor conferir os
animais que me deu como salário, verá que fui honesto. Se encontrar em meu
rebanho alguma cabra que não seja salpicada ou malhada, ou alguma ovelha que
não seja preta, saberá que as roubei do senhor”.

\bigskip   
\textit{\tiny 34}
“Está bem”, respondeu Labão. “Será como você diz.” 

\bigskip   
\textit{\tiny 35}
Naquele mesmo dia,
porém, Labão saiu e tirou do rebanho todos os bodes listrados e malhados, todas
as cabras salpicadas e malhadas ou com manchas brancas, e todas as ovelhas
pretas. Colocou os animais sob os cuidados de seus filhos, 
\textit{\tiny 36}
que os levaram a um
lugar a três dias de viagem de onde Jacó estava. Assim, Jacó ficou e tomou conta
do resto do rebanho de Labão.

\bigskip   
\textit{\tiny 37}
Então Jacó pegou alguns galhos verdes de álamo, amendoeira e plátano e
removeu tiras das cascas, formando listras brancas nos galhos. 
\textit{\tiny 38}
Em seguida,
colocou os galhos descascados junto aos bebedouros onde os rebanhos iam beber
água, pois era ali que se acasalavam. 
\textit{\tiny 39}
Quando se acasalavam diante desses galhos
descascados com listras brancas, davam crias listradas, salpicadas e malhadas.
\textit{\tiny 40}
Jacó separava esses cordeiros do rebanho de Labão. Na época do cio, colocava o
rebanho de frente para os animais listrados e pretos de Labão. Assim, Jacó foi
formando seu próprio rebanho, que mantinha separado do de Labão.
\textit{\tiny 41}
Sempre que as fêmeas mais fortes estavam no cio, Jacó colocava os galhos
descascados nos bebedouros em frente delas, para que se acasalassem diante dos
galhos. 
\textit{\tiny 42}
Não fazia o mesmo, porém, com as fêmeas mais fracas, de modo que as
crias mais fracas ficavam com Labão, e as mais fortes, com Jacó. 
\textit{\tiny 43}
O resultado foi
que Jacó se tornou muito rico, dono de grandes rebanhos e também de servos e
servas e muitos camelos e jumentos.

\bigskip   
\textbf{\large 31}
 Logo, porém, Jacó percebeu que os filhos de Labão estavam reclamando
dele: “Jacó roubou tudo que era de nosso pai! À custa de nosso pai, adquiriu toda a
sua riqueza!”. 
\textit{\tiny 2} 
Jacó também começou a notar uma mudança na atitude de Labão
para com ele.
\textit{\tiny 3} 
Então o SENHOR disse a Jacó: “Volte para a terra de seu pai e de seu avô, a terra
de seus parentes, e eu estarei com você”.

\bigskip   
\textit{\tiny 4} 
Jacó mandou chamar Raquel e Lia ao campo onde ele cuidava de seus
rebanhos 
\textit{\tiny 5} 
e disse a elas: “Notei que seu pai mudou de atitude em relação a mim.
O Deus de meu pai, porém, tem estado comigo. 
\textit{\tiny 6} 
Vocês sabem como tenho
trabalhado arduamente a serviço de seu pai. 
\textit{\tiny 7} 
Contudo, ele me enganou e mudou
meu salário dez vezes. Mas Deus não permitiu que ele me prejudicasse. 
\textit{\tiny 8} 
Se ele
dizia: ‘Os salpicados serão o seu salário’, o rebanho começava a dar crias
salpicadas. E, quando mudava de ideia e dizia: ‘Os listrados serão o seu salário’,
então o rebanho inteiro dava crias listradas. 
\textit{\tiny 9} 
Desse modo, Deus tirou os animais
de seu pai e os deu a mim.
\textit{\tiny 10}
“Certa vez, na época do acasalamento, tive um sonho e vi que os bodes que se
acasalavam com as cabras eram listrados, salpicados e malhados. 
\textit{\tiny 11}
Então, em
meu sonho, o anjo de Deus me disse: ‘Jacó!’. E eu respondi: ‘Aqui estou!’.
\textit{\tiny 12}
“E o anjo disse: ‘Levante os olhos e você verá que apenas os machos listrados,
salpicados e malhados estão se acasalando com as fêmeas de seu rebanho, pois
vejo como Labão tem tratado você. 
\textit{\tiny 13}
Eu sou o Deus que lhe apareceu em Betel,
o lugar onde você ungiu a coluna de pedra e fez seu voto a mim. Agora, apronte-
se, saia desta terra e volte à sua terra natal’”.

\bigskip   
\textit{\tiny 14}
Raquel e Lia responderam: “Da nossa parte, tudo bem! Afinal, não
herdaremos coisa alguma da riqueza de nosso pai. 
\textit{\tiny 15}
Ele reduziu nossos direitos
aos mesmos que têm as mulheres estrangeiras. Depois que nos vendeu,
desperdiçou todo o dinheiro que você pagou por nós. 
\textit{\tiny 16}
Toda a riqueza que Deus
tirou de nosso pai e deu a você é, por direito, nossa e de nossos filhos. Por isso,
faça o que Deus ordenou”.

\bigskip   
\textit{\tiny 17}
Então Jacó montou suas mulheres e seus filhos em camelos 
\textit{\tiny 18}
e conduziu
adiante todos os seus rebanhos. Juntou todos os bens que havia adquirido em
Padã-Arã e partiu para a terra de Canaã, onde vivia Isaque, seu pai. 

\bigskip   
\textit{\tiny 19}
Quando
partiram, Labão estava num lugar afastado, tosquiando suas ovelhas. Raquel
roubou os ídolos da casa que pertenciam a seu pai e os levou consigo. 
\textit{\tiny 20}
Assim,
Jacó enganou Labão, o arameu, partindo sem avisá-lo de que iam embora. 
\textit{\tiny 21}
Jacó
levou todos os seus bens e atravessou o rio Eufrates,
rumo à região montanhosa
de Gileade.

\bigskip   
\textit{\tiny 22}
Três dias depois, Labão foi informado de que Jacó havia fugido. 
\textit{\tiny 23}
Reuniu um
grupo de parentes e saiu em perseguição a Jacó. Sete dias depois o alcançou, na
região montanhosa de Gileade. 
\textit{\tiny 24}
Na noite anterior, porém, Deus havia aparecido
em sonho a Labão, o arameu, e dito a ele: “Estou avisando: deixe Jacó em paz!”.

\bigskip   
\textit{\tiny 25}
Labão o alcançou enquanto Jacó estava acampado na região montanhosa de
Gileade e armou seu acampamento ali perto. 
\textit{\tiny 26}
“O que você fez?”, perguntou
Labão. “Como ousou me enganar e levar minhas filhas embora, como se fossem
prisioneiras de guerra? 
\textit{\tiny 27}
Por que fugiu em segredo? Por que me enganou? E por
que não avisou que desejava partir? Eu lhe teria dado uma festa de despedida,
com canções e música, ao som de tamborins e harpas. 
\textit{\tiny 28}
Por que não me deixou beijar minhas filhas e meus netos e me despedir deles? Você agiu de forma
extremamente tola! 
\textit{\tiny 29}
Eu poderia destruí-lo, mas o Deus de seu pai me apareceu
ontem à noite e me advertiu: ‘Deixe Jacó em paz!’. 
\textit{\tiny 30}
Entendo sua vontade de
partir e seu desejo de voltar à casa de seu pai. Mas por que roubou meus deuses?”

\bigskip   
\textit{\tiny 31}
Jacó respondeu: “Fugi porque tive medo. Pensei que o senhor tiraria suas
filhas de mim à força. 
\textit{\tiny 32}
Quanto a seus deuses, veja se consegue encontrá-los, e
quem os tiver roubado deve morrer! Se encontrar qualquer outra coisa que lhe
pertença, identifique-a diante de todos estes nossos parentes, e eu a devolverei”.
Jacó, porém, não sabia que Raquel havia roubado os ídolos da casa.

\bigskip   
\textit{\tiny 33}
Labão foi procurar primeiro na tenda de Jacó e, depois, nas tendas de Lia e
das duas servas, mas nada encontrou. Por fim, entrou na tenda de Raquel.
\textit{\tiny 34}
Acontece que Raquel havia pego os ídolos da casa e os escondido na sela do seu
camelo, e estava sentada em cima deles. Quando Labão terminou de vasculhar
toda a sua tenda sem encontrar os ídolos, 
\textit{\tiny 35}
Raquel disse ao pai: “Por favor,
perdoe-me por não me levantar para o senhor, mas estou em meu período
menstrual”. Labão continuou a busca, mas não encontrou os ídolos da casa.

\bigskip   
\textit{\tiny 36}
Jacó se enfureceu e discutiu com Labão. “Qual foi o meu crime?”, perguntou
ele. “O que fiz de errado para o senhor me perseguir como se eu fosse um
criminoso? 
\textit{\tiny 37}
O senhor vasculhou todos os meus bens. Por acaso encontrou algum
objeto que lhe pertença? Coloque-o aqui, diante de nossos parentes, para que
todos vejam. Que eles julguem entre nós dois!
\textit{\tiny 38}
“Estive vinte anos com o senhor, cuidando de seus rebanhos. Ao longo de
todo esse tempo, suas ovelhas e cabras nunca abortaram. Não me servi de um
carneiro sequer de seu rebanho para alimento. 
\textit{\tiny 39}
Quando algum deles era
despedaçado por um animal selvagem, eu nunca lhe mostrava a carcaça. Não, eu
mesmo arcava com o prejuízo! O senhor me obrigava a pagar por todo animal
roubado, quer à plena luz do dia, quer na escuridão da noite.
\textit{\tiny 40}
“Trabalhei para o senhor em dias de calor escaldante e também em noites
frias e insones. 
\textit{\tiny 41}
Sim, por vinte anos trabalhei feito um escravo em sua casa!
Trabalhei catorze anos por suas duas filhas e, depois, mais seis anos para formar
meu rebanho. E dez vezes o senhor mudou meu salário. 
\textit{\tiny 42}
De fato, se o Deus de
meu pai não estivesse comigo, o Deus de Abraão e o Deus temível de Isaque,
o
senhor teria me mandado embora de mãos vazias. Mas Deus viu como fui
maltratado, apesar de meu árduo trabalho. Por isso ele lhe apareceu na noite
passada e o repreendeu!”

\bigskip   
\textit{\tiny 43}
Labão respondeu a Jacó: “Essas mulheres são minhas filhas, as crianças são
meus netos, e os rebanhos são meus rebanhos. Na verdade, tudo que você vê é
meu. Mas o que posso fazer agora por minhas filhas e pelos filhos delas?
\textit{\tiny 44}
Façamos, portanto, você e eu, uma aliança que sirva de testemunho do nosso
compromisso”.

\bigskip   
\textit{\tiny 45}
Então Jacó pegou uma pedra e a colocou em pé como monumento. 
\textit{\tiny 46}
Em
seguida, disse aos membros de sua família: “Juntem algumas pedras”. Eles
pegaram as pedras e as amontoaram. Jacó e Labão se sentaram perto do monte de
pedras e fizeram uma refeição para selar a aliança. 
\textit{\tiny 47}
A fim de comemorar a
ocasião, Labão chamou o lugar de Jegar-Saaduta, e Jacó o chamou de Galeede.

\bigskip  
\textit{\tiny 48}
Labão declarou: “Este monte de pedras servirá de testemunha para nos
lembrar da aliança que fizemos hoje”. Isso explica por que o lugar foi chamado de
Galeede. 
\textit{\tiny 49}
Também foi chamado de Mispá,
\textit{\tiny 12}
 pois Labão disse: “Vigie o SENHOR a
você e a mim para garantir que guardaremos esta aliança quando estivermos
longe um do outro. 
\textit{\tiny 50}
Se você maltratar minhas filhas ou se casar com outras
mulheres, mesmo que ninguém mais veja, Deus verá. Ele é testemunha desta
aliança entre nós.”
\textit{\tiny 51}
“Veja este monte de pedras”, prosseguiu Labão. “Veja também este
monumento que levantei entre nós dois. 
\textit{\tiny 52}
Estão entre mim e você como
testemunhas dos nossos votos. Nunca atravessarei para o lado de lá do monte de
pedras a fim de prejudicá-lo, e você jamais deve atravessar para o lado de cá a fim
de prejudicar-me. 
\textit{\tiny 53}
Invoco o Deus de nossos antepassados, o Deus de seu avô
Abraão e o Deus de meu avô Naor, para que sirva de juiz entre nós.”

\bigskip  
   Assim, diante do Deus temível de seu pai Isaque,
\textit{\tiny 13}
 Jacó jurou respeitar a linha
divisória. 
\textit{\tiny 54}
Então Jacó ofereceu sacrifício a Deus na montanha e convidou todos
para a refeição comemorativa. Depois de comerem, passaram a noite ali.
\textit{\tiny 55}
 Na manhã seguinte, Labão se levantou cedo, beijou seus netos e suas filhas
e os abençoou. Depois, partiu e voltou para casa.

\bigskip
\subsubsection*{Jacó e a luta com Deus}
\textbf{\large 32}
              Quando Jacó seguiu viagem, anjos de Deus vieram encontrar-se com
ele. 
\textit{\tiny 2} 
Ao vê-los, Jacó exclamou: “Este é o acampamento de Deus!”. Por isso, chamou
o lugar de Maanaim.
\textit{\tiny 3} 
Então Jacó enviou adiante dele mensageiros a seu irmão Esaú, que vivia na região
de Seir, na terra de Edom. 

----------------------------------------------------------------------
