\section*{Parashat Chayyei Sarah 23:1 - 25:18}

\subsubsection*{Morte de Sara}

\textbf{\large 23}
 Quando Sara estava com 127 anos, 
\textit{\tiny 2}
morreu em Quiriate-Arba (hoje
chamada Hebrom), na terra de Canaã. Abraão lamentou a morte de Sara e chorou
por ela.

\bigskip
\textit{\tiny 3}
Depois, deixou ali o corpo de sua mulher e disse aos hititas: 
\textit{\tiny 4}
“Tenho vivido
como forasteiro e estrangeiro entre vocês. Por favor, vendam-me um pedaço de
terra, para que eu possa dar um sepultamento digno à minha mulher”.
\textit{\tiny 5}
Os hititas responderam a Abraão: 
\textit{\tiny 6}
“Ouça-nos; o senhor é um príncipe
honrado em nosso meio. Escolha o melhor dos nossos túmulos e nele sepulte sua
mulher. Nenhum de nós se recusará a dar ao senhor o local para a sepultura”.
\textit{\tiny 7}
Abraão curvou-se diante dos hititas 
\textit{\tiny 8}
e disse: “Visto que estão dispostos a me
dar o local para a sepultura, façam a gentileza de pedir a Efrom, filho de Zoar,
\textit{\tiny 9}
que me permita comprar sua caverna em Macpela, na fronteira do seu campo.
Ele me venderá a terra pelo preço que vocês considerarem justo, e assim terei uma
sepultura permanente para minha família”.

\bigskip
\textit{\tiny 10}
Efrom estava sentado no meio do seu povo e respondeu a Abraão enquanto
os demais ouviam, pronunciando-se publicamente diante dos hititas que se
reuniam à porta da cidade. 
\textit{\tiny 11}
“Não, meu senhor”, disse ele a Abraão. “Ouça-me; eu
lhe dou o campo e a caverna. Aqui, na presença do meu povo, eu lhe dou a
propriedade. Vá e sepulte a sua falecida.”
\textit{\tiny 12}
Abraão se curvou outra vez diante do povo daquela terra 
\textit{\tiny 13}
e respondeu a
Efrom, enquanto todos ouviam: “Ouça-me, por favor; eu os comprarei de você.
Deixe-me pagar o preço justo pelo campo, para que possa sepultar ali a minha
falecida”.
\textit{\tiny 14}
Efrom respondeu a Abraão: 
\textit{\tiny 15}
“Meu senhor, ouça-me; a propriedade vale
quatrocentas peças de prata, mas o que é isso entre amigos? Vá e sepulte a sua
falecida”.
\textit{\tiny 16}
Abraão concordou com o preço e pagou a quantia que Efrom sugeriu:
quatrocentas peças de prata, pesadas de acordo com o padrão do mercado. E os
hititas testemunharam a transação.

\bigskip
\textit{\tiny 17}
Assim, Abraão comprou o pedaço de terra pertencente a Efrom em Macpela,
perto de Manre. A propriedade incluía o campo, a caverna e todas as árvores ao
redor. 
\textit{\tiny 18}
Foi transferida a Abraão como sua propriedade permanente, na presença
dos anciãos hititas à porta da cidade. 
\textit{\tiny 19}
Então Abraão sepultou Sara, sua mulher,
em Canaã, na caverna de Macpela, perto de Manre (também chamado Hebrom).
\textit{\tiny 20}
O campo e a caverna foram transferidos dos hititas para Abraão como sepultura
permanente.

\bigskip
\subsubsection*{Isaque e Rebeca}
\textbf{\large 24}
 Abraão estava bem velho, e o SENHOR o havia abençoado em tudo. 
\textit{\tiny 2}
Certo
dia, Abraão disse a seu servo mais antigo, o homem encarregado de sua casa:
“Faça um juramento colocando a mão debaixo da minha coxa. 
\textit{\tiny 3}
Jure pelo SENHOR,
o Deus dos céus e da terra, que não deixará meu filho se casar com uma das
mulheres cananitas que aqui vivem, 
\textit{\tiny 4}
mas irá à minha terra natal, aos meus
parentes, procurar uma mulher para meu filho Isaque”.
\textit{\tiny 5}
O servo perguntou: “E se eu não encontrar uma moça disposta a viajar para um
lugar tão distante de sua terra? Devo levar Isaque para morar entre seus parentes
na terra de onde o senhor veio?”.
\textit{\tiny 6}
“Não!”, respondeu Abraão. “Cuidado! Não leve meu filho para lá de jeito
nenhum. 
\textit{\tiny 7}
O SENHOR, o Deus dos céus, que me tirou da casa de meu pai e de minha
terra natal, prometeu solenemente dar esta terra a meus descendentes. Ele
enviará um anjo à sua frente e providenciará para que você encontre ali uma
mulher para meu filho. 
\textit{\tiny 8}
Se ela não estiver disposta a acompanhá-lo de volta, você
estará livre do seu juramento. Mas não leve meu filho para lá, de maneira
nenhuma.”
\textit{\tiny 9}
Então o servo colocou a mão debaixo da coxa de Abraão, seu senhor, e jurou
seguir suas instruções. 

\bigskip
\textit{\tiny 10}
Em seguida, pegou dez camelos de Abraão, carregou-os
com presentes valiosos de todo tipo da parte de seu senhor e viajou para a terra
distante de Arã-Naaraim. Chegando lá, dirigiu-se à cidade onde Naor, irmão de
Abraão, havia se estabelecido. 
\textit{\tiny 11}
Ao entardecer, quando as mulheres saíam para
tirar água, ele fez os camelos se ajoelharem perto de um poço nos arredores da
cidade.

\bigskip
\textit{\tiny 12}
Então o servo orou: “Ó SENHOR, Deus do meu senhor Abraão, por favor, dá-me
sucesso hoje e sê bondoso com o meu senhor Abraão. 
\textit{\tiny 13}
Como vês, estou aqui
junto desta fonte, e as moças da cidade estão vindo tirar água. 
\textit{\tiny 14}
Esta é minha
súplica. Pedirei a uma delas: ‘Por favor, dê-me um pouco de água do seu cântaro
para eu beber’. Se ela disser: ‘Sim, beba. Também darei água a seus camelos’, que
seja ela a moça que escolheste para ser mulher do teu servo Isaque. Desse modo,
saberei que foste bondoso com o meu senhor”.

\bigskip
\textit{\tiny 15}
Antes de terminar a oração, o servo viu aproximar-se uma moça chamada
Rebeca, que trazia um cântaro no ombro. Ela era filha de Betuel, filho do irmão de
Abraão, Naor, e de sua mulher, Milca. 
\textit{\tiny 16}
Rebeca era muito bonita, tinha idade para
casar e era virgem. Ela desceu à fonte, encheu o cântaro e voltou. 
\textit{\tiny 17}
O servo de
Abraão correu até ela e lhe pediu: “Por favor, dê-me um pouco de água do seu
cântaro para eu beber”.
\textit{\tiny 18}
“Sim, meu senhor, beba”, respondeu ela e, prontamente, baixou o cântaro do
ombro e lhe deu de beber. 
\textit{\tiny 19}
Depois que lhe deu de beber, disse: “Tirarei água
para seus camelos também, até que estejam satisfeitos”. 
\textit{\tiny 20}
Esvaziou depressa o
cântaro no bebedouro e correu de volta ao poço a fim de tirar água para todos os
camelos.
\textit{\tiny 21}
O homem a observou em silêncio, pensando se o SENHOR lhe tinha dado
sucesso em sua missão. 

\bigskip
\textit{\tiny 22}
Por fim, quando os camelos terminaram de beber, o
servo deu à moça uma argola de ouro para o nariz e duas pulseiras grandes de
ouro para os braços.
\textit{\tiny 23}
“De quem você é filha?”, perguntou ele. “Diga-me, por favor, se seu pai tem
lugar para nos hospedar esta noite.”
\textit{\tiny 24}
“Sou filha de Betuel, e meus avós são Naor e Milca”, respondeu ela. 
\textit{\tiny 25}
“Temos
bastante palha e forragem para os camelos e espaço para hóspedes.”
\textit{\tiny 26}
O homem se prostrou e adorou o SENHOR. 
\textit{\tiny 27}
“Louvado seja o SENHOR, Deus do
meu senhor Abraão!”, disse ele. “O SENHOR demonstrou bondade e fidelidade ao
meu senhor, pois me conduziu até seus parentes.”
   
\bigskip
\textit{\tiny 28}
A moça correu para casa e contou à família tudo que havia acontecido.
\textit{\tiny 29}
Rebeca tinha um irmão chamado Labão, que foi prontamente à fonte para
conhecer o homem. 
\textit{\tiny 30}
Ele havia visto a argola para o nariz e as pulseiras nos
braços da irmã, e tinha ouvido Rebeca contar o que o homem dissera. Assim,
apressou-se até a fonte, onde o homem ainda estava parado perto dos camelos.
\textit{\tiny 31}
Labão lhe disse: “Venha e fique conosco, abençoado do SENHOR! Por que ficar aí
fora? Já mandei arrumar acomodações para você e seus homens e lugar para os
camelos”.
\textit{\tiny 32}
Então o homem foi com ele para casa. Labão mandou descarregar os
camelos, dar palha para os animais se deitarem e forragem para comerem, e água
para o homem e seus ajudantes lavarem os pés. 

\bigskip
\textit{\tiny 33}
Quando a refeição foi servida,
porém, o servo de Abraão disse: “Não comerei enquanto não explicar o motivo da
minha vinda”.
   “Está bem”, disse Labão. “Fale.”
\textit{\tiny 34}
“Sou servo de Abraão”, explicou ele. 
\textit{\tiny 35}
“O SENHOR abençoou grandemente o
meu senhor, e ele se tornou um homem rico. O SENHOR lhe deu rebanhos de
ovelhas e bois, uma fortuna em prata e ouro, e muitos servos e servas, camelos e
jumentos.
\textit{\tiny 36}
“Quando Sara, mulher do meu senhor, era muito idosa, deu à luz o filho dele.
Meu senhor deu tudo que possui a esse filho 
\textit{\tiny 37}
e me fez jurar, dizendo: ‘Não
permita que meu filho se case com uma das mulheres cananitas que aqui vivem.
\textit{\tiny 38}
Vá à casa de meu pai, aos meus parentes, procurar uma mulher para meu filho’.
\textit{\tiny 39}
“Mas eu perguntei ao meu senhor: ‘E se eu não encontrar uma moça disposta
a voltar comigo?’. 
\textit{\tiny 40}
Ele respondeu: ‘O SENHOR, em cuja presença tenho vivido,
enviará um anjo com você e lhe dará sucesso em sua missão. Encontre uma
mulher para meu filho entre os meus parentes, da família de meu pai. 
\textit{\tiny 41}
Então
você terá cumprido sua obrigação. Se, porém, você for aos meus parentes e eles
não deixarem a moça acompanhá-lo, estará livre do juramento’.
\textit{\tiny 42}
“Hoje, quando cheguei à fonte, fiz a seguinte oração: ‘Ó SENHOR, Deus do meu
senhor Abraão, por favor, dá-me sucesso em minha missão. 
\textit{\tiny 43}
Como vês, estou
aqui junto desta fonte. Esta é minha súplica. Quando uma jovem vier tirar água, eu
lhe direi: ‘Por favor, dê-me um pouco de água do seu cântaro’. 
\textit{\tiny 44}
Se ela disser:
‘Sim, beba. Também darei água a seus camelos’, que seja ela a moça que
escolheste para ser mulher do filho do meu senhor’.
\textit{\tiny 45}
“Antes de terminar de orar em meu coração, vi Rebeca vindo com o cântaro
no ombro. Ela desceu à fonte e tirou água. Eu lhe disse: ‘Por favor, dê-me um
pouco de água do seu cântaro para eu beber’. 
\textit{\tiny 46}
Prontamente, ela baixou o cântaro
do ombro e disse: ‘Sim, beba. Também darei água aos seus camelos’. Eu bebi, e ela
deu água aos camelos.
\textit{\tiny 47}
“Em seguida, perguntei-lhe: ‘De quem você é filha?’. Ela respondeu: ‘Sou filha
de Betuel, e meus avós são Naor e Milca’. Então coloquei a argola em seu nariz e as
pulseiras em seus braços.
\textit{\tiny 48}
“Depois, prostrei-me e adorei o SENHOR. Louvei o SENHOR, Deus do meu
senhor Abraão, pois ele havia me conduzido até a sobrinha-neta do meu senhor,
para que ela seja mulher do filho do meu senhor. 
\textit{\tiny 49}
Agora, digam-me se
mostrarão bondade e fidelidade ao meu senhor. Por favor, respondam-me ‘sim’
ou ‘não’, para que eu saiba o que fazer em seguida.”

\bigskip
\textit{\tiny 50}
Labão e Betuel responderam: “É evidente que o SENHOR o trouxe até aqui.
Sendo assim, não há nada que possamos dizer. 
\textit{\tiny 51}
Aqui está Rebeca; tome-a e leve-
a com você. Que ela seja mulher do filho do seu senhor, como disse o SENHOR”.

\bigskip
\textit{\tiny 52}
Quando o servo de Abraão ouviu a resposta, prostrou-se no chão e adorou o
SENHOR. 
\textit{\tiny 53}
Em seguida, entregou a Rebeca joias de prata e ouro e vestidos.
Também deu presentes valiosos ao irmão e à mãe de Rebeca. 

\bigskip
\textit{\tiny 54}
Então o servo e os
homens que o acompanhavam comeram e passaram a noite ali.
   Logo cedo na manhã seguinte, o servo de Abraão disse: “Enviem-me de volta ao
meu senhor”.
\textit{\tiny 55}
Mas o irmão e a mãe de Rebeca disseram: “Queremos que Rebeca fique
conosco pelo menos dez dias; depois, ela poderá partir”.
\textit{\tiny 56}
O servo, porém, disse: “Não me detenham. O SENHOR me deu sucesso em
minha missão; agora, enviem-me de volta ao meu senhor”.
\textit{\tiny 57}
“Pois bem”, disseram eles. “Chamaremos Rebeca e pediremos a opinião
dela.” 
\textit{\tiny 58}
Chamaram Rebeca e lhe perguntaram: “Você está disposta a ir com este
homem?”.
   E ela respondeu: “Sim, estou”.
\textit{\tiny 59}
Com isso, eles se despediram de Rebeca e a enviaram com o servo de Abraão
e seus homens. A serva que havia amamentado Rebeca a acompanhou. 
\textit{\tiny 60}
Na hora
da partida, abençoaram Rebeca, dizendo:
  “Nossa irmã, que você se torne
    mãe de muitos milhares!
  Que seus descendentes
    conquistem as cidades de seus inimigos!”.
\textit{\tiny 61}
Então Rebeca e suas servas montaram nos camelos e seguiram o homem. Assim,
o servo de Abraão partiu levando Rebeca.

\bigskip
\textit{\tiny 62}
Nesse meio-tempo, Isaque, que morava no Neguebe, havia regressado de
Beer-Laai-Roi. 
\textit{\tiny 63}
Ao entardecer, enquanto caminhava pelo campo e meditava,
levantou os olhos e viu que camelos se aproximavam. 
\textit{\tiny 64}
Quando Rebeca levantou
os olhos e viu Isaque, desceu do camelo no mesmo instante 
\textit{\tiny 65}
e perguntou ao
servo: “Quem é aquele homem que vem pelo campo ao nosso encontro?”.
   Quando ele respondeu: “É meu senhor”, Rebeca cobriu o rosto com o véu.
\textit{\tiny 66}
Depois, o servo contou a Isaque tudo que havia feito.
\textit{\tiny 67}
Isaque a levou para a tenda de Sara, sua mãe, e Rebeca se tornou sua mulher.
Ele a amava profundamente e nela encontrou consolação depois que sua mãe
morreu.
A morte de Abraão

\bigskip   
\subsubsection*{Morte de Abraão}
\textbf{\large 25}
 Abraão se casou outra vez, com uma mulher chamada Quetura. 
\textit{\tiny 2}
Ela deu à
luz Zinrã, Jocsã, Medã, Midiã, Isbaque e Suá. 
\textit{\tiny 3}
Jocsã gerou Sabá e Dedã. Os
descendentes de Dedã foram os assuritas, os letusitas e os leumitas. 
\textit{\tiny 4}
Os filhos de
Midiã foram Efá, Éfer, Enoque, Abida e Elda. Todos eles foram descendentes de
Abraão por meio de Quetura.

\bigskip   
\textit{\tiny 5}
Abraão deu tudo que possuía a seu filho Isaque. 
\textit{\tiny 6}
Antes de morrer, porém, deu
presentes aos filhos de suas concubinas e os separou de Isaque, enviando-os para
as terras do leste.
\textit{\tiny 7}
Abraão viveu 175 anos 
\textit{\tiny 8}
e morreu em boa velhice, depois de uma vida longa e
feliz. Deu o último suspiro e, ao morrer, reuniu-se a seus antepassados. 

\bigskip   
\textit{\tiny 9}
Seus
filhos Isaque e Ismael o sepultaram na caverna de Macpela, perto de Manre, no
campo de Efrom, filho de Zoar, o hitita. 
\textit{\tiny 10}
Esse era o campo que Abraão havia
comprado dos hititas e onde havia sepultado Sara, sua mulher. 
\textit{\tiny 11}
Depois da morte
de Abraão, Deus abençoou Isaque, e ele se estabeleceu perto de Beer-Laai-Roi, no
Neguebe.
Os descendentes de Ismael

\bigskip
\subsubsection*{Descendentes de Ismael}
\textit{\tiny 12}
Este é o relato da família de Ismael, filho de Abraão com Hagar, serva egípcia de
Sara. 
\textit{\tiny 13}
São estes os descendentes de Ismael por nome e clã: Nebaiote, o mais
velho, seguido de Quedar, Adbeel, Misbão, 
\textit{\tiny 14}
Misma, Dumá, Massá, 
\textit{\tiny 15}
Hadade,
Temá, Jetur, Nafis e Quedemá. 
\textit{\tiny 16}
Esses doze filhos de Ismael deram origem a doze
tribos, cada uma com o nome de seu fundador, relacionadas de acordo com o
lugar onde se estabeleceram e acamparam. 
\textit{\tiny 17}
Ismael viveu 137 anos. Deu o último
suspiro e, ao morrer, reuniu-se a seus antepassados. 
\textit{\tiny 18}
Os descendentes de Ismael
ocuparam a região que vai de Havilá a Sur, a leste do Egito, na direção de Assur.
Ali, viveram em franca oposição a todos os seus parentes.

----------------------------------------------------------------------
