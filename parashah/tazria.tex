\section*{Parashat Tazria (12:1–13:59)}

\subsubsection*{A saude e o parto}
   
\textbf{\large 12}
 O SENHOR disse a Moisés: 
\textit{\tiny 2} 
“Dê as seguintes instruções ao povo de Israel. 

\smallskip
Se uma mulher engravidar e der à luz um filho, ficará cerimonialmente impura por
sete dias, como acontece durante a menstruação. 
\textit{\tiny 3} 
No oitavo dia, circuncidem o
menino. 
\textit{\tiny 4} 
Depois de esperar 33 dias, a mulher estará purificada do sangramento
do parto. Durante o período de purificação, não deverá tocar em coisa alguma que
seja consagrada. Também não poderá entrar no santuário enquanto não terminar
o período de purificação. 

\smallskip
\textit{\tiny 5} 
Se a mulher der à luz uma filha, ficará
cerimonialmente impura por duas semanas, como acontece durante a
menstruação. Depois de esperar 66 dias, estará purificada do sangramento do parto. 

\smallskip
\textit{\tiny 6} 
“Quando se completar o tempo de purificação pelo nascimento de um filho ou
de uma filha, a mulher levará um cordeiro de um ano como holocausto e um
pombinho ou rolinha para a oferta pelo pecado. Levará as ofertas ao sacerdote à
entrada da tenda do encontro. 
\textit{\tiny 7} 
O sacerdote as apresentará ao SENHOR para fazer
expiação pela mulher. Ela voltará a ficar cerimonialmente pura depois do
sangramento do parto. Essas são as instruções para a mulher depois do
nascimento de um filho ou de uma filha. 
\textit{\tiny 8} 
“Se a mulher não tiver condições de levar um cordeiro, levará duas rolinhas ou
dois pombinhos. Um será para o holocausto, e o outro, para a oferta pelo pecado.
O sacerdote os sacrificará para fazer expiação pela mulher, e ela ficará
cerimonialmente pura”.

  
\bigskip 
\subsubsection*{A saude e a pele}
\textbf{\large 13}
 O SENHOR disse a Moisés e a Arão: 

\smallskip
\textit{\tiny 2} 
“Se alguém tiver um inchaço, uma
erupção ou uma descoloração que possa ser sinal de lepra,
 essa pessoa será
levada ao sacerdote Arão ou a um de seus filhos. 
\textit{\tiny 3} 
O sacerdote examinará a região
afetada da pele. Se houver ali pelos que ficaram brancos e parecer que o
problema é mais profundo que a pele, é lepra, e o sacerdote que examinar a
pessoa a declarará cerimonialmente impura. 
\textit{\tiny 4} 
“Se, contudo, a região afetada da pele apresentar apenas uma descoloração
branca e a mancha não for mais profunda que a pele, e se os pelos da região não
se tornaram brancos, o sacerdote isolará a pessoa por sete dias. 
\textit{\tiny 5} 
No sétimo dia,
ele a examinará novamente. Se constatar que a região afetada não mudou e o
problema não se espalhou pela pele, isolará a pessoa por mais sete dias. 
\textit{\tiny 6} 
No sétimo dia, voltará a examiná-la. Se constatar que a área afetada diminuiu e não se
espalhou, o sacerdote declarará a pessoa cerimonialmente pura; era apenas uma
erupção. A pessoa lavará suas roupas e ficará pura. 
\textit{\tiny 7} 
Mas, se a erupção vier a se
espalhar depois de o sacerdote examinar a pessoa e a declarar pura, ela voltará
para ser examinada. 
\textit{\tiny 8} 
Se o sacerdote constatar que a erupção se espalhou,
declarará a pessoa cerimonialmente impura, pois é, de fato, lepra. 

\bigskip 
\textit{\tiny 9} 
“Quem apresentar algum sinal de lepra irá ao sacerdote para ser examinado.
\textit{\tiny 10}
Se o sacerdote encontrar um inchaço branco na pele, se alguns pelos sobre a
mancha tiverem ficado brancos e se houver uma ferida aberta na região afetada,
\textit{\tiny 11}
é um caso crônico de lepra, e o sacerdote declarará a pessoa cerimonialmente
impura. Nesses casos, não será necessário isolar a pessoa para avaliá-la, pois é
evidente que a pele está contaminada pela doença.

\smallskip
\textit{\tiny 12}
“Se a lepra se espalhar por toda a pele da pessoa e cobrir seu corpo da cabeça
aos pés, o sacerdote examinará a pessoa infectada. 
\textit{\tiny 13}
Se constatar que a doença
cobre todo o corpo, declarará a pessoa cerimonialmente pura. Uma vez que a pele
se tornou completamente branca, a pessoa está pura. 
\textit{\tiny 14}
Mas, se aparecerem
feridas abertas, a pessoa infectada será declarada cerimonialmente impura. 
\textit{\tiny 15}
O
sacerdote fará essa declaração assim que vir uma ferida aberta, pois esse tipo de
ferida indica a presença de lepra. 
\textit{\tiny 16}
Se, contudo, as feridas sararem e se tornarem
brancas como o resto da pele, a pessoa voltará ao sacerdote 
\textit{\tiny 17}
para ser examinada.
Se as regiões afetadas tiverem, de fato, se tornado brancas, o sacerdote declarará a
pessoa cerimonialmente pura, e assim ela estará.
   
\bigskip 
\textit{\tiny 18}
“Se alguém tiver na pele uma ferida purulenta e ela sarar, 
\textit{\tiny 19}
mas surgir em
seu lugar um inchaço branco ou uma mancha branca avermelhada, a pessoa irá ao
sacerdote para ser examinada. 
\textit{\tiny 20}
Se o sacerdote a examinar e constatar que a
mancha é mais profunda que a pele, e se os pelos da região afetada tiverem ficado
brancos, o sacerdote declarará a pessoa cerimonialmente impura. A ferida
purulenta indica lepra. 
\textit{\tiny 21}
Mas, se o sacerdote não encontrar pelos brancos na
região afetada e parecer que a mancha não é mais profunda que a pele, e até
diminuiu, o sacerdote isolará a pessoa por sete dias. 
\textit{\tiny 22}
Se, nesse período, a mancha
ou o inchaço se espalharem na pele, o sacerdote declarará a pessoa
cerimonialmente impura, pois é sinal de lepra. 
\textit{\tiny 23}
Se, contudo, a região afetada não
aumentar nem se espalhar, é apenas a cicatriz da ferida, e o sacerdote declarará a
pessoa cerimonialmente pura.
   
\bigskip 
\textit{\tiny 24}
“Se alguém sofrer uma queimadura na pele e aparecerem na região feridas
abertas de cor branca avermelhada ou completamente branca, 
\textit{\tiny 25}
o sacerdote a
examinará. Se constatar que os pelos na região afetada ficaram brancos, e se
parecer que a mancha é mais profunda que a pele, surgiu lepra na queimadura. O
sacerdote declarará a pessoa cerimonialmente impura, pois, sem dúvida, é lepra.
\textit{\tiny 26}
Mas, se não encontrar pelos brancos na região afetada, e se parecer que a ferida
não é mais profunda que a pele e tiver diminuído, o sacerdote isolará a pessoa por
sete dias. 
\textit{\tiny 27}
No sétimo dia, examinará a pessoa novamente. Se o problema tiver se
espalhado na pele, o sacerdote declarará a pessoa cerimonialmente impura, pois,
sem dúvida, é lepra. 
\textit{\tiny 28}
Se, contudo, a região afetada não tiver mudado ou se o
problema não tiver se espalhado na pele, mas tiver diminuído, é apenas o inchaço
da queimadura. O sacerdote declarará a pessoa cerimonialmente pura, pois é
apenas a cicatriz da queimadura.
   
\bigskip 
\textit{\tiny 29}
“Se um homem ou uma mulher tiver uma ferida na cabeça ou no queixo, 
\textit{\tiny 30}
o
sacerdote a examinará. Se constatar que a mancha é mais profunda que a pele e
tem pelos amarelados e finos, o sacerdote declarará a pessoa cerimonialmente
impura. É uma ferida causada por sarna na cabeça ou no queixo. 
\textit{\tiny 31}
Se o sacerdote
examinar a ferida e constatar que não é mais profunda que a pele, mas não tem
pelos escuros, isolará a pessoa por sete dias. 
\textit{\tiny 32}
No sétimo dia, o sacerdote
examinará a ferida novamente. Se constatar que ela não se espalhou, que não há
pelos amarelados e que não parece mais profunda que a pele, 
\textit{\tiny 33}
a pessoa raspará
todos os pelos, exceto na região afetada. Em seguida, o sacerdote isolará a pessoa
infectada por mais sete dias. 
\textit{\tiny 34}
No sétimo dia, examinará a ferida novamente. Se
ela não tiver se espalhado, e se não parecer mais profunda que a pele, o sacerdote
declarará a pessoa cerimonialmente pura. A pessoa lavará suas roupas e ficará
pura. 
\textit{\tiny 35}
Mas, se a ferida de sarna começar a se espalhar depois de a pessoa ter sido
declarada cerimonialmente pura, 
\textit{\tiny 36}
o sacerdote a examinará novamente. Se
constatar que a ferida se espalhou, não é necessário procurar pelos amarelados; a
pessoa infectada está cerimonialmente impura. 
\textit{\tiny 37}
Se, contudo, a cor da ferida de
sarna não mudar e pelos pretos voltarem a crescer na região afetada, a sarna está
curada, e o sacerdote declarará a pessoa cerimonialmente pura.
   
\bigskip 
\textit{\tiny 38}
“Se um homem ou uma mulher tiver manchas brancas na pele, 
\textit{\tiny 39}
o sacerdote
examinará a região afetada. Se constatar que as manchas brancas são opacas, é
uma simples erupção de pele, e a pessoa está cerimonialmente pura.
   
\bigskip 
\textit{\tiny 40}
“Se os cabelos de um homem caírem e ele ficar calvo, continua
cerimonialmente puro. 
\textit{\tiny 41}
Se caírem os cabelos da parte da frente da cabeça, ele
simplesmente ficou calvo na frente e continua puro. 
\textit{\tiny 42}
Mas, se uma ferida branca
avermelhada aparecer na região calva no alto ou na parte de trás da cabeça, é
lepra. 
\textit{\tiny 43}
O sacerdote o examinará e, se constatar que há inchaço ao redor da ferida
branca avermelhada em qualquer parte da calva do homem com aparência de
lepra, 
\textit{\tiny 44}
o homem está, de fato, infectado com lepra e está impuro. O sacerdote o
declarará cerimonialmente impuro por causa da ferida na cabeça.
\textit{\tiny 45}
“Quem sofrer de lepra rasgará as roupas e deixará o cabelo despenteado.
Cobrirá a boca e gritará: ‘Impuro! Impuro!’. 
\textit{\tiny 46}
Enquanto durar a lepra, ficará
cerimonialmente impuro e viverá isolado, fora do acampamento.”


\bigskip 
\textit{\tiny 47}
“Quando o mofo
 contaminar uma peça de roupa de lã ou de linho, 
\textit{\tiny 48}
um
tecido de lã ou de linho, a pele de um animal ou qualquer objeto de couro, 
\textit{\tiny 49}
e
quando a região contaminada da roupa, da pele do animal, do tecido liso ou
trançado, ou do artigo de couro se tornar esverdeada ou avermelhada, está
contaminada com mofo e deverá ser mostrada ao sacerdote. 
\textit{\tiny 50}
Depois de
examinar a região afetada, o sacerdote isolará o objeto afetado por sete dias. 
\textit{\tiny 51}
No
sétimo dia, examinará o objeto novamente. Se a região afetada tiver se espalhado,
a peça de roupa, o tecido liso ou trançado ou o artigo de couro foi, sem dúvida,
contaminado por mofo corrosivo e está cerimonialmente impuro. 
\textit{\tiny 52}
O sacerdote
queimará a peça de roupa, o tecido de lã ou de linho ou o artigo de couro, pois foi
contaminado por mofo corrosivo. Deve ser completamente destruído com fogo.

\smallskip   
\textit{\tiny 53}
“Se, contudo, o sacerdote examinar o objeto e constatar que a região
contaminada não se espalhou pela peça de roupa, pelo tecido liso ou trançado, ou
pelo artigo de couro, 
\textit{\tiny 54}
ordenará que o objeto seja lavado e, depois, isolado por
mais sete dias. 
\textit{\tiny 55}
O sacerdote examinará novamente o objeto depois de lavado. Se
constatar que a região contaminada não mudou de cor depois de ser lavada,
mesmo que a mancha não tenha se espalhado, o objeto está contaminado. Deve
ser completamente queimado, quer o mofo esteja do lado de dentro ou de fora.
\textit{\tiny 56}
Mas, se o sacerdote examinar o objeto e constatar que a região contaminada
diminuiu depois de ser lavada, cortará a mancha da peça de roupa, do tecido liso
ou trançado, ou do couro. 
\textit{\tiny 57}
Se a mancha reaparecer na peça de roupa, no tecido
liso ou trançado, ou no artigo de couro, é evidente que o mofo está se espalhando,
e o objeto contaminado deverá ser queimado. 
\textit{\tiny 58}
Se, contudo, a mancha
desaparecer da peça de roupa, do tecido, ou do artigo de couro depois de ter sido
lavado, o objeto será lavado novamente e, por fim, estará cerimonialmente puro.
\textit{\tiny 59}
“Essas são as instruções referentes ao mofo que contamina roupas de lã ou
linho, tecidos lisos ou trançados ou qualquer objeto de couro. É dessa forma que o
sacerdote determinará se os objetos estão cerimonialmente puros ou impuros”.

----------------------------------------------------------------------
