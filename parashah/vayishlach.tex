\section*{Parashat Vayishlach 32:4 - 36:43}

\textit{\tiny 4}
Disse-lhes: “Deem a seguinte mensagem ao meu
senhor Esaú: ‘Assim diz seu servo Jacó: Até o momento, estava morando com
nosso tio Labão 
\textit{\tiny 5}
e, agora, tenho bois, jumentos, rebanhos de ovelhas e cabras,
além de muitos servos e servas. Enviei estes mensageiros para informar meu
senhor da minha chegada, na esperança de que me receba amistosamente”.

\bigskip
\textit{\tiny 6}
Depois de transmitirem a mensagem, voltaram a Jacó e lhe disseram:
“Estivemos com seu irmão Esaú, e ele está vindo ao seu encontro com um bando
de quatrocentos homens!”. 
\textit{\tiny 7}
Quando ouviu a notícia, Jacó ficou apavorado. Dividiu
em dois grupos sua família e seus servos, e também os rebanhos, os bois e os
camelos, 
\textit{\tiny 8}
pois pensou: “Se Esaú encontrar um dos grupos e atacá-lo, talvez o
outro consiga escapar”.
\textit{\tiny 9}
Então Jacó orou: “Ó Deus de meu avô, Abraão, e Deus de meu pai, Isaque; ó
SENHOR, tu me disseste: ‘Volte para sua terra natal, para seus parentes’. E
prometeste: ‘Tratarei bem de você ’. 
\textit{\tiny 10}
Não sou digno de toda a bondade e
fidelidade que tens mostrado a mim, teu servo. Quando saí de casa e atravessei o
rio Jordão, não possuía nada além de um cajado. Agora, minha família e meus
servos formam duas caravanas! 
\textit{\tiny 11}
Por favor, salva-me de meu irmão, Esaú. Estou
com medo de que ele venha atacar tanto a mim quanto a minhas mulheres e meus
filhos. 
\textit{\tiny 12}
Mas tu prometeste: ‘Certamente tratarei bem de você e multiplicarei seus
descendentes até que se tornem tão numerosos quanto a areia à beira do mar, que
não se pode contar’”.
   
\bigskip
\textit{\tiny 13}
Jacó passou a noite ali. Depois, escolheu entre seus bens os seguintes
presentes para seu irmão, Esaú: 
\textit{\tiny 14}
duzentas cabras, vinte bodes, duzentas ovelhas,
vinte carneiros, 
\textit{\tiny 15}
trinta fêmeas de camelo com seus filhotes, quarenta vacas, dez
touros, vinte jumentas e dez jumentos. 
\textit{\tiny 16}
Dividiu esses animais em rebanhos,
entregou cada rebanho a um servo e lhes disse: “Vão à minha frente com os
animais, mas mantenham certa distância entre os rebanhos”.
   
\bigskip
\textit{\tiny 17}
Aos homens encarregados do primeiro grupo, deu as seguintes instruções:
“Quando meu irmão, Esaú, se encontrar com vocês, ele perguntará: ‘De quem são
servos? Para onde vão? Quem é o dono destes animais?’. 
\textit{\tiny 18}
Respondam: ‘Eles
pertencem ao seu servo Jacó, mas são um presente para Esaú, o senhor dele. Veja,
ele está vindo atrás de nós’”.
\textit{\tiny 19}
Jacó deu a mesma instrução aos encarregados do segundo e do terceiro
grupo e a todos que seguiam os rebanhos: “Digam a mesma coisa a Esaú quando o
encontrarem, 
\textit{\tiny 20}
e não se esqueçam de acrescentar: ‘Veja, seu servo Jacó está vindo
atrás de nós’”.
   Jacó pensou: “Tentarei apaziguá-lo com os presentes que estou enviando à
minha frente. Quando o vir, quem sabe ele me receberá amistosamente”. 
\textit{\tiny 21}
Assim,
os presentes foram enviados à frente, enquanto Jacó passou aquela noite no
acampamento.

\bigskip
\textit{\tiny 22}
Durante a noite, Jacó se levantou, tomou suas duas mulheres, suas duas servas e
seus onze filhos e atravessou com eles o rio Jaboque. 
\textit{\tiny 23}
Depois de levá-los para a
outra margem, fez passar todos os seus bens.
\textit{\tiny 24}
Com isso, Jacó ficou sozinho no acampamento. Veio então um homem, que
lutou com ele até o amanhecer. 
\textit{\tiny 25}
Quando o homem viu que não poderia vencer,
tocou a articulação do quadril de Jacó e a deslocou. 

\bigskip
\textit{\tiny 26}
O homem disse: “Deixe-me
ir, pois está amanhecendo!”.
    Jacó, porém, respondeu: “Não o deixarei ir enquanto não me abençoar”.
\textit{\tiny 27}
“Qual é seu nome?”, perguntou o homem.
    “Jacó”, respondeu ele.
\textit{\tiny 28}
O homem disse: “Seu nome não será mais Jacó. De agora em diante, você se
chamará Israel,
 pois lutou com Deus e com os homens e venceu”.
\textit{\tiny 29}
“Por favor, diga-me qual é seu nome”, disse Jacó.
    “Por que quer saber meu nome?”, replicou o homem. E abençoou Jacó ali.
    
\bigskip
\textit{\tiny 30}
Jacó chamou aquele lugar de Peniel,
 pois disse: “Vi Deus face a face e, no
entanto, minha vida foi poupada”. 

\bigskip
\textit{\tiny 31}
O sol estava nascendo quando Jacó partiu de
Peniel,
 mancando por causa do quadril deslocado. 
\textit{\tiny 32}
(Até hoje, o povo de Israel
não come o tendão perto da articulação do quadril, por causa do que aconteceu
naquela noite em que o homem feriu o tendão do quadril de Jacó.)

\bigskip   
\subsubsection*{Israel, Esaú e o reencontro}
\textbf{\large 33}
 Jacó levantou os olhos e viu Esaú aproximando-se com seus quatrocentos
homens. Assim, dividiu os filhos entre Lia, Raquel e as duas servas. 
\textit{\tiny 2} 
Colocou as
servas e os filhos delas à frente, Lia e seus filhos em seguida, e Raquel e José por
último. 

\bigskip   
\textit{\tiny 3} 
Jacó passou à frente e, ao aproximar-se de seu irmão, curvou-se até o
chão sete vezes. 
\textit{\tiny 4} 
Esaú correu ao encontro de Jacó e o abraçou; pôs os braços em
volta do pescoço do irmão e o beijou. E os dois choraram.

\bigskip   
\textit{\tiny 5} 
Então   Esaú viu as mulheres e as crianças e perguntou: “Quem são estas
pessoas que estão com você?”.
   Jacó respondeu: “São os filhos que Deus, em sua bondade, concedeu a seu
servo”. 
\textit{\tiny 6} 
As servas e seus filhos se aproximaram e se curvaram diante de Esaú. 
\textit{\tiny 7} 
Em
seguida, Lia e seus filhos vieram e se curvaram diante dele. Por fim, José e Raquel
se aproximaram e se curvaram diante dele.

\bigskip   
\textit{\tiny 8} 
“E o que eram todos aqueles rebanhos que encontrei no caminho?”, perguntou
Esaú.
   Jacó respondeu: “São presentes, meu senhor, para garantir sua amizade”.
\textit{\tiny 9} 
“Meu irmão, eu já tenho muitos bens”, disse Esaú. “Guarde para você o que é
seu.”
\textit{\tiny 10}
Mas Jacó insistiu: “Não! Se obtive seu favor, peço que aceite meu presente. E
que alívio é ver seu sorriso amigável! É como ver a face de Deus! 
\textit{\tiny 11}
Por favor, aceite
o presente que eu lhe trouxe, pois Deus tem sido muito bondoso comigo. Tenho
mais que suficiente”. Diante da insistência de Jacó, Esaú acabou aceitando o
presente.
   
\bigskip   
\textit{\tiny 12}
Então Esaú disse: “Vamos andando. Eu o acompanharei”.
\textit{\tiny 13}
Jacó, porém, respondeu: “Como meu senhor pode ver, algumas das crianças
são bem pequenas, e os rebanhos também têm crias. Se os forçarmos demais,
mesmo que por um dia, pode ser que os animais morram. 
\textit{\tiny 14}
Por favor, meu
senhor, vá adiante do seu servo. Seguiremos mais devagar, em um ritmo que os
rebanhos e as crianças possam acompanhar. Encontrarei com meu senhor em
Seir”.
\textit{\tiny 15}
“Está bem”, disse Esaú. “Mas, pelo menos, permita-me deixar alguns dos
meus homens para acompanhá-lo.”
   Jacó respondeu: “Não é necessário. Para mim, ter sido bem recebido por meu
senhor já é o bastante!”.

\bigskip   
\textit{\tiny 16}
Esaú deu meia-volta e regressou a Seir naquele mesmo dia. 
\textit{\tiny 17}
Jacó, por sua
vez, viajou até Sucote, onde construiu uma casa para si e abrigos para seus
rebanhos. Por isso, aquele lugar é chamado de Sucote.

\bigskip   
\subsubsection*{Israel em Siquém}
\textit{\tiny 18}
Depois de percorrer todo o caminho desde Padã-Arã, Jacó chegou em
segurança à cidade de Siquém, na terra de Canaã, e acampou em seus arredores.
\textit{\tiny 19}
Jacó comprou da família de Hamor, pai de Siquém, o terreno onde estava
acampado, por cem peças de prata.
\textit{\tiny 20}
Ali, construiu um altar e o chamou de El-
Elohe-Israel.

\bigskip   
\textbf{\large 34}
 Certa vez, Diná, filha de Jacó e Lia, saiu para visitar algumas moças que
viviam na região. 
\textit{\tiny 2} 
O príncipe daquela terra era Siquém, filho de Hamor, o heveu.
Quando ele viu Diná, a agarrou e a violentou, 
\textit{\tiny 3} 
mas depois apaixonou-se por ela e
tentou conquistar sua afeição com palavras carinhosas. 
\textit{\tiny 4} 
Disse a seu pai, Hamor:
“Consiga-me essa moça, pois quero me casar com ela”. 

\bigskip   
\textit{\tiny 5} 
Jacó logo soube que Siquém tinha violentado Diná, sua filha. Mas, como seus
filhos estavam no campo cuidando dos rebanhos, não disse nada até que eles
voltassem. 
\textit{\tiny 6} 
Hamor, pai de Siquém, foi tratar da questão com Jacó. 
\textit{\tiny 7} 
Nesse meio tempo, os filhos de Jacó voltaram do campo assim que souberam o que havia
acontecido. Ficaram abalados e furiosos porque sua irmã havia sido violentada.

\bigskip   
Siquém tinha cometido um ato vergonhoso contra a família de Jacó,
 algo que
jamais se deve fazer. 

\bigskip   
\textit{\tiny 8} 
Hamor fez um pedido a Jacó e seus filhos: “Meu filho Siquém se apaixonou
por sua filha. Por favor, permitam que ele se case com ela. 
\textit{\tiny 9} 
Aliás, podemos
arranjar outros casamentos: vocês entregam suas filhas para nossos filhos, e nós
entregamos nossas filhas para seus filhos. 
\textit{\tiny 10}
Vocês poderão viver em nosso meio;
a terra está à sua disposição! Estabeleçam-se aqui e façam negócios conosco.
Fiquem à vontade para comprar propriedades na região”.
   
\bigskip   
\textit{\tiny 11}
Então o próprio Siquém falou ao pai e aos irmãos de Diná: “Por favor, sejam
bondosos comigo e deixem que eu me case com ela”, implorou. “Eu lhes darei o
que me pedirem. 
\textit{\tiny 12}
Seja qual for o dote ou o presente que pedirem, eu o pagarei,
por maior que seja; só peço que me entreguem a moça para ser minha mulher.”
   
\bigskip   
\textit{\tiny 13}
Os filhos de Jacó responderam com falsidade a Siquém e a seu pai, Hamor,
uma vez que Siquém tinha violado Diná, a irmã deles. 
\textit{\tiny 14}
Disseram: “Não podemos
permitir uma coisa dessas, pois você não é circuncidado. Seria uma vergonha para
nossa irmã casar-se com um homem como você. 
\textit{\tiny 15}
Porém, temos uma solução. Se
todos os homens do seu povo forem circuncidados, como nós somos,
\textit{\tiny 16}
entregaremos nossas filhas e nos casaremos com suas filhas. Viveremos entre
vocês e nos tornaremos um só povo. 
\textit{\tiny 17}
Mas, se não concordarem em ser
circuncidados, tomaremos nossa irmã e iremos embora”.
   
\bigskip   
\textit{\tiny 18}
Hamor e seu filho Siquém aceitaram a proposta. 
\textit{\tiny 19}
Sem demora, Siquém fez
o que tinham pedido, pois desejava ardentemente a filha de Jacó. Siquém era o
mais respeitado dos membros de sua família 
\textit{\tiny 20}
e foi com seu pai, Hamor,
apresentar a proposta aos líderes que estavam à porta da cidade.
\textit{\tiny 21}
“Esses homens são nossos amigos”, disseram eles. “Devemos convidá-los para
viver entre nós e negociar livremente conosco. Há bastante espaço para eles nesta
terra. Podemos nos casar com as filhas deles, e eles, com as nossas. 
\textit{\tiny 22}
Mas eles só
aceitarão ficar aqui e tornar-se um só povo conosco se todos os nossos homens
forem circuncidados, como eles são. 
\textit{\tiny 23}
Se o fizermos, todos os seus rebanhos e
bens passarão, com o tempo, a ser nossos. Aceitemos a condição deles e deixemos
que se estabeleçam entre nós.”
\textit{\tiny 24}
Todos os membros do conselho da cidade concordaram com Hamor e
Siquém, e todos os homens da cidade foram circuncidados. 

\bigskip   
\textit{\tiny 25}
Três dias depois,
quando eles ainda sentiam dores, dois filhos de Jacó, Simeão e Levi, irmãos de
Diná por parte de pai e mãe, tomaram suas espadas e entraram na cidade sem
encontrar resistência. Então, massacraram todos os homens de lá 
\textit{\tiny 26}
e mataram
Hamor e seu filho Siquém ao fio da espada. Depois, tiraram Diná da casa de
Siquém e voltaram para o acampamento.
   
\bigskip   
\textit{\tiny 27}
Enquanto isso, os outros filhos de Jacó chegaram à cidade. Vendo eles que
todos os homens estavam mortos, saquearam a cidade, pois sua irmã tinha sido
violentada ali. 
\textit{\tiny 28}
Levaram as ovelhas, os bois e os jumentos, tudo que conseguiram
encontrar dentro da cidade e nos campos. 
\textit{\tiny 29}
Tomaram todas as riquezas,
saquearam as casas e levaram as crianças e mulheres como prisioneiras.
   
\bigskip   
\textit{\tiny 30}
Depois de tudo isso, Jacó disse a Simeão e a Levi: “Vocês arruinaram minha
vida! Serei odiado por todos os povos desta terra, pelos cananeus e ferezeus.
Somos tão poucos que eles se unirão e nos esmagarão. Eles me atacarão, e toda a
minha família será exterminada!”.
   
\bigskip   
\textit{\tiny 31}
Mas eles responderam: “Por acaso deveríamos permitir que nossa irmã fosse
tratada como prostituta?”.
   
\bigskip   
\subsubsection*{Israel em Betel}
\textbf{\large 35}
 Deus disse a Jacó: “Apronte-se, mude-se para Betel e estabeleça-se ali. Ao
chegar, construa um altar para o Deus que lhe apareceu quando você estava
fugindo de seu irmão, Esaú”. 

\bigskip   
\textit{\tiny 2} 
Jacó disse à sua família e a todos que estavam com ele: “Joguem fora todos os
seus ídolos pagãos, purifiquem-se e vistam roupas limpas. 
\textit{\tiny 3} 
Vamos a Betel, onde
construirei um altar para o Deus que respondeu às minhas orações quando eu
estava angustiado. Ele tem estado comigo por onde ando”. 
\textit{\tiny 4} 
Então   entregaram a Jacó todos os ídolos pagãos e as argolas que usavam nas
orelhas, e ele os enterrou ao pé da grande árvore perto de Siquém. 
\textit{\tiny 5} 

\bigskip   
Quando
partiram, o terror de Deus se espalhou de tal forma entre os moradores das
cidades próximas que ninguém atacou a família de Jacó. 
\textit{\tiny 6} 
Por fim, Jacó e todos que estavam com ele chegaram a Luz (também chamada
Betel), em Canaã. 
\textit{\tiny 7} 
Jacó construiu um altar ali e chamou o lugar de El-Betel,
 pois
Deus lhe havia aparecido em Betel quando ele estava fugindo de seu irmão. 
\textit{\tiny 8} 
Pouco tempo depois, Débora, a serva que havia amamentado Rebeca, morreu e
foi sepultada ao pé do carvalho no vale perto de Betel. Desde então, a árvore é
chamada de Alom-Bacute. 

\bigskip   
\textit{\tiny 9} 
Agora que Jacó havia regressado de Padã-Arã, Deus lhe apareceu outra vez em
Betel e o abençoou: 
\textit{\tiny 10}
“Seu nome é Jacó, mas você não se chamará mais Jacó. De
agora em diante, seu nome será Israel”. Assim, Deus deu a ele o nome de Israel.
   
\bigskip   
\textit{\tiny 11}
Deus também lhe disse: “Eu sou o Deus Todo-poderoso.
 Seja fértil e
multiplique-se. Você se tornará uma grande nação, até mesmo muitas nações.
Haverá reis entre seus descendentes. 
\textit{\tiny 12}
Eu lhe darei a terra que dei a Abraão e
Isaque. Sim, eu a darei a você e a seus descendentes”. 
\textit{\tiny 13}
Em seguida, Deus se
elevou do lugar onde havia falado a Jacó.
   
\bigskip   
\textit{\tiny 14}
Jacó levantou uma coluna de pedra para marcar o lugar onde Deus lhe havia
falado. Depois, derramou vinho sobre a coluna, como oferta a Deus, e a ungiu com
azeite de oliva. 
\textit{\tiny 15}
Chamou o lugar de Betel,
 pois ali Deus lhe havia falado.

\bigskip   
\subsubsection*{Morte de Raquel}
\textit{\tiny 16}
Depois que partiram de Betel, rumaram para Efrata. Raquel, porém, sentiu
fortes dores e entrou em trabalho de parto quando ainda estavam a certa
distância da cidade. 
\textit{\tiny 17}
As dores de parto aumentaram, e a parteira lhe disse: “Não
tenha medo! Você terá outro menino!”. 
\textit{\tiny 18}
Raquel estava quase morrendo, mas,
com seu último suspiro, chamou o menino de Benoni.
 O pai do bebê, no
entanto, o chamou de Benjamim.
 
\bigskip   
\textit{\tiny 19}
Assim, Raquel morreu e foi sepultada junto
ao caminho para Efrata (ou seja, Belém). 
\textit{\tiny 20}
Sobre o túmulo de Raquel, Jacó
levantou um monumento de pedra, que está lá até hoje.
   
\bigskip   
\subsubsection*{Ruben e Bila}
\textit{\tiny 21}
Então Jacó
 seguiu viagem e acampou além de Migdal-Éder. 
\textit{\tiny 22}
Enquanto
moravam ali, Rúben teve relações com Bila, concubina de seu pai, e Jacó ficou
sabendo disso.


\bigskip   
\subsubsection*{Descendentes de Jacó}
\textit{\tiny 23}
Os  filhos de Lia foram Rúben (o filho mais velho de Jacó), Simeão, Levi, Judá,
 Issacar e Zebulom.

\bigskip   
\textit{\tiny 24}
Os filhos de Raquel foram José e Benjamim.

\bigskip   
\textit{\tiny 25}
Os filhos de Bila, serva de Raquel, foram Dã e Naftali.

\bigskip   
\textit{\tiny 26}
Os filhos de Zilpa, serva de Lia, foram Gade e Aser.

\bigskip   
Esses são os filhos que nasceram a Jacó em Padã-Arã.

\bigskip   
\subsubsection*{Morte de Isaque}
\textit{\tiny 27}
Então Jacó voltou à casa de seu pai, Isaque, em Manre, perto de Quiriate-Arba
(hoje chamada Hebrom), onde Abraão e Isaque viveram como estrangeiros.

\bigskip   
\textit{\tiny 28}
Isaque viveu 180 anos. 
\textit{\tiny 29}
Deu o último suspiro e, ao morrer em boa velhice, reuniu-se a seus antepassados. Seus filhos, Esaú e Jacó, o sepultaram.
   
\bigskip   
\subsubsection*{Descendentes de Esaú}
\textbf{\large 36}
 Este é o relato dos descendentes de Esaú (também chamado Edom). 
\textit{\tiny 2} 
Esaú
se casou com duas moças de Canaã: Ada, filha de Elom, o hitita, e Oolibama, filha
de Aná e neta de Zibeão, o heveu. 
\textit{\tiny 3} 
Também se casou com sua prima Basemate,
que era filha de Ismael e irmã de Nebaiote. 
\textit{\tiny 4} 
Ada deu à luz um filho chamado
Elifaz. Basemate deu à luz um filho chamado Reuel. 
\textit{\tiny 5} 
Oolibama deu à luz filhos
chamados Jeús, Jalão e Corá. Todos esses filhos nasceram a Esaú na terra de
Canaã. 

\bigskip   
\textit{\tiny 6} 
Esaú tomou suas mulheres, seus filhos e filhas e todos os de sua casa, além de
seus rebanhos e o gado, toda a riqueza que havia adquirido na terra de Canaã, e se
mudou para longe de seu irmão, Jacó. 
\textit{\tiny 7} 
Seus rebanhos e bens eram tantos que a
terra onde moravam não era suficiente para sustentá-los. 
\textit{\tiny 8} 
Portanto, Esaú
(também chamado Edom) se estabeleceu na região montanhosa de Seir. 

\bigskip   
\subsubsection*{Edomitas}
\textit{\tiny 9} 
Este é o relato dos descendentes de Esaú, os edomitas, que viviam na região
montanhosa de Seir.
\textit{\tiny 10}
Estes são os nomes dos filhos de Esaú: Elifaz, filho de Ada, mulher de Esaú; e
 Reuel, filho de Basemate, mulher de Esaú.
\textit{\tiny 11}
Os descendentes de Elifaz foram: Temã, Omar, Zefô, Gaetã e Quenaz. 
\textit{\tiny 12}
Timna,
 concubina de Elifaz, filho de Esaú, deu à luz um filho chamado Amaleque. Esses
 são os descendentes de Ada, mulher de Esaú.
\textit{\tiny 13}
Os descendentes de Reuel foram: Naate, Zerá, Samá e Mizá. Esses são os
 descendentes de Basemate, mulher de Esaú.
\textit{\tiny 14}
Esaú também teve filhos com Oolibama, filha de Aná, neta de Zibeão. Seus
 nomes eram: Jeús, Jalão e Corá.

\bigskip   
\textit{\tiny 15}
Estes são os descendentes de Esaú que se tornaram chefes de vários clãs:
Os descendentes de Elifaz, filho mais velho de Esaú, se tornaram chefes dos clãs
 de Temã, Omar, Zefô, Quenaz, 
\textit{\tiny 16}
Corá, Gaetã e Amaleque. Esses são os chefes de
 clãs descendentes de Elifaz na terra de Edom. Todos eles foram descendentes de
 Ada, mulher de Esaú.
\textit{\tiny 17}
Os descendentes de Reuel, filho de Esaú, se tornaram chefes dos clãs de Naate,
 Zaerá, Samá e Mizá. Esses são os chefes dos clãs descendentes de Reuel na terra
 de Edom. Todos eles foram descendentes de Basemate, mulher de Esaú.
\textit{\tiny 18}
Os descendentes de Esaú e sua mulher Oolibama se tornaram os chefes dos clãs
 de Jeús, Jalão e Corá. Esses são os chefes dos clãs descendentes de Oolibama,
 mulher de Esaú, filha de Aná.

\bigskip   
\textit{\tiny 19}
Esses são os clãs que descenderam de Esaú (também chamado de Edom), cada
um identificado pelo nome de seu chefe.
\textit{\tiny 20}
Estes são os nomes das tribos que descenderam de Seir, o horeu, que habitavam
na terra de Edom: Lotã, Sobal, Zibeão e Aná, 
\textit{\tiny 21}
Disom, Ézer e Disã. Estes são os
chefes dos clãs horeus, descendentes de Seir, que habitavam na terra de Edom.
\textit{\tiny 22}
Os  descendentes de Lotã foram: Hori e Hemã. A irmã de Lotã se chamava
 Timna.
\textit{\tiny 23}
Os descendentes de Sobal foram: Alvã, Manaate, Ebal, Sefô e Onã.
\textit{\tiny 24}
Os descendentes de Zibeão foram: Aiá e Aná. (Foi este Aná que descobriu as
 fontes de águas quentes no deserto enquanto levava os jumentos de seu pai para
 pastar.)
\textit{\tiny 25}
Os descendentes de Aná foram: seu filho Disom e sua filha Oolibama.
\textit{\tiny 26}
Os descendentes de Disom
\textit{\tiny 31}
 foram: Hendã, Esbã, Itrã e Querã.
\textit{\tiny 27}
Os descendentes de Ézer foram: Bilã, Zaavã e Acã.
\textit{\tiny 28}
Os descendentes de Disã foram Uz e Arã.
\textit{\tiny 29}
Estes, portanto, foram os chefes dos clãs horeus: Lotã, Sobal, Zibeão, Aná,
\textit{\tiny 30}
Disom, Ézer e Disã. Os clãs horeus são identificados pelo nome de seus chefes,
que habitavam na terra de Seir.



\bigskip   
\textit{\tiny 31}
Estes    são os reis que governaram na terra de Edom antes de os israelitas terem
rei:
\textit{\tiny 32}
Belá, filho de Beor, reinou em Edom, na cidade de Dinabá.
\textit{\tiny 33}
Quando Belá morreu, Jobabe, filho de Zerá, de Bozra, foi seu sucessor.
\textit{\tiny 34}
Quando Jobabe morreu, Husã, da terra dos temanitas, foi seu sucessor.
\textit{\tiny 35}
Quando    Husã morreu, Hadade, filho de Bedade, foi seu sucessor na cidade de
 Avite. Foi Hadade quem derrotou os midianitas na terra de Moabe.
\textit{\tiny 36}
Quando Hadade morreu, Samlá, da cidade de Masreca, foi seu sucessor.
\textit{\tiny 37}
Quando Samlá morreu, Saul, da cidade de Reobote, próxima ao Eufrates,
 foi
 seu sucessor.
\textit{\tiny 38}
Quando Saul morreu, Baal-Hanã, filho de Acbor, foi seu sucessor.
\textit{\tiny 39}
Quando Baal-Hanã, filho de Acbor, morreu, Hadade
 foi seu sucessor na
 cidade de Paú. Sua mulher era Meetabel, filha de Matrede e neta de Mezaabe.
\textit{\tiny 40}
Estes  são os nomes dos chefes dos clãs descendentes de Esaú, que habitavam
nos lugares que têm seus nomes: Timna, Alvá, Jetete, 
\textit{\tiny 41}
Oolibama, Elá, Pinom,
\textit{\tiny 42}
Quenaz, Temã, Mibzar, 
\textit{\tiny 43}
Magdiel e Irã. 

\medskip   
Esses são os chefes dos clãs de Edom,
relacionados de acordo com seus assentamentos na terra que ocupavam. Todos
eles foram descendentes de Esaú, antepassado dos edomitas.

----------------------------------------------------------------------
