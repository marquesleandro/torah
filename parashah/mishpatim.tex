\section*{Parashat Mishpatim 21:1–24:18}
   
\textbf{\large 21}
 “Estes são os decretos que você apresentará a Israel: 

\bigskip
\subsubsection*{O homem e o escravo hebreu homem}
\textit{\tiny 2} 
“Se você comprar um escravo hebreu, ele não poderá servi-lo por mais de seis
anos. Liberte-o no sétimo ano, e ele nada lhe deverá pela liberdade. 

\smallskip
\textit{\tiny 3} 
Se ele era
solteiro quando se tornou seu escravo, partirá solteiro. Mas, se era casado antes de
se tornar seu escravo, a esposa deverá ser liberta com ele. 

\smallskip
\textit{\tiny 4} 
“Se seu senhor lhe deu uma mulher em casamento enquanto ele era escravo, e
se o casal teve filhos e filhas, somente o homem será liberto no sétimo ano. A
mulher e os filhos continuarão a pertencer ao senhor. 

\smallskip
\textit{\tiny 5} 
O escravo, contudo,
poderá declarar: ‘Amo meu senhor, minha esposa e meus filhos. Não desejo ser
liberto’. 
\textit{\tiny 6} 
Nesse caso, seu senhor o apresentará aos juízes. Em seguida, o levará
até a porta ou até o batente da porta e furará a orelha dele com um furador.
Depois disso, o escravo servirá a seu senhor pelo resto da vida. 

\bigskip
\subsubsection*{O homem e o escravo mulher}
\textit{\tiny 7} 
“Quando um homem vender a filha como escrava, ela não será liberta como os
homens. 

\smallskip
\textit{\tiny 8} 
Se ela não agradar seu senhor, ele permitirá que alguém lhe pague o
resgate, mas não poderá vendê-la a estrangeiros, pois rompeu o contrato com ela. 

\smallskip
\textit{\tiny 9} 
Mas, se o senhor da escrava a entregar como mulher ao filho dele, não a tratará
mais como escrava, mas sim como filha.
   
\smallskip
\textit{\tiny 10}
“Se um homem que se casou com uma escrava tomar para si outra esposa,
não deverá descuidar dos direitos da primeira mulher com respeito a
alimentação, vestuário e intimidade sexual. 
\textit{\tiny 11}
Se ele não cumprir alguma dessas
obrigações, ela poderá sair livre, sem pagar coisa alguma.”

\bigskip
\subsubsection*{O homem, seu semelhante e o assassinato}
\textit{\tiny 12}
“Quem agredir e matar outra pessoa será executado.

\smallskip
\textit{\tiny 13}
Mas, se for apenas um
acidente permitido por Deus, definirei um lugar de refúgio para onde o
responsável pela morte possa fugir. 

\smallskip
\textit{\tiny 14}
Se, contudo, alguém matar outra pessoa
intencionalmente, o assassino será preso e executado, mesmo que tenha buscado
refúgio em meu altar.
   
\bigskip
\subsubsection*{O homem, seu semelhante e a agressão}
\textit{\tiny 15}
“Quem agredir seu pai ou sua mãe será executado.
   
\smallskip
\textit{\tiny 16}
“Quem sequestrar alguém será executado, quer a vítima seja encontrada em
seu poder, quer ele a tenha vendido como escrava.

\smallskip
\textit{\tiny 17}
“Quem ofender a honra de seu pai ou de sua mãe será executado.
  
\smallskip
\textit{\tiny 18}
“Se dois homens brigarem e um deles acertar o outro com uma pedra ou com
o punho e o outro não morrer, mas ficar de cama, 
\textit{\tiny 19}
o agressor não será castigado
se, posteriormente, o que foi ferido conseguir voltar a andar fora de casa, mesmo
que precise de muletas; o agressor indenizará a vítima pelos salários que ela
perder e se responsabilizará por sua total recuperação.
   
\smallskip
\textit{\tiny 20}
“Se um senhor espancar seu escravo ou sua escrava com uma vara e, como
resultado, o escravo morrer, o senhor será castigado. 
\textit{\tiny 21}
Mas, se o escravo se
recuperar em um ou dois dias, o senhor não receberá castigo algum, pois o
escravo é sua propriedade.
   
\smallskip
\textit{\tiny 22}
“Se dois homens brigarem e um deles atingir, por acidente, uma mulher
grávida e ela der à luz prematuramente, sem que haja outros danos, o homem
que atingiu a mulher pagará a indenização que o marido dela exigir e os juízes
aprovarem. 
\textit{\tiny 23}
Mas, se houver outros danos, o castigo deverá corresponder à
gravidade do dano causado: vida por vida, 
\textit{\tiny 24}
olho por olho, mão por mão, pé por
pé, 
\textit{\tiny 25}
queimadura por queimadura, ferida por ferida, contusão por contusão.
   
\smallskip
\textit{\tiny 26}
“Se um senhor ferir seu escravo ou sua escrava no olho e o cegar, libertará o
escravo como compensação pelo olho. 
\textit{\tiny 27}
Se quebrar o dente de seu escravo ou de
sua escrava, libertará o escravo como compensação pelo dente.
   
\bigskip
\subsubsection*{O homem, o animal e a agressão}
\textit{\tiny 28}
“Se um boi matar a chifradas um homem ou uma mulher, o boi será
apedrejado, e não será permitido comer sua carne. Nesse caso, porém, o dono do
boi não será responsabilizado. 

\smallskip
\textit{\tiny 29}
Mas, se o boi costumava chifrar pessoas e o dono
havia sido informado, porém não manteve o animal sob controle, se o boi matar
alguém, será apedrejado, e o dono também será executado. 
\textit{\tiny 30}
Os parentes do
morto, no entanto, poderão aceitar uma indenização pela vida perdida. O dono do
boi poderá resgatar a própria vida ao pagar o que for exigido.
   
\smallskip
\textit{\tiny 31}
“A mesma lei se aplica se o boi chifrar um menino ou uma menina. 

\smallskip
\textit{\tiny 32}
Mas, se
o boi chifrar um escravo ou uma escrava, o dono do boi pagará trinta moedas de
prata ao senhor do escravo, e o boi será apedrejado.
   
\smallskip
\textit{\tiny 33}
“Se alguém cavar ou destampar um poço e um boi ou jumento cair dentro
dele, 
\textit{\tiny 34}
o proprietário do poço indenizará totalmente o dono do animal, mas
poderá ficar com o animal morto.
   
\smallskip
\textit{\tiny 35}
“Se o boi de alguém ferir o boi do vizinho e o animal ferido morrer, os dois
donos venderão o animal vivo e dividirão o dinheiro entre si em partes iguais;
também dividirão entre si o animal morto. 

\smallskip
\textit{\tiny 36}
Mas, se o boi costumava chifrar e o
dono não manteve o animal sob controle, o dono entregará um boi vivo como
indenização pelo boi morto e poderá ficar com o animal morto.”
 
\bigskip
\subsubsection*{O homem, seu semelhante e o roubo}
\textbf{\large 22}
           “Se alguém roubar um boi ou uma ovelha e matar o animal ou vendê-
lo, o ladrão pagará cinco bois para cada boi roubado e quatro ovelhas para cada
ovelha roubada. 

\smallskip
\textit{\tiny 2} 
“Se um ladrão for pego em flagrante arrombando uma casa e for ferido e
morto no confronto, a pessoa que matou o ladrão não será culpada de homicídio. 

\smallskip
\textit{\tiny 3} 
Mas, se isso acontecer durante o dia, a pessoa que matou o ladrão será culpada
de homicídio.

\smallskip
   “O ladrão que for pego restituirá o valor total daquilo que roubou. Se não puder
restituir o valor, será vendido como escravo para pagar pelos bens roubados. 

\smallskip
\textit{\tiny 4} 
Se alguém roubar um boi, um jumento ou uma ovelha e o animal for encontrado
vivo, em poder do ladrão, ele pagará o dobro do valor do animal roubado. 

\bigskip
\subsubsection*{O homem, seu semelhante e as posses}
\textit{\tiny 5} 
“Se um animal estiver pastando no campo ou na videira e o dono o soltar para
pastar no campo de outra pessoa, o dono do animal entregará como indenização o
melhor de seus cereais ou de suas uvas. 

\smallskip
\textit{\tiny 6} 
“Se alguém estiver queimando espinheiros e o fogo se espalhar para o campo
de outra pessoa e destruir o cereal já colhido, ou a plantação pronta para a
colheita, ou a lavoura inteira, aquele que começou o fogo pagará por todo o
prejuízo. 

\smallskip
\textit{\tiny 7} 
“Se alguém entregar valores ou bens a um vizinho para que este os guarde e
eles forem roubados da casa do vizinho, o ladrão, se for pego, restituirá o dobro
do valor dos itens roubados. 
\textit{\tiny 8} 
Mas, se o ladrão não for pego, o dono da casa
comparecerá diante dos juízes para que se determine se foi ele quem roubou os
bens. 

\smallskip
\textit{\tiny 9} 
“Em qualquer caso de disputa entre vizinhos em que ambos afirmem ser
donos de determinado boi, jumento, ovelha, peça de roupa ou objeto perdido, as
duas partes comparecerão diante dos juízes, e a pessoa que eles considerarem
culpada pagará o dobro à outra.
   
\smallskip
\textit{\tiny 10}
“Se alguém deixar um jumento, um boi, uma ovelha ou outro animal sob os
cuidados de outra pessoa e o animal morrer, for ferido ou levado embora, e
ninguém vir o que aconteceu, 
\textit{\tiny 11}
a pessoa que estava cuidando do animal fará
diante do SENHOR um juramento de que não roubou o animal; o dono aceitará o
juramento e não será exigido pagamento algum. 
\textit{\tiny 12}
Mas, se o animal for roubado
do vizinho, ele indenizará o dono. 
\textit{\tiny 13}
Se tiver sido despedaçado por um animal
selvagem, o que restou da carcaça será apresentado como prova, e não será
exigido pagamento algum.
   
\smallskip
\textit{\tiny 14}
“Se alguém pedir um animal emprestado ao vizinho e o animal for ferido ou
morrer na ausência do dono, a pessoa que pediu o animal emprestado indenizará
o dono totalmente. 
\textit{\tiny 15}
Mas, se o dono estiver presente, não será exigido pagamento
algum. Também não será exigida indenização alguma se o animal tiver sido
alugado, pois o valor do aluguel cobrirá a perda.”

\bigskip
\subsubsection*{O homem e as responsabilidades gerais}
\textit{\tiny 16}
“Se um homem seduzir uma moça virgem que não esteja comprometida e tiver
relações sexuais com ela, pagará à família dela o preço costumeiro do dote e se
casará com ela. 
\textit{\tiny 17}
Mas, se o pai da moça não permitir o casamento, o homem lhe
pagará o equivalente ao dote de uma virgem.
   
\smallskip
\textit{\tiny 18}
“Não deixe que a feiticeira viva.
   
\smallskip
\textit{\tiny 19}
“Quem tiver relações sexuais com um animal certamente será executado.
   
\smallskip
\textit{\tiny 20}
“Quem sacrificar a qualquer outro deus além do SENHOR será destruído.
   
\smallskip
\textit{\tiny 21}
“Não maltrate nem oprima os estrangeiros. Lembre-se de que vocês também
foram estrangeiros na terra do Egito.
   
\smallskip
\textit{\tiny 22}
“Não explore a viúva nem o órfão. 
\textit{\tiny 23}
Se você os explorar e eles clamarem a
mim, certamente ouvirei seu clamor. 
\textit{\tiny 24}
Minha ira se acenderá contra você e o
matarei pela espada. Então sua esposa ficará viúva e seus filhos ficarão órfãos.
   
\smallskip
\textit{\tiny 25}
“Se você emprestar dinheiro a alguém do meu povo que esteja necessitado,
não cobre juros visando lucro, como fazem os credores. 
\textit{\tiny 26}
Se tomar a capa do seu
próximo como garantia para um empréstimo, devolva-a antes do pôr do sol.
\textit{\tiny 27}
Talvez a capa seja a única coberta que ele tem para se aquecer. Como ele poderá
dormir sem ela? Se não a devolver e se o seu próximo pedir socorro a mim, eu o
ouvirei, pois sou misericordioso.
   
\smallskip
\textit{\tiny 28}
“Não blasfeme contra Deus nem amaldiçoe as autoridades do seu povo.
   
\smallskip
\textit{\tiny 29}
“Quando entregar as ofertas das colheitas, do vinho e do azeite, não retenha
coisa alguma.
   “Consagre a mim seu primeiro filho.
\textit{\tiny 30}
“Também entregue a mim os machos das primeiras crias das vacas, das
ovelhas e das cabras. Deixe o animal com a mãe por sete dias e, no oitavo,
entregue-o a mim.

\smallskip
\textit{\tiny 31}
Vocês serão meu povo santo. Por isso, não comam a carne de animais
despedaçados e mortos por feras no campo; joguem a carne para os cães.”
 
\bigskip
\subsubsection*{O homem, seu semelhante e o testemunho}
\textbf{\large 23}
 “Não espalhe boatos falsos. Não coopere com pessoas perversas sendo
falsa testemunha. 

\smallskip
\textit{\tiny 2} 
“Não se deixe levar pela maioria na prática do mal. Quando o chamarem para
testemunhar em um processo legal, não permita que a multidão o influencie a
perverter a justiça. 
\textit{\tiny 3} 
E não incline seu testemunho em favor de uma pessoa só
porque ela é pobre. 

\smallskip
\textit{\tiny 4} 
“Se você deparar com o boi ou o jumento perdido de seu inimigo, leve-o de
volta ao dono. 
\textit{\tiny 5} 
Se vir o jumento de alguém que o odeia cair sob o peso de sua
carga, não faça de conta que não viu. Pare e ajude o dono a levantá-lo. 

\smallskip
\textit{\tiny 6} 
“Não negue a justiça ao pobre em um processo legal. 

\smallskip
\textit{\tiny 7} 
“Jamais acuse alguém falsamente. 

\smallskip
Jamais condene à morte uma pessoa
inocente ou íntegra, pois eu nunca declaro inocente aquele que é culpado. 

\smallskip
\textit{\tiny 8} 
“Não aceite subornos, pois eles o levam a fazer vista grossa para algo que se
pode ver claramente. O suborno faz até o justo distorcer a verdade. 

\bigskip
\subsubsection*{O homem, seu semelhante e o descanso}
\textit{\tiny 9} 
“Não explore os estrangeiros. Vocês sabem o que significa viver em terra
estranha, pois foram estrangeiros no Egito.
   
\smallskip
\textit{\tiny 10}
“Plantem e colham os produtos da terra por seis anos, 
\textit{\tiny 11}
mas, no sétimo ano,
deixem que ela se renove e descanse sem cultivo. Permitam que os pobres do
povo colham o que crescer espontaneamente durante esse ano. Deixem o resto
para servir de alimento aos animais selvagens. Façam o mesmo com os vinhedos e
os olivais.
   
\smallskip
\textit{\tiny 12}
“Vocês têm seis dias da semana para realizar suas tarefas habituais, mas não
devem trabalhar no sétimo. Desse modo, seu boi e seu jumento descansarão, e os
escravos e estrangeiros que vivem entre vocês recuperarão as forças.
   
\bigskip
\subsubsection*{O homem, o Eterno e as festas no santuário}
\textit{\tiny 13}
“Prestem muita atenção a todas as minhas instruções. Não invoquem o nome
de outros deuses; nem mesmo mencionem o nome deles.”

\smallskip
\textit{\tiny 14}
“A cada ano, celebrem três festas em minha honra. 

\smallskip
\textit{\tiny 15}
Primeiro, celebrem a Festa
dos Pães sem Fermento. Durante sete dias, o pão que vocês comerem será
preparado sem fermento, conforme eu lhes ordenei. Celebrem essa festa
anualmente no tempo determinado, no mês de abibe,60 pois é o aniversário de sua
partida do Egito. Ninguém deve se apresentar diante de mim de mãos vazias.
   
\smallskip
\textit{\tiny 16}
“Celebrem também a Festa da Colheita,61 quando me trarão os primeiros
frutos de suas colheitas.

\smallskip
   “Por fim, celebrem a Festa da Última Colheita no final da safra, quando
tiverem colhido todos os produtos de seus campos. 

\smallskip
\textit{\tiny 17}
A cada ano, nessas três
ocasiões, todos os homens de Israel devem comparecer diante do Soberano, o
SENHOR.
\textit{\tiny 18}
“Não ofereçam o sangue de meus sacrifícios com pão que contenha
fermento. E não guardem até a manhã seguinte a gordura das ofertas da festa.
\textit{\tiny 19}
“Quando fizerem a colheita, levem à casa do SENHOR, seu Deus, o melhor de
seus primeiros frutos.
   “Não cozinhem o cabrito no leite da mãe dele.”

\bigskip
\subsubsection*{O homem, o Eterno e a Terra Prometida}
\textit{\tiny 20}
“Vejam, eu enviarei um anjo à sua frente para protegê-los ao longo da jornada e
conduzi-los em segurança ao lugar que lhes preparei. 
\textit{\tiny 21}
Prestem muita atenção
nele e obedeçam a suas instruções. Não se rebelem contra ele, pois é meu
representante e não perdoará sua rebeldia. 
\textit{\tiny 22}
Mas, se tiverem o cuidado de lhe
obedecer e de seguir todas as minhas instruções, serei inimigo de seus inimigos e
farei oposição aos que se opuserem a vocês. 
\textit{\tiny 23}
Meu anjo irá à sua frente e os
conduzirá à terra dos amorreus, hititas, ferezeus, cananeus, heveus e jebuseus, e
eu destruirei todas essas nações. 
\textit{\tiny 24}
Não adorem seus deuses, nem os sirvam de
maneira alguma, e nem sequer imitem suas práticas. Antes, destruam-nas
completamente e despedacem suas colunas sagradas.
   
\smallskip
\textit{\tiny 25}
“Sirvam somente ao SENHOR, seu Deus, e eu os abençoarei com alimento e
água e os protegerei de doenças. 
\textit{\tiny 26}
Em sua terra, nenhuma grávida sofrerá
aborto e nenhuma mulher será estéril. Eu lhes darei vida longa e plena.
\textit{\tiny 27}
“Enviarei pavor à sua frente e criarei pânico entre os povos cujas terras vocês
invadirem. Farei todos os seus inimigos darem meia-volta e fugirem. 
\textit{\tiny 28}
Sim,
enviarei terror adiante de vocês para expulsar os heveus, os cananeus e os
hititas, 
\textit{\tiny 29}
mas não os expulsarei num só ano, pois a terra ficaria deserta e os
animais se multiplicariam e se tornariam uma ameaça para vocês. 
\textit{\tiny 30}
Eu os
expulsarei aos poucos, até que sua população tenha aumentado o suficiente para
tomar posse da terra. 
\textit{\tiny 31}
Estabelecerei os limites de seu território desde o mar
Vermelho até o mar Mediterrâneo, e do deserto do leste até o rio Eufrates.Entregarei em suas mãos os povos que hoje vivem na terra e os expulsarei de
diante de vocês.
  
\smallskip
\textit{\tiny 32}
“Não façam tratados com eles nem com seus deuses. 
\textit{\tiny 33}
Esses povos não
devem habitar em sua terra, pois os fariam pecar contra mim. Se vocês servirem
aos deuses deles, cairão na armadilha da idolatria”.

 
\bigskip
\subsubsection*{O homem, o Eterno e a Aliança}
\textbf{\large 24}
 Então o SENHOR disse a Moisés: “Suba ao monte para encontrar-se comigo
e traga Arão, Nadabe, Abiú e setenta líderes de Israel. Todos devem adorar de
longe. 
\textit{\tiny 2} 
Somente Moisés está autorizado a se aproximar do SENHOR. Os outros não
devem chegar perto, e ninguém mais do povo tem permissão de subir ao monte
com ele”. 

\smallskip
\textit{\tiny 3} 
Moisés desceu e transmitiu ao povo todas as instruções e ordens do SENHOR, e
todo o povo respondeu em uma só voz: “Faremos tudo que o SENHOR ordenou!”. 
\smallskip
\textit{\tiny 4} 
Moisés anotou com exatidão todas as instruções do SENHOR. 

\smallskip
Logo cedo na
manhã seguinte, levantou-se e construiu um altar ao pé do monte. Também
ergueu doze colunas, uma para cada tribo de Israel. 
\textit{\tiny 5} 
Em seguida, enviou alguns
rapazes israelitas para apresentarem ao SENHOR holocaustos e touros sacrificados
como ofertas de paz. 
\textit{\tiny 6} 
Moisés colocou em vasilhas metade do sangue desses
animais e aspergiu a outra metade sobre o altar. 


\smallskip
\textit{\tiny 7} 
Depois, pegou o Livro da Aliança e o leu em voz alta para o povo. Mais uma vez,
todos responderam: “Obedeceremos ao SENHOR! Faremos tudo que ele ordenou!”. 
\textit{\tiny 8} 
Moisés pegou o sangue das vasilhas, aspergiu-o sobre o povo e declarou: “Este
sangue confirma a aliança que o SENHOR fez com vocês quando lhes deu estas
instruções”. 

\smallskip
\textit{\tiny 9} 
Depois, Moisés, Arão, Nadabe, Abiú e os setenta líderes de Israel subiram ao
monte, 
\textit{\tiny 10}
onde viram o Deus de Israel, e sob os pés dele havia uma superfície
azulada como a safira e clara como o céu. 
\textit{\tiny 11}
E, embora esses nobres de Israel
tenham visto Deus, ele não os destruiu, e eles participaram de uma refeição na
presença dele.
   
\smallskip
\textit{\tiny 12}
Então o SENHOR disse a Moisés: “Suba ao monte para encontrar-se comigo.
Fique lá e eu lhe darei tábuas de pedra nas quais gravei a lei e os mandamentos
para ensinar ao povo”. 

\smallskip
\textit{\tiny 13}
Moisés e seu auxiliar, Josué, partiram e subiram ao
monte de Deus.    
\textit{\tiny 14}
“Esperem aqui até voltarmos”, disse Moisés aos líderes. “Arão e Hur ficarão
com vocês. Quem tiver algum problema para resolver durante minha ausência
poderá consultá-los.”
\textit{\tiny 15}
Então Moisés subiu ao monte, e a nuvem cobriu o monte. 

\smallskip
\textit{\tiny 16}
A glória do
SENHOR pousou sobre o monte Sinai, e a nuvem o cobriu por seis dias. No sétimo
dia, o SENHOR chamou Moisés de dentro da nuvem. 
\textit{\tiny 17}
Para os israelitas que
estavam ao pé do monte, a glória do SENHOR no alto do Sinai parecia um fogo
consumidor. 

\smallskip
\textit{\tiny 18}
Moisés desapareceu na nuvem ao subir ao monte e ali
permaneceu quarenta dias e quarenta noites.



----------------------------------------------------------------------
