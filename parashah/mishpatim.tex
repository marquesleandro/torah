\section*{}

Os dez mandamentos
∆   20 Então o SENHOR deu ao povo todas estas palavras:
2“Eu  sou o SENHOR, seu Deus, que o libertou da terra do Egito, onde você era
 escravo.
3“Não tenha outros deuses além de mim.
4“Não faça para si espécie alguma de ídolo ou imagem de qualquer coisa no céu,
 na terra ou no mar. 5Não se curve diante deles nem os adore, pois eu, o SENHOR,
 seu Deus, sou um Deus zeloso. Trago as consequências do pecado dos pais sobre
 os filhos até a terceira e quarta geração dos que me rejeitam, 6mas demonstro
 amor por até mil gerações49 dos que me amam e obedecem a meus
 mandamentos.
7“Não use o nome do SENHOR, seu Deus, de forma indevida. O SENHOR não deixará
 impune quem usar o nome dele de forma indevida.
8“Lembre-se de guardar o sábado, fazendo dele um dia santo. 9Você tem seis dias
 na semana para fazer os trabalhos habituais, 10mas o sétimo dia é o sábado do
 SENHOR, seu Deus. Nesse dia, ninguém em sua casa fará trabalho algum: nem
 você, nem seus filhos e filhas, nem seus servos e servas, nem seus animais, nem
 os estrangeiros que vivem entre vocês. 11O SENHOR fez os céus, a terra, o mar e
 tudo que neles há em seis dias; no sétimo dia, porém, descansou. Por isso o
 SENHOR abençoou o sábado e fez dele um dia santo.
12“Honre seu pai e sua mãe. Assim você terá vida longa e plena na terra que o
 SENHOR, seu Deus, lhe dá.
13“Não mate.
14“Não cometa adultério.
15“Não roube.
16“Não dê falso testemunho contra o seu próximo.
17“Não cobice a casa do seu próximo. Não cobice a mulher dele, nem seus servos
 ou servas, nem seu boi ou jumento, nem qualquer outra coisa que lhe pertença”.
  18Quando    o povo ouviu os trovões e o som forte da trombeta, e quando viu o
clarão dos raios e a fumaça que subia do monte, ficou a distância, tremendo de
medo.
   19Disseram a Moisés: “Fale você conosco e ouviremos; mas não deixe que Deus
nos fale diretamente, pois morreríamos!”.
   20Moisés respondeu: “Não tenham medo, pois Deus veio desse modo para
prová-los e para que o temor a ele os impeça de pecar”.
   21Enquanto o povo continuava a distância, Moisés se aproximou da nuvem
escura onde Deus estava.
O uso apropriado dos altares
22O SENHOR disse a Moisés: “Diga ao povo de Israel: Vocês viram com os próprios
olhos que eu lhes falei do céu. 23Lembrem-se de que não devem fazer ídolos de
prata ou ouro que tomem o meu lugar.
  24“Construam para mim um altar feito de terra e nele ofereçam holocaustos e
ofertas de paz, sacrifícios de ovelhas e bois. Em todo lugar onde eu exaltar meu
nome, construam um altar. Eu virei até vocês e os abençoarei. 25Se usarem pedras
para construir meu altar, que sejam apenas pedras inteiras, em sua forma natural.
Não alterem a forma das pedras com alguma ferramenta, pois isso tornaria o altar
impróprio para o uso sagrado. 26E não usem degraus para chegarem diante do
meu altar, para que sua nudez não seja exposta.”
O justo tratamento dos escravos
∆   21 “Estes são os decretos que você apresentará a Israel:
    2“Sevocê comprar um escravo hebreu, ele não poderá servi-lo por mais de seis
anos. Liberte-o no sétimo ano, e ele nada lhe deverá pela liberdade. 3Se ele era
solteiro quando se tornou seu escravo, partirá solteiro. Mas, se era casado antes de
se tornar seu escravo, a esposa deverá ser liberta com ele.
   4“Se seu senhor lhe deu uma mulher em casamento enquanto ele era escravo, e
se o casal teve filhos e filhas, somente o homem será liberto no sétimo ano. A
mulher e os filhos continuarão a pertencer ao senhor. 5O escravo, contudo,
poderá declarar: ‘Amo meu senhor, minha esposa e meus filhos. Não desejo ser
liberto’. 6Nesse caso, seu senhor o apresentará aos juízes.50 Em seguida, o levará
até a porta ou até o batente da porta e furará a orelha dele com um furador.
Depois disso, o escravo servirá a seu senhor pelo resto da vida.
   7“Quando um homem vender a filha como escrava, ela não será liberta como os
homens. 8Se ela não agradar seu senhor, ele permitirá que alguém lhe pague o
resgate, mas não poderá vendê-la a estrangeiros, pois rompeu o contrato com ela.
9Mas, se o senhor da escrava a entregar como mulher ao filho dele, não a tratará
mais como escrava, mas sim como filha.
   10“Se um homem que se casou com uma escrava tomar para si outra esposa,
não deverá descuidar dos direitos da primeira mulher com respeito a
alimentação, vestuário e intimidade sexual. 11Se ele não cumprir alguma dessas
obrigações, ela poderá sair livre, sem pagar coisa alguma.”
Casos de danos pessoais
12“Quem agredir e matar outra pessoa será executado, 13mas se for apenas um
acidente permitido por Deus, definirei um lugar de refúgio para onde o
responsável pela morte possa fugir. 14Se, contudo, alguém matar outra pessoa
intencionalmente, o assassino será preso e executado, mesmo que tenha buscado
refúgio em meu altar.
   15“Quem agredir seu pai ou sua mãe será executado.
   16“Quem sequestrar alguém será executado, quer a vítima seja encontrada em
seu poder, quer ele a tenha vendido como escrava.
  17“Quem ofender a honra de51 seu pai ou de sua mãe será executado.
  18“Se dois homens brigarem e um deles acertar o outro com uma pedra ou com
o punho e o outro não morrer, mas ficar de cama, 19o agressor não será castigado
se, posteriormente, o que foi ferido conseguir voltar a andar fora de casa, mesmo
que precise de muletas; o agressor indenizará a vítima pelos salários que ela
perder e se responsabilizará por sua total recuperação.
   20“Se um senhor espancar seu escravo ou sua escrava com uma vara e, como
resultado, o escravo morrer, o senhor será castigado. 21Mas, se o escravo se
recuperar em um ou dois dias, o senhor não receberá castigo algum, pois o
escravo é sua propriedade.
   22“Se dois homens brigarem e um deles atingir, por acidente, uma mulher
grávida e ela der à luz prematuramente,52 sem que haja outros danos, o homem
que atingiu a mulher pagará a indenização que o marido dela exigir e os juízes
aprovarem. 23Mas, se houver outros danos, o castigo deverá corresponder à
gravidade do dano causado: vida por vida, 24olho por olho, mão por mão, pé por
pé, 25queimadura por queimadura, ferida por ferida, contusão por contusão.
   26“Se um senhor ferir seu escravo ou sua escrava no olho e o cegar, libertará o
escravo como compensação pelo olho. 27Se quebrar o dente de seu escravo ou de
sua escrava, libertará o escravo como compensação pelo dente.
   28“Se um boi53 matar a chifradas um homem ou uma mulher, o boi será
apedrejado, e não será permitido comer sua carne. Nesse caso, porém, o dono do
boi não será responsabilizado. 29Mas, se o boi costumava chifrar pessoas e o dono
havia sido informado, porém não manteve o animal sob controle, se o boi matar
alguém, será apedrejado, e o dono também será executado. 30Os parentes do
morto, no entanto, poderão aceitar uma indenização pela vida perdida. O dono do
boi poderá resgatar a própria vida ao pagar o que for exigido.
   31“A mesma lei se aplica se o boi chifrar um menino ou uma menina. 32Mas, se
o boi chifrar um escravo ou uma escrava, o dono do boi pagará trinta moedas54 de
prata ao senhor do escravo, e o boi será apedrejado.
   33“Se alguém cavar ou destampar um poço e um boi ou jumento cair dentro
dele, 34o proprietário do poço indenizará totalmente o dono do animal, mas
poderá ficar com o animal morto.
   35“Se o boi de alguém ferir o boi do vizinho e o animal ferido morrer, os dois
donos venderão o animal vivo e dividirão o dinheiro entre si em partes iguais;
também dividirão entre si o animal morto. 36Mas, se o boi costumava chifrar e o
dono não manteve o animal sob controle, o dono entregará um boi vivo como
indenização pelo boi morto e poderá ficar com o animal morto.”
Proteção da propriedade
∆   22 155
           “Se alguém roubar um boi56 ou uma ovelha e matar o animal ou vendê-
lo, o ladrão pagará cinco bois para cada boi roubado e quatro ovelhas para cada
ovelha roubada.
   257 “Se um ladrão for pego em flagrante arrombando uma casa e for ferido e
morto no confronto, a pessoa que matou o ladrão não será culpada de homicídio.
3Mas, se isso acontecer durante o dia, a pessoa que matou o ladrão será culpada
de homicídio.
   “O ladrão que for pego restituirá o valor total daquilo que roubou. Se não puder
restituir o valor, será vendido como escravo para pagar pelos bens roubados. 4Se
alguém roubar um boi, um jumento ou uma ovelha e o animal for encontrado
vivo, em poder do ladrão, ele pagará o dobro do valor do animal roubado.
   5“Se um animal estiver pastando no campo ou na videira e o dono o soltar para
pastar no campo de outra pessoa, o dono do animal entregará como indenização o
melhor de seus cereais ou de suas uvas.
   6“Se alguém estiver queimando espinheiros e o fogo se espalhar para o campo
de outra pessoa e destruir o cereal já colhido, ou a plantação pronta para a
colheita, ou a lavoura inteira, aquele que começou o fogo pagará por todo o
prejuízo.
   7“Se alguém entregar valores ou bens a um vizinho para que este os guarde e
eles forem roubados da casa do vizinho, o ladrão, se for pego, restituirá o dobro
do valor dos itens roubados. 8Mas, se o ladrão não for pego, o dono da casa
comparecerá diante dos juízes58 para que se determine se foi ele quem roubou os
bens.
   9“Em qualquer caso de disputa entre vizinhos em que ambos afirmem ser
donos de determinado boi, jumento, ovelha, peça de roupa ou objeto perdido, as
duas partes comparecerão diante dos juízes, e a pessoa que eles considerarem59
culpada pagará o dobro à outra.
   10“Se alguém deixar um jumento, um boi, uma ovelha ou outro animal sob os
cuidados de outra pessoa e o animal morrer, for ferido ou levado embora, e
ninguém vir o que aconteceu, 11a pessoa que estava cuidando do animal fará
diante do SENHOR um juramento de que não roubou o animal; o dono aceitará o
juramento e não será exigido pagamento algum. 12Mas, se o animal for roubado
do vizinho, ele indenizará o dono. 13Se tiver sido despedaçado por um animal
selvagem, o que restou da carcaça será apresentado como prova, e não será
exigido pagamento algum.
   14“Se alguém pedir um animal emprestado ao vizinho e o animal for ferido ou
morrer na ausência do dono, a pessoa que pediu o animal emprestado indenizará
o dono totalmente. 15Mas, se o dono estiver presente, não será exigido pagamento
algum. Também não será exigida indenização alguma se o animal tiver sido
alugado, pois o valor do aluguel cobrirá a perda.”
Responsabilidades gerais
16“Se um homem seduzir uma moça virgem que não esteja comprometida e tiver
relações sexuais com ela, pagará à família dela o preço costumeiro do dote e se
casará com ela. 17Mas, se o pai da moça não permitir o casamento, o homem lhe
pagará o equivalente ao dote de uma virgem.
   18“Não deixe que a feiticeira viva.
   19“Quem tiver relações sexuais com um animal certamente será executado.
   20“Quem sacrificar a qualquer outro deus além do SENHOR será destruído.
   21“Não maltrate nem oprima os estrangeiros. Lembre-se de que vocês também
foram estrangeiros na terra do Egito.
   22“Não explore a viúva nem o órfão. 23Se você os explorar e eles clamarem a
mim, certamente ouvirei seu clamor. 24Minha ira se acenderá contra você e o
matarei pela espada. Então sua esposa ficará viúva e seus filhos ficarão órfãos.
   25“Se você emprestar dinheiro a alguém do meu povo que esteja necessitado,
não cobre juros visando lucro, como fazem os credores. 26Se tomar a capa do seu
próximo como garantia para um empréstimo, devolva-a antes do pôr do sol.
27Talvez a capa seja a única coberta que ele tem para se aquecer. Como ele poderá
dormir sem ela? Se não a devolver e se o seu próximo pedir socorro a mim, eu o
ouvirei, pois sou misericordioso.
   28“Não blasfeme contra Deus nem amaldiçoe as autoridades do seu povo.
   29“Quando entregar as ofertas das colheitas, do vinho e do azeite, não retenha
coisa alguma.
   “Consagre a mim seu primeiro filho.
   30“Também entregue a mim os machos das primeiras crias das vacas, das
ovelhas e das cabras. Deixe o animal com a mãe por sete dias e, no oitavo,
entregue-o a mim.
           serão meu povo santo. Por isso, não comam a carne de animais
    31“Vocês
despedaçados e mortos por feras no campo; joguem a carne para os cães.”
Um chamado à prática da justiça
∆   23 “Não espalhe boatos falsos. Não coopere com pessoas perversas sendo
falsa testemunha.
   2“Não se deixe levar pela maioria na prática do mal. Quando o chamarem para
testemunhar em um processo legal, não permita que a multidão o influencie a
perverter a justiça. 3E não incline seu testemunho em favor de uma pessoa só
porque ela é pobre.
   4“Se você deparar com o boi ou o jumento perdido de seu inimigo, leve-o de
volta ao dono. 5Se vir o jumento de alguém que o odeia cair sob o peso de sua
carga, não faça de conta que não viu. Pare e ajude o dono a levantá-lo.
   6“Não negue a justiça ao pobre em um processo legal.
   7“Jamais acuse alguém falsamente. Jamais condene à morte uma pessoa
inocente ou íntegra, pois eu nunca declaro inocente aquele que é culpado.
   8“Não aceite subornos, pois eles o levam a fazer vista grossa para algo que se
pode ver claramente. O suborno faz até o justo distorcer a verdade.
   9“Não explore os estrangeiros. Vocês sabem o que significa viver em terra
estranha, pois foram estrangeiros no Egito.
   10“Plantem e colham os produtos da terra por seis anos, 11mas, no sétimo ano,
deixem que ela se renove e descanse sem cultivo. Permitam que os pobres do
povo colham o que crescer espontaneamente durante esse ano. Deixem o resto
para servir de alimento aos animais selvagens. Façam o mesmo com os vinhedos e
os olivais.
   12“Vocês têm seis dias da semana para realizar suas tarefas habituais, mas não
devem trabalhar no sétimo. Desse modo, seu boi e seu jumento descansarão, e os
escravos e estrangeiros que vivem entre vocês recuperarão as forças.
   13“Prestem muita atenção a todas as minhas instruções. Não invoquem o nome
de outros deuses; nem mesmo mencionem o nome deles.”
Três festas anuais
14“A cada ano, celebrem três festas em minha honra. 15Primeiro, celebrem a Festa
dos Pães sem Fermento. Durante sete dias, o pão que vocês comerem será
preparado sem fermento, conforme eu lhes ordenei. Celebrem essa festa
anualmente no tempo determinado, no mês de abibe,60 pois é o aniversário de sua
partida do Egito. Ninguém deve se apresentar diante de mim de mãos vazias.
   16“Celebrem também a Festa da Colheita,61 quando me trarão os primeiros
frutos de suas colheitas.
   “Por fim, celebrem a Festa da Última Colheita62 no final da safra, quando
tiverem colhido todos os produtos de seus campos. 17A cada ano, nessas três
ocasiões, todos os homens de Israel devem comparecer diante do Soberano, o
SENHOR.
   18“Não ofereçam o sangue de meus sacrifícios com pão que contenha
fermento. E não guardem até a manhã seguinte a gordura das ofertas da festa.
   19“Quando fizerem a colheita, levem à casa do SENHOR, seu Deus, o melhor de
seus primeiros frutos.
   “Não cozinhem o cabrito no leite da mãe dele.”
Promessa da presença do SENHOR
20“Vejam, eu enviarei um anjo à sua frente para protegê-los ao longo da jornada e
conduzi-los em segurança ao lugar que lhes preparei. 21Prestem muita atenção
nele e obedeçam a suas instruções. Não se rebelem contra ele, pois é meu
representante e não perdoará sua rebeldia. 22Mas, se tiverem o cuidado de lhe
obedecer e de seguir todas as minhas instruções, serei inimigo de seus inimigos e
farei oposição aos que se opuserem a vocês. 23Meu anjo irá à sua frente e os
conduzirá à terra dos amorreus, hititas, ferezeus, cananeus, heveus e jebuseus, e
eu destruirei todas essas nações. 24Não adorem seus deuses, nem os sirvam de
maneira alguma, e nem sequer imitem suas práticas. Antes, destruam-nas
completamente e despedacem suas colunas sagradas.
   25“Sirvam somente ao SENHOR, seu Deus, e eu os abençoarei com alimento e
água e os protegerei63 de doenças. 26Em sua terra, nenhuma grávida sofrerá
aborto e nenhuma mulher será estéril. Eu lhes darei vida longa e plena.
   27“Enviarei pavor à sua frente e criarei pânico entre os povos cujas terras vocês
invadirem. Farei todos os seus inimigos darem meia-volta e fugirem. 28Sim,
enviarei terror64 adiante de vocês para expulsar os heveus, os cananeus e os
hititas, 29mas não os expulsarei num só ano, pois a terra ficaria deserta e os
animais se multiplicariam e se tornariam uma ameaça para vocês. 30Eu os
expulsarei aos poucos, até que sua população tenha aumentado o suficiente para
tomar posse da terra. 31Estabelecerei os limites de seu território desde o mar
Vermelho até o mar Mediterrâneo,65 e do deserto do leste até o rio Eufrates.66
Entregarei em suas mãos os povos que hoje vivem na terra e os expulsarei de
diante de vocês.
  32“Não façam tratados com eles nem com seus deuses. 33Esses povos não
devem habitar em sua terra, pois os fariam pecar contra mim. Se vocês servirem
aos deuses deles, cairão na armadilha da idolatria”.
Israel aceita a aliança do SENHOR
∆   24 Então o SENHOR disse a Moisés: “Suba ao monte para encontrar-se comigo
e traga Arão, Nadabe, Abiú e setenta líderes de Israel. Todos devem adorar de
longe. 2Somente Moisés está autorizado a se aproximar do SENHOR. Os outros não
devem chegar perto, e ninguém mais do povo tem permissão de subir ao monte
com ele”.
   3Moisés desceu e transmitiu ao povo todas as instruções e ordens do SENHOR, e
todo o povo respondeu em uma só voz: “Faremos tudo que o SENHOR ordenou!”.
   4Moisés anotou com exatidão todas as instruções do SENHOR. Logo cedo na
manhã seguinte, levantou-se e construiu um altar ao pé do monte. Também
ergueu doze colunas, uma para cada tribo de Israel. 5Em seguida, enviou alguns
rapazes israelitas para apresentarem ao SENHOR holocaustos e touros sacrificados
como ofertas de paz. 6Moisés colocou em vasilhas metade do sangue desses
animais e aspergiu a outra metade sobre o altar.
   7Depois, pegou o Livro da Aliança e o leu em voz alta para o povo. Mais uma vez,
todos responderam: “Obedeceremos ao SENHOR! Faremos tudo que ele ordenou!”.
   8Moisés pegou o sangue das vasilhas, aspergiu-o sobre o povo e declarou: “Este
sangue confirma a aliança que o SENHOR fez com vocês quando lhes deu estas
instruções”.
   9Depois, Moisés, Arão, Nadabe, Abiú e os setenta líderes de Israel subiram ao
monte, 10onde viram o Deus de Israel, e sob os pés dele havia uma superfície
azulada como a safira e clara como o céu. 11E, embora esses nobres de Israel
tenham visto Deus, ele não os destruiu, e eles participaram de uma refeição na
presença dele.
   12Então o SENHOR disse a Moisés: “Suba ao monte para encontrar-se comigo.
Fique lá e eu lhe darei tábuas de pedra nas quais gravei a lei e os mandamentos
para ensinar ao povo”. 13Moisés e seu auxiliar, Josué, partiram e subiram ao
monte de Deus.
    14“Esperemaqui até voltarmos”, disse Moisés aos líderes. “Arão e Hur ficarão
com vocês. Quem tiver algum problema para resolver durante minha ausência
poderá consultá-los.”
   15Então Moisés subiu ao monte, e a nuvem cobriu o monte. 16A glória do
SENHOR pousou sobre o monte Sinai, e a nuvem o cobriu por seis dias. No sétimo
dia, o SENHOR chamou Moisés de dentro da nuvem. 17Para os israelitas que
estavam ao pé do monte, a glória do SENHOR no alto do Sinai parecia um fogo
consumidor. 18Moisés desapareceu na nuvem ao subir ao monte e ali
permaneceu quarenta dias e quarenta noites.
Ofertas para o tabernáculo



----------------------------------------------------------------------
