\section*{Parashat Noach 6:9 - 11:32}
\subsubsection*{Noé e a arca}
\textit{\tiny 9}
 Este é o relato de Noé e sua família. Noé era um homem justo, a única pessoa íntegra naquele tempo, e andava em comunhão com Deus. 
\textit{\tiny 10}
Noé gerou três filhos: Sem, Cam e Jafé.



\bigskip
\textit{\tiny 11}
 Deus viu que a terra tinha se corrompido e estava cheia de violência. 
\textit{\tiny 12}
 Deus observou a grande maldade no mundo, pois todos na terra haviam se corrompido. 
\textit{\tiny 13}
 Assim, Deus disse a Noé: “Decidi acabar com todos os seres vivos, pois encheram a terra de violência. Sim, destruirei todos eles e também a terra!



\bigskip
\textit{\tiny 14}
 “Construa uma grande embarcação, uma arca de madeira de cipreste, e cubraa com betume por dentro e por fora, para que não entre água. Divida toda a parte interna em pisos e compartimentos. 
\textit{\tiny 15}
 A arca deve ter 135 metros de comprimento, 
22,5metros de largura e 
13,5metros de altura. 
\textit{\tiny 16}
 Deixe uma abertura de 45 centímetros debaixo do teto ao redor de toda a arca. Coloque uma porta lateral e construa três pisos na parte interna: inferior, médio e superior.



\bigskip
\textit{\tiny 17}
 “Preste atenção! Em breve, cobrirei a terra com um dilúvio que destruirá todos os seres vivos que respiram. Tudo que há na terra morrerá. 
\textit{\tiny 18}
 Com você, porém, firmarei minha aliança. Portanto, entre na arca com sua mulher, seus filhos e as mulheres deles. 
\textit{\tiny 19}
 Leve na arca com você um casal de cada espécie de animal selvagem e doméstico, um macho e uma fêmea, para mantê-los com vida. 
\textit{\tiny 20}
 Um casal de cada espécie de ave, de cada espécie de animal e de cada espécie de animal que rasteja pelo chão virá até você, para que os mantenha com vida. 
\textit{\tiny 21}
 Cuide bem para que haja alimento suficiente para sua família e para todos os animais”.



\bigskip
\textit{\tiny 22}
 Noé fez tudo exatamente como Deus lhe havia ordenado.

\bigskip
\subsubsection*{Inicio do diluvio}
\textbf{\large 7}
 O SENHOR disse a Noé: “Entre na arca com toda a sua família, pois vejo que, de todas as pessoas na terra, apenas você é justo. 
\textit{\tiny 2}
 Leve com você sete casais, macho e fêmea, de cada espécie de animal puro, e um casal, macho e fêmea, de cada espécie de animal impuro. 
\textit{\tiny 3}
 Leve também sete casais de cada espécie de ave. Cada casal deve ter um macho e uma fêmea para garantir que todas as espécies sobreviverão na terra depois do dilúvio. 
\textit{\tiny 4}
 Daqui a sete dias, farei chover sobre a terra. Choverá por quarenta dias e quarenta noites, até que eu tenha eliminado da terra todos os seres vivos que criei”.



\bigskip
\textit{\tiny 5}
 Noé fez tudo exatamente como o SENHOR lhe havia ordenado. 
\textit{\tiny 6}
 Noé tinha 600 anos quando o dilúvio cobriu a terra. 
\textit{\tiny 7}
 Entrou na arca, junto com a mulher, os filhos e as mulheres deles, para escapar do dilúvio. 
\textit{\tiny 8}
 Entraram com eles animais de todas as espécies: os puros e os impuros, as aves e todos os animais que rastejam pelo chão. 
\textit{\tiny 9}
 Entraram na arca em pares, macho e fêmea, como Deus tinha ordenado a Noé. 
\textit{\tiny 10}
 Depois de sete dias, vieram as águas do dilúvio e cobriram a terra.



\bigskip
\textit{\tiny 11}
 Quando Noé tinha 600 anos, no décimo sétimo dia do segundo mês, todas as fontes subterrâneas de água jorraram da terra, e a chuva caiu do céu em grandes temporais 
\textit{\tiny 12}
 e continuou sem parar por quarenta dias e quarenta noites. 
\textit{\tiny 13}
 Naquele mesmo dia, Noé tinha entrado na arca com a esposa, os filhos, Sem, Cam e Jafé, e as mulheres deles. 
\textit{\tiny 14}
 Entraram com eles na arca casais de todas as espécies de animais: animais domésticos e selvagens, grandes e pequenos, e aves de toda espécie. 
\textit{\tiny 15}
 Entraram de dois em dois na arca, representando todos os seres vivos que respiram. 
\textit{\tiny 16}
 Um macho e uma fêmea de cada espécie entraram, como Deus tinha ordenado a Noé. 
Então o SENHOR fechou a porta. 

\bigskip
\textit{\tiny 17}
 Durante quarenta dias, as águas do dilúvio se tornaram cada vez mais profundas, cobriram o solo e elevaram a arca bem acima da terra. 
\textit{\tiny 18}
 Enquanto as águas subiam cada vez mais acima do solo, a arca flutuava em segurança em sua superfície. 
\textit{\tiny 19}
 Por fim, as águas cobriram até as montanhas mais altas da terra 
\textit{\tiny 20}
 e se elevaram quase sete metros acima dos picos mais altos. 
\textit{\tiny 21}
 Todos os seres vivos que havia na terra morreram: as aves, os animais domésticos, os animais selvagens, os animais que rastejavam pelo chão e todos os seres humanos. 
\textit{\tiny 22}
 Tudo que respirava e vivia em terra firme morreu. 
\textit{\tiny 23}
 Deus exterminou todos os seres vivos que havia na terra: os seres humanos, os animais domésticos, os animais que rastejavam pelo chão e as aves do céu. Todos foram destruídos. Apenas Noé e os que estavam com ele na arca sobreviveram.



\bigskip
\textit{\tiny 24}
 E as águas do dilúvio cobriram a terra por 150 dias.



\bigskip
\subsubsection*{Fim do diluvio}
\textbf{\large 8}
 Então Deus se lembrou de Noé e de todos os animais selvagens e domésticos que estavam com ele na arca. Deus fez soprar um vento sobre a terra, e as águas do dilúvio começaram a baixar. 
\textit{\tiny 2}
 As fontes subterrâneas pararam de jorrar, e as chuvas torrenciais cessaram. 
\textit{\tiny 3}
 As águas do dilúvio foram baixando aos poucos.

\bigskip
Depois de 150 dias, 
\textit{\tiny 4}
 exatamente cinco meses depois do início do dilúvio, a arca repousou sobre as montanhas de Ararate. 
\textit{\tiny 5}
 Dois meses e meio depois, à medida que as águas continuaram a baixar, apareceram os picos de outras montanhas. 
\textit{\tiny 6}
 Passados mais quarenta dias, Noé abriu a janela que havia feito na arca 
\textit{\tiny 7}
 e soltou um corvo, que ia e voltava até as águas do dilúvio secarem sobre a terra. 
\textit{\tiny 8}
 Noé também soltou uma pomba para ver se as águas tinham baixado e se ela encontraria terra seca, 
\textit{\tiny 9}
 mas a pomba não encontrou lugar para pousar, pois a água ainda cobria todo osolo. Então a pomba retornou à arca,e Noé estendeu a mão e a trouxe de volta para dentro.


\bigskip
\textit{\tiny 10}
 Depois de esperar mais sete dias, Noé soltou a pomba mais uma vez. 
\textit{\tiny 11}
 Quando ela voltou ao entardecer, trouxe no bico uma folha nova de oliveira. Noé concluiu que restava pouca água do dilúvio. 
\textit{\tiny 12}
 Esperou outros sete dias e soltou a pomba novamente. Dessa vez, ela não voltou.


\bigskip
\textit{\tiny 13}
 Noé tinha completado 601 anos. No primeiro dia do novo ano, dez meses e meio depois do início do dilúvio, quase não havia mais água sobre a terra. Noé levantou a cobertura da arca e viu que o solo estava praticamente seco. 
\textit{\tiny 14}
 Mais dois meses se passaram e, por fim, a terra estava completamente seca.


\bigskip
\textit{\tiny 15}
 Então Deus disse a Noé: 
\textit{\tiny 16}
 “Saiam da arca, você, sua mulher, seus filhos e as mulheres deles. 
\textit{\tiny 17}
 Solte todos os animais, as aves, os animais domésticos e os animais que rastejam pelo chão, para que sejam férteis e se multipliquem na terra”. 
\textit{\tiny 18}
 Noé, sua mulher, seus filhos e as mulheres deles desembarcaram. 
\textit{\tiny 19}
 Todos os animais, grandes e pequenos, e as aves saíram da arca, um casal de cada vez.


\bigskip
\subsubsection*{Noé e a aliança}
\textit{\tiny 20}
 Em seguida, Noé construiu um altar ao SENHOR e ali ofereceu como holocaustos alguns animais e aves puros. 
\textit{\tiny 21}
 O aroma do sacrifício agradou ao SENHOR, que disse consigo: “Nunca mais amaldiçoarei a terra por causa do ser humano, embora todos os seus pensamentos e seus propósitos se inclinem para o mal desde a infância. Nunca mais destruirei todos os seres vivos. 
\textit{\tiny 22}
 Enquanto durar a terra, haverá plantio e colheita, frio e calor, verão e inverno, dia e noite”.

\bigskip
\textbf{\large 9}
 Então Deus abençoou Noé e seus filhos e lhes disse: “Sejam férteis e multipliquem-se. Encham a terra. 
\textit{\tiny 2}
 Todos os animais da terra, todas as aves do céu, todos os animais que rastejam pelo chão e todos os peixes do mar terão medo e pavor de vocês. Eu os coloquei sob o seu domínio. 
\textit{\tiny 3}
 Assim como dei a vocês os cereais e os vegetais por alimento, também lhes dou os animais. 
\textit{\tiny 4}
 Mas nunca comam carne com sangue, pois sangue é vida. 
\textit{\tiny 5}
 “Exigirei o sangue de todo aquele que tirar a vida de alguém. Se um animal selvagem matar alguém, deverá ser morto; quem cometer assassinato, também deverá morrer. 
\textit{\tiny 6}
 Quem tirar a vida humana, por mãos humanas perderá a vida. Pois eu criei o ser humano à minha imagem. 
\textit{\tiny 7}
 Agora, sejam férteis e multipliquem-se, povoem a terra outra vez”.



\bigskip
\textit{\tiny 8}
 Então Deus disse a Noé e seus filhos: 
\textit{\tiny 9}
 “Confirmo aqui a minha aliança com vocês, seus descendentes 
\textit{\tiny 10}
 e todos os animais que estavam com vocês na embarcação: as aves, os animais domésticos e os animais selvagens, todos os seres vivos da terra. 
\textit{\tiny 11}
 Sim, confirmo a minha aliança com vocês. Nunca mais os seres vivos serão exterminados pelas águas; nunca mais a terra será destruída por um dilúvio”. 
\textit{\tiny 12}
 Então Deus disse: “Eu lhes dou um sinal da minha aliança com vocês e com todos os seres vivos, para todas as gerações futuras. 
\textit{\tiny 13}
 Coloquei o arco-íris nas nuvens. Ele é o sinal da minha aliança com toda a terra. 
\textit{\tiny 14}
 Quando eu enviar nuvens sobre a terra, nelas aparecerá o arco-íris, 
\textit{\tiny 15}
 e eu me lembrarei da minha aliança com vocês e com todos os seres vivos. Nunca mais as águas de um dilúvio destruirão toda a vida. 
\textit{\tiny 16}
 Ao olhar para o arco-íris nas nuvens, eu me lembrarei da aliança eterna entre Deus e todos os seres vivos da terra”.



\bigskip
\textit{\tiny 17}
 Então Deus disse a Noé: “Este arco-íris é o sinal da aliança que confirmo com todas as criaturas da terra”.



\bigskip
\subsubsection*{Descendentes de Noé}
\textit{\tiny 18}
 Os filhos de Noé que saíram da arca com o pai foram Sem, Cam e Jafé. (Cam é o pai de Canaã.) 
\textit{\tiny 19}
 Desses três filhos de Noé vêm todas as pessoas que agora povoam a terra.



\bigskip
\textit{\tiny 20}
 Depois do dilúvio, Noé começou a cultivar o solo e plantou uma videira. 
\textit{\tiny 21}
 Certo dia, bebeu do vinho que ele próprio havia produzido, ficou embriagado e foi deitarse nu em sua tenda. 
\textit{\tiny 22}
 Cam, pai de Canaã, viu que seu pai estava nu e saiu para contar aos irmãos. 
\textit{\tiny 23}
 Então Sem e Jafé pegaram um manto e o colocaram sobre os ombros. Em seguida, entraram na tenda de costas e, olhando para o outro lado a fim de não ver a nudez do pai, cobriramno com o manto.


\bigskip
\textit{\tiny 24}
 Quando Noé se recuperou da bebedeira e descobriu o que Cam, seu filho mais novo, havia feito, 
\textit{\tiny 25}
 exclamou: “Maldito seja Canaã! Que ele seja o servo mais insignificante de seus parentes!”. 
\textit{\tiny 26}
 E disse ainda: “Bendito seja o SENHOR, o Deus de Sem, e que Canaã seja servo de seu irmão! 
\textit{\tiny 27}
 Que Deus amplie o território de Jafé! Que Jafé compartilhe da prosperidade de Sem e Canaã seja seu servo”.


\bigskip
\textit{\tiny 28}
 Depois do dilúvio, Noé viveu mais 350 anos. Viveu, ao todo, 950 anos e morreu.


\bigskip
\textbf{\large 10}
 Este é o relato das famílias de Sem, Cam e Jafé, os três filhos de Noé, que geraram muitos filhos depois do dilúvio.


\bigskip
\textit{\tiny 2}
 Os descendentes de Jafé foram: Gômer, Magogue, Madai, Javã, Tubal, Meseque e Tirás. 
\textit{\tiny 3}
 Os descendentes de Gômer foram: Asquenaz, Rifate e Togarma. 
\textit{\tiny 4}
 Os descendentes de Javã foram: Elisá, Társis, Quitim e Rodanim. 
\textit{\tiny 5}
 Seus descendentes se espalharam por vários territórios junto ao mar, formando nações de acordo com suas línguas, seus clãs e seus povos.


\bigskip
\textit{\tiny 6}
 Os descendentes de Cam foram: Cuxe, Mizraim, Putee Canaã. 
\textit{\tiny 7}
 Os descendentes de Cuxeforam: Sebá, Havilá, Sabtá, Raamá e Sabtecá. Os descendentes de Raamá foram: Sabá e Dedã. 
\textit{\tiny 8}
 Cuxe também foi o antepassado de Ninrode, o primeiro guerreiro valente da terra. 
\textit{\tiny 9}
 Porque era o mais corajoso dos caçadores, seu nome deu origem ao provérbio: “Este homem é como Ninrode, o mais corajoso dos caçadores”. 
\textit{\tiny 10}
 Ninrode construiu seu reino na terra da Babilônia, fundando as cidades de Babel, Ereque, Acade e Calné. 
\textit{\tiny 11}
 Expandiu seu território até a Assíria, onde construiu as cidades de Nínive, Reobote-Ir, Calá 
\textit{\tiny 12}
 e Resém, a grande cidade situada entre Nínive e Calá. 
\textit{\tiny 13}
 Mizraim foi o antepassado dos luditas, anamitas, leabitas, naftuítas, 
\textit{\tiny 14}
 patrusitas, casluítas e dos caftoritas, dos quais descendem os filisteus. 
\textit{\tiny 15}
 O filho mais velho de Canaã foi Sidom, antepassado dos sidônios. Canaã foi o antepassado dos hititas, 
\textit{\tiny 16}
 jebuseus, amorreus, girgaseus, 
\textit{\tiny 17}
 heveus, arqueus, sineus, 
\textit{\tiny 18}
 arvadeus, zemareus e hamateus. Com o tempo, os clãs cananeus se espalharam. 
\textit{\tiny 19}
 O território de Canaã se estendia desde Sidom, ao norte, até Gerar e Gaza, ao sul, e, a leste, até Sodoma, Gomorra, Admá e Zeboim, próximo a Lasa. 
\textit{\tiny 20}
 Esses foram os descendentes de Cam, de acordo com seus clãs, línguas, territórios e povos.


\bigskip
\textit{\tiny 21}
 Sem, irmão mais velho de Jafé, também teve filhos. Sem foi o antepassado de todos os descendentes de Héber. 
\textit{\tiny 22}
 Os descendentes de Sem foram: Elão, Assur, Arfaxade, Lude e Arã. 
\textit{\tiny 23}
 Os descendentes de Arã foram: Uz, Hul, Géter e Más. 
\textit{\tiny 24}
 Arfaxade gerou Salá, e Salá gerou Héber. 
\textit{\tiny 25}
 Héber teve dois filhos. O primeiro recebeu o nome de Pelegue, pois em sua época a terra foi dividida. O irmão de Pelegue recebeu o nome de Joctã. 
\textit{\tiny 26}
 Joctã foi o antepassado de Almodá, Salefe, Hazarmavé, Jerá, 
\textit{\tiny 27}
 Adorão, Uzal, Dicla, 
\textit{\tiny 28}
 Obal, Abimael, Sabá, 
\textit{\tiny 29}
 Ofir, Havilá e Jobabe. Todos eles foram descendentes de Joctã. 
\textit{\tiny 30}
 O território que ocupavam se estendia desde Messa até Sefar, nas montanhas ao leste. 
\textit{\tiny 31}
 Esses foram os descendentes de Sem, de acordo com seus clãs, línguas, territórios e povos.



\bigskip
\textit{\tiny 32}
 Esses foram os clãs descendentes dos filhos de Noé, de acordo com suas linhagens. Todas as nações da terra vieram desses clãs depois do dilúvio.



\bigskip
\subsubsection*{A criação das linguas}
\textbf{\large 11}
 Houve um tempo em que todos os habitantes do mundo falavam a mesma língua e usavam as mesmas palavras. 
\textit{\tiny 2}
 Ao migrarem do leste, encontraram uma planície na terra da Babilônia, onde se estabeleceram. 
\textit{\tiny 3}
 Começaram a dizer uns aos outros: “Venham, vamos fazer tijolos e endurecê-los no fogo”. (Naquela região, era costume usar tijolos em vez de pedras, e betume em vez de argamassa.) 
\textit{\tiny 4}
 Depois, disseram: “Venham, vamos construir uma cidade com uma torre que chegue até o céu. Assim, ficaremos famosos e não seremos espalhados pelo mundo”.



\bigskip
\textit{\tiny 5}
 O SENHOR, porém, desceu para ver a cidade e a torre que estavam construindo. 
\textit{\tiny 6}
 “Vejam!”, disse o SENHOR. “Todos se uniram e falam a mesma língua. Se isto é o começo do que fazem, nada do que se propuserem a fazer daqui em diante lhes será impossível. 
\textit{\tiny 7}
 Venham, vamos descer e confundi-los com línguas diferentes, para que não consigam mais entender uns aos outros.” 
\textit{\tiny 8}
 Assim, o SENHOR os espalhou pelo mundo inteiro, e eles pararam de construir a cidade. 
\textit{\tiny 9}
 Ela recebeu o nome de Babel, pois ali o SENHOR confundiu as pessoas com línguas diferentes e as espalhou pelo mundo.



\bigskip
\subsubsection*{Descendentes de Sem}
\textit{\tiny 10}
 Este é o relato da família de Sem. Dois anos depois do dilúvio, aos 100 anos, Sem gerou Arfaxade. 
\textit{\tiny 11}
 Depois do nascimento de Arfaxade, Sem viveu mais 500 anos e teve outros filhos e filhas



\bigskip
\textit{\tiny 12}
 Aos 35 anos, Arfaxade gerou Salá. 
\textit{\tiny 13}
 Depois do nascimento de Salá, Arfaxade viveu mais 403 anos e teve outros filhos e filhas.



\bigskip
\textit{\tiny 14}
 Aos 30 anos, Salá gerou Héber. 
\textit{\tiny 15}
 Depois do nascimento de Héber, Salá viveu mais 403 anos e teve outros filhos e filhas.



\bigskip
\textit{\tiny 16}
 Aos 34 anos, Héber gerou Pelegue. 
\textit{\tiny 17}
 Depois do nascimento de Pelegue, Héber viveu mais 430 anos e teve outros filhos e filhas.


\bigskip
\textit{\tiny 18}
 Aos 30 anos, Pelegue gerou Reú. 
\textit{\tiny 19}
 Depois do nascimento de Reú, Pelegue viveu mais 209 anos e teve outros filhos e filhas.


\bigskip
\textit{\tiny 20}
 Aos 32 anos, Reú gerou Serugue. 
\textit{\tiny 21}
 Depois do nascimento de Serugue, Reú viveu mais 207 anos e teve outros filhos e filhas.


\bigskip
\textit{\tiny 22}
 Aos 30 anos, Serugue gerou Naor. 
\textit{\tiny 23}
 Depois do nascimento de Naor, Serugue viveu mais 200 anos e teve outros filhos e filhas.


\bigskip
\textit{\tiny 24}
 Aos 29 anos, Naor gerou Terá. 
\textit{\tiny 25}
 Depois do nascimento de Terá, Naor viveu mais 119 anos e teve outros filhos e filhas.


\bigskip
\textit{\tiny 26}
 Depois que completou 70 anos, Terá gerou Abrão, Naor e Harã.



\bigskip
\subsubsection*{Descendentes de Terá}
\textit{\tiny 27}
 Este é o relato da família de Terá,pai de Abrão, Naor e Harã. Harã, que foi o pai de Ló, 
\textit{\tiny 28}
 morreu em Ur dos caldeus, sua terra natal, enquanto seu pai, Terá, ainda vivia. 
\textit{\tiny 29}
 Tanto Abrão como Naor se casaram. A mulher de Abrão se chamava Sarai, e a mulher de Naor, Milca. (Milca e sua irmã, Iscá, eram filhas de Harã, irmão de Naor.) 
\textit{\tiny 30}
 Sarai, porém, não conseguia engravidar e não tinha filhos. 
\textit{\tiny 31}
 Certo dia, Terá tomou seu filho Abrão, sua nora Sarai (mulher de seu filho Abrão) e seu neto Ló (filho de seu filho Harã) e se mudou de Ur dos caldeus. Partiram em direção à terra de Canaã, mas pararam em Harã e se estabeleceram ali.


\bigskip
\textit{\tiny 32}
Terá viveu 205 anos e morreu enquanto ainda estava em Harã.

----------------------------------------------------------------------
