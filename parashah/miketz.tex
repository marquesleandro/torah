\section*{Parashat Miketz 41:1 - 44:17}

   
\subsubsection*{José e o Faraó}

\textbf{\large 41}
 Dois anos inteiros se passaram, e o faraó sonhou que estava em pé na
margem do rio Nilo. 
\textit{\tiny 2} 
Em seu sonho, viu sete vacas gordas e saudáveis saírem do
rio e começarem a pastar no meio dos juncos. 
\textit{\tiny 3} 
Em seguida, viu outras sete vacas
saírem do Nilo. Eram feias e magras e pararam junto das vacas gordas à beira do
rio. 
\textit{\tiny 4} 
Então as vacas feias e magras comeram as sete vacas gordas e saudáveis.
Nessa parte do sonho, o faraó acordou. 

\bigskip   
\textit{\tiny 5}
Depois, voltou a dormir e teve outro sonho. Dessa vez, viu sete espigas de trigo,
cheias e boas, que cresciam em um só talo. 
\textit{\tiny 6} 
Em seguida, apareceram mais sete
espigas, mas elas eram murchas e ressequidas pelo vento do leste. 
\textit{\tiny 7} 
Então as
espigas miúdas engoliram as sete espigas cheias e bem formadas. O faraó acordou
novamente e percebeu que era um sonho.
 
\bigskip   
\textit{\tiny 8}
Na manhã seguinte, perturbado com os sonhos, o faraó chamou todos os
magos e os sábios do Egito. Contou-lhes os sonhos, mas ninguém conseguiu
interpretá-los. 
\textit{\tiny 9}
Por fim, o chefe dos copeiros se pronunciou. “Hoje eu me lembrei do meu
erro”, disse ao faraó. 
\textit{\tiny 10}
“Algum tempo atrás, o senhor se irou com o chefe dos
padeiros e comigo e mandou prender-nos no palácio do capitão da guarda.
\textit{\tiny 11}
Certa noite, o chefe dos padeiros e eu tivemos, cada um, um sonho, e cada
sonho tinha o seu significado. 
\textit{\tiny 12}
Estava conosco na prisão um rapaz hebreu que
era escravo do capitão da guarda. Contamos a ele nossos sonhos, e ele explicou o
que cada um significava. 
\textit{\tiny 13}
E tudo aconteceu exatamente como ele havia previsto.
Fui restaurado ao meu cargo de chefe dos copeiros, e o chefe dos padeiros foi
enforcado em público.”

\bigskip   
\textit{\tiny 14}
Na mesma hora, o faraó mandou chamar José, e ele foi trazido depressa da
prisão. Depois de barbear-se e trocar de roupa, apresentou-se ao faraó. 
\textit{\tiny 15}
Disse o
faraó a José: “Tive um sonho esta noite e ninguém aqui conseguiu me dizer o que
ele significa. Soube, porém, que ao ouvir um sonho você é capaz de interpretá-lo”.
\textit{\tiny 16}
José respondeu: “Essa capacidade não está em minhas mãos, mas Deus pode
revelar o significado ao faraó e acalmá-lo”.

\bigskip   
\textit{\tiny 17}
Então   o faraó contou o sonho a José: “Em meu sonho, eu estava em pé na
margem do rio Nilo 
\textit{\tiny 18}
e vi sete vacas gordas e saudáveis saírem do rio e
começarem a pastar no meio dos juncos. 
\textit{\tiny 19}
Em seguida, vi saírem do rio sete vacas
feias e magras que pareciam doentes. Nunca vi animais tão horríveis em toda a
terra do Egito. 
\textit{\tiny 20}
Essas vacas feias e magras comeram as sete vacas gordas.
\textit{\tiny 21}
Contudo, não parecia que haviam acabado de comer as outras vacas, pois
continuavam tão magras e feias quanto antes. Então, acordei.
\textit{\tiny 22}
“Em meu sonho, também vi sete espigas de trigo, cheias e boas, que cresciam
em um só talo. 
\textit{\tiny 23}
Em seguida, apareceram outras sete espigas, mas elas eram
murchas, miúdas e ressequidas pelo vento do leste. 
\textit{\tiny 24}
As espigas miúdas
engoliram as sete espigas saudáveis. Contei os sonhos aos magos, mas ninguém
foi capaz de dizer o que significam”.

\bigskip   
\textit{\tiny 25}
José respondeu: “Os dois sonhos do faraó significam a mesma coisa. Deus
está dizendo ao faraó de antemão o que ele vai fazer. 
\textit{\tiny 26}
As sete vacas saudáveis e
as sete espigas de trigo cheias representam sete anos de prosperidade. 
\textit{\tiny 27}
As sete
vacas feias e magras e as sete espigas miúdas e ressequidas pelo vento do leste
representam sete anos de fome.
\textit{\tiny 28}
“Acontecerá exatamente como eu descrevi, pois Deus revelou ao faraó de
antemão o que ele vai fazer. 
\textit{\tiny 29}
Os próximos sete anos serão um período de grande
prosperidade em toda a terra do Egito. 
\textit{\tiny 30}
Depois, haverá sete anos de fome tão
grande que toda essa prosperidade será esquecida no Egito, pois a fome destruirá
a terra. 
\textit{\tiny 31}
A escassez de alimento será tão terrível que apagará até a lembrança dos
anos de fartura. 
\textit{\tiny 32}
Quanto ao fato de terem sido dois sonhos parecidos, significa
que esses acontecimentos foram decretados por Deus, e ele os fará ocorrer em
breve.

\bigskip   
\textit{\tiny 33}
“Portanto, o faraó deve encontrar um homem inteligente e sábio e encarregá-
lo de administrar o Egito. 
\textit{\tiny 34}
O faraó também deve nomear supervisores sobre a
terra, para que recolham um quinto de todas as colheitas durante os sete anos de
fartura. 
\textit{\tiny 35}
Encarregue-os de juntar todo o alimento produzido nos anos bons que
virão e levá-lo para os armazéns do faraó. Mande-os estocar e guardar os cereais,
para que haja mantimento nas cidades. 
\textit{\tiny 36}
Desse modo, quando os sete anos de
fome vierem sobre a terra do Egito, haverá comida suficiente. Assim, a fome não
destruirá a terra”.

\bigskip   
\textit{\tiny 37}
O  faraó e seus oficiais gostaram das sugestões de José. 
\textit{\tiny 38}
Por isso, o faraó
perguntou aos oficiais: “Será que encontraremos alguém como este homem? Sem
dúvida, há nele o espírito de Deus!”. 
\textit{\tiny 39}
Então o faraó disse a José: “Uma vez que
Deus lhe revelou o significado dos sonhos, é evidente que não há ninguém tão
inteligente ou sábio quanto você. 
\textit{\tiny 40}
Ficará encarregado de minha corte, e todo o
meu povo obedecerá às suas ordens. Apenas eu, que ocupo o trono, terei uma
posição superior à sua”.
\textit{\tiny 41}
O faraó acrescentou: “Eu o coloco oficialmente no comando de toda a terra
do Egito”. 
\textit{\tiny 42}
Então o faraó tirou do dedo o seu anel com o selo real e o pôs no dedo
de José. Mandou vesti-lo com roupas de linho fino e pôs uma corrente de ouro em
seu pescoço. 
\textit{\tiny 43}
Também o fez andar na carruagem reservada para quem era o
segundo no poder, e, por onde José passava, gritava-se a ordem: “Ajoelhem-se!”.
Assim, o faraó colocou José no comando de todo o Egito 
\textit{\tiny 44}
e lhe disse: “Eu sou o
faraó, mas ninguém levantará a mão ou o pé em toda a terra do Egito sem a sua
permissão”.

\bigskip   
\textit{\tiny 45}
O faraó deu a José um nome egípcio: Zafenate-Paneia.
 Também lhe deu
uma mulher, que se chamava Azenate. Ela era filha de Potífera, sacerdote de
Om. Assim, José recebeu autoridade sobre todo o Egito. 
\textit{\tiny 46}
Tinha 30 anos
quando começou a servir na corte do faraó, o rei do Egito. Depois de sair da
presença do faraó, José foi inspecionar toda a terra do Egito.


\bigskip   
\textit{\tiny 47}
Como previsto, durante sete anos a terra produziu fartas colheitas. 
\textit{\tiny 48}
Ao
longo desse tempo, José juntou todas as colheitas do Egito e armazenou nas
cidades os cereais produzidos nos campos ao redor. 
\textit{\tiny 49}
Armazenou uma
quantidade imensa de cereais, como a areia do mar. Por fim, parou de manter
registros, pois havia demais para medir.

\bigskip   
\textit{\tiny 50}
Durante esse tempo, antes do primeiro ano de fome, José e sua mulher,
Azenate, filha de Potífera, sacerdote de Om, tiveram dois filhos. 
\textit{\tiny 51}
José chamou o
filho mais velho de Manassés,
 pois disse: “Deus me fez esquecer todas as
minhas dificuldades e toda a família de meu pai”. 
\textit{\tiny 52}
José chamou o segundo filho
de Efraim,
 pois disse: “Deus me fez prosperar na terra da minha aflição”.

\bigskip   
\textit{\tiny 53}
Por fim, terminaram os sete anos de colheitas fartas em toda a terra do Egito,
\textit{\tiny 54}
e começaram os sete anos de fome, como José havia previsto. A fome também
afetou as regiões vizinhas, mas havia alimento de sobra em todo o Egito. 
\textit{\tiny 55}
Depois
de algum tempo, porém, a fome também se espalhou pelo Egito. Quando o povo
clamou ao faraó para que lhe desse alimento, ele respondeu a todos os egípcios:
“Dirijam-se a José e sigam as instruções dele”. 
\textit{\tiny 56}
Quando faltou alimento em toda
parte, José mandou abrir os armazéns e vendeu cereais aos egípcios, pois a fome
era terrível em toda a terra do Egito. 
\textit{\tiny 57}
Gente de todos os lugares ia ao Egito
comprar cereais de José, pois a fome era terrível no mundo inteiro.

\bigskip   
\subsubsection*{José e os irmãos (parte 1)}
\textbf{\large 42}
 Quando Jacó soube que no Egito havia cereais, disse a seus filhos: “Por que
vocês estão aí parados, olhando uns para os outros? 
\textit{\tiny 2} 
Ouvi dizer que há cereais no
Egito. Desçam até lá e comprem cereais em quantidade suficiente para nos
mantermos vivos. Do contrário, morreremos”. 
\textit{\tiny 3}
Então os dez irmãos mais velhos de José desceram ao Egito para comprar
cereais. 
\textit{\tiny 4} 
Mas Jacó não deixou Benjamim, o irmão mais novo de José, ir com eles,
pois tinha medo de que algum mal lhe acontecesse. 

\bigskip   
\textit{\tiny 5} 
Os filhos de Jacó
chegaram ao Egito junto com outros para comprar mantimentos, porque também
havia fome em Canaã. 
\textit{\tiny 6}
Uma vez que José era governador do Egito e o encarregado de vender cereais a
todos, foi a ele que seus irmãos se dirigiram. Quando chegaram, curvaram-se
diante dele com o rosto no chão. 

\bigskip   
\textit{\tiny 7} 
José reconheceu os irmãos de imediato, mas
fingiu não saber quem eram e lhes perguntou com aspereza: “De onde vocês
vêm?”.
   “Da terra de Canaã”, responderam eles. “Viemos comprar mantimentos.” 
\textit{\tiny 8}
Embora José tivesse reconhecido seus irmãos, eles não o reconheceram. 

\bigskip   
\textit{\tiny 9} 
José
se lembrou dos sonhos que tivera a respeito deles muitos anos antes e lhes disse:
“Vocês são espiões! Vieram para descobrir os pontos fracos de nossa terra”.
\textit{\tiny 10}
“Não, meu senhor!”, responderam eles. “Seus servos vieram apenas para
comprar mantimentos. 
\textit{\tiny 11}
Somos todos irmãos, membros da mesma família.
Somos homens honestos, meu senhor, e não espiões!”
\textit{\tiny 12}
Mas José insistiu: “São espiões, sim! Vieram para descobrir os pontos fracos
de nossa terra”.
\textit{\tiny 13}
Eles disseram: “Senhor, na verdade, nós, seus servos, éramos doze irmãos,
todos filhos de um homem que vive na terra de Canaã. Nosso irmão mais novo
está em casa com o pai, e um de nossos irmãos já não está conosco.”
\textit{\tiny 14}
José, porém, continuou a insistir: “Como eu disse, vocês são espiões! 
\textit{\tiny 15}
Mas há
uma forma de verificar sua história. Juro pela vida do faraó que vocês só deixarão
o Egito quando seu irmão mais novo vier para cá. 
\textit{\tiny 16}
Um de vocês deve buscá-lo.
Os outros ficarão presos aqui. Então veremos se sua história é verdadeira ou não.
Pela vida do faraó, se não tiverem um irmão mais novo, saberei com certeza que
são espiões”.

\bigskip   
\textit{\tiny 17}
Então José os colocou na prisão por três dias. 
\textit{\tiny 18}
No terceiro dia, José lhes
disse: “Sou um homem temente a Deus. Façam o que direi e viverão. 
\textit{\tiny 19}
Se são
mesmo homens honestos, escolham um de seus irmãos para continuar preso. Os
demais podem voltar para casa com cereais para seus parentes que estão
passando fome. 
\textit{\tiny 20}
Tragam-me, porém, seu irmão mais novo. Com isso, provarão
que estão dizendo a verdade e não morrerão”.

\bigskip   
   Eles concordaram e, 
\textit{\tiny 21}
conversando entre si, disseram: “É evidente que estamos
sendo castigados por aquilo que fizemos a José tanto tempo atrás. Vimos sua
angústia quando ele implorou por sua vida, mas nós o ignoramos. Por isso
estamos nesta situação difícil”.
\textit{\tiny 22}
Rúben disse: “Não lhes falei que não pecassem contra o rapaz? Mas vocês não
quiseram me ouvir. Agora, temos de prestar contas pelo sangue dele!”.
\textit{\tiny 23}
Não sabiam, porém, que José os entendia, pois falava com eles por meio de
um intérprete. 
\textit{\tiny 24}
José se afastou dos irmãos e começou a chorar. Quando se
recompôs, voltou a falar com eles. Escolheu Simeão e mandou amarrá-lo diante
dos demais.

\bigskip   
\textit{\tiny 25}
Em seguida, José ordenou que seus servos enchessem de cereais os sacos que
os irmãos haviam trazido e, em segredo, devolvessem o pagamento, colocando o
dinheiro na boca de cada saco. Também mandou que lhes dessem mantimentos
para a viagem, e assim fizeram. 
\textit{\tiny 26}
Os irmãos colocaram os sacos de cereal sobre
seus jumentos e partiram de volta para casa.
\textit{\tiny 27}
Contudo, quando um deles abriu a bagagem a fim de pegar cereal para seu
jumento, encontrou o dinheiro na boca do saco. 
\textit{\tiny 28}
“Vejam só!”, exclamou para
seus irmãos. “Devolveram meu dinheiro; está aqui no saco!” O coração deles
desfaleceu e, tremendo, disseram uns aos outros: “O que Deus fez conosco?”.

\bigskip   
\textit{\tiny 29}
Quando os irmãos chegaram à casa de Jacó, seu pai, na terra de Canaã,
relataram-lhe tudo que havia acontecido com eles. 
\textit{\tiny 30}
Disseram: “O homem que
governa o país falou conosco asperamente e nos acusou de sermos espiões em sua
terra, 
\textit{\tiny 31}
mas nós lhe garantimos: ‘Somos homens honestos, e não espiões.
\textit{\tiny 32}
Somos doze irmãos, filhos do mesmo pai. Um de nossos irmãos já não está
conosco, e o mais novo está em casa com nosso pai, na terra de Canaã’.
\textit{\tiny 33}
“Então o homem que governa o país disse: ‘Saberei com certeza se vocês são
homens honestos da seguinte forma: deixem um de seus irmãos comigo e voltem
para casa levando cereais para seus parentes que estão passando fome. 
\textit{\tiny 34}
Tragam-me, porém, seu irmão mais novo e saberei que são homens honestos, e não
espiões. Então eu lhes devolverei seu irmão, e vocês poderão negociar livremente
nesta terra’”.
\textit{\tiny 35}
Ao esvaziarem os sacos, viram que dentro de cada um havia uma bolsa com o
dinheiro do pagamento pelos cereais. Os irmãos e o pai ficaram apavorados
quando viram as bolsas de dinheiro. 

\bigskip   
\textit{\tiny 36}
Jacó disse: “Vocês estão tirando meus
filhos de mim! José se foi, Simeão não está aqui, e agora querem levar Benjamim
também. Tudo está contra mim!”.
\textit{\tiny 37}
Então Rúben disse ao pai: “Se eu não trouxer Benjamim de volta, o senhor
pode matar meus dois filhos. Eu me responsabilizo por ele e prometo trazê-lo de
volta”.
\textit{\tiny 38}
Jacó, porém, respondeu: “Meu filho não descerá com vocês. Seu irmão José
morreu, e Benjamim é tudo que me resta. Se alguma coisa acontecesse com ele na
viagem, vocês me mandariam velho e infeliz para a sepultura”.

\bigskip
\subsubsection*{José e os irmãos (parte 2)}
\textbf{\large 43}
 A fome se agravou na terra de Canaã. 
\textit{\tiny 2} 
Quando os cereais que eles haviam
trazido do Egito estavam para acabar, Jacó disse a seus filhos: “Voltem e comprem
um pouco mais de mantimento para nós”. 

\bigskip
\textit{\tiny 3}
Judá, porém, respondeu: “O homem estava falando sério quando nos advertiu:
‘Vocês não me verão novamente se não trouxerem seu irmão’. 
\textit{\tiny 4} 
Se o senhor enviar
Benjamim conosco, desceremos e compraremos mais mantimento, 
\textit{\tiny 5} 
mas, se não
deixar Benjamim ir, nós também não iremos. Lembre-se de que o homem disse:
‘Vocês não me verão novamente se não trouxerem seu irmão’”. 

\bigskip
\textit{\tiny 6}
“Por que vocês foram tão cruéis comigo?”, lamentou-se Jacó.
 “Por que
disseram ao homem que tinham outro irmão?” 
\textit{\tiny 7}
“Ele fez uma porção de perguntas sobre nossa família”, responderam. “Quis
saber: ‘Seu pai ainda está vivo? Vocês têm outro irmão?’. Nós apenas respondemos
às perguntas dele. Como poderíamos imaginar que ele diria: ‘Tragam seu irmão’?” 
\textit{\tiny 8}
Judá disse a seu pai: “Deixe o rapaz ir comigo e partiremos. Do contrário,
todos nós morreremos de fome, e não apenas nós, mas também nossos
pequeninos. 
\textit{\tiny 9} 
Garanto pessoalmente a segurança dele. O senhor pode me
responsabilizar se eu não o trouxer de volta. Carregarei a culpa para sempre. 
\textit{\tiny 10}
Se
não tivéssemos perdido todo esse tempo, poderíamos ter ido e voltado duas
vezes”.

\bigskip
\textit{\tiny 11}
Por  fim, Jacó, seu pai, lhes disse: “Se não há outro jeito, pelo menos façam o
seguinte. Coloquem na bagagem os melhores produtos desta terra: bálsamo, mel,
especiarias e mirra, pistache e amêndoas, e levem de presente para o homem.
\textit{\tiny 12}
Levem também o dobro do dinheiro que foi devolvido, pois alguém deve tê-lo
colocado nos sacos por engano. 
\textit{\tiny 13}
Depois, peguem seu irmão e voltem àquele
homem. 
\textit{\tiny 14}
Que o Deus Todo-poderoso
 lhes conceda misericórdia quando
estiverem diante daquele homem, para que ele liberte Simeão e deixe Benjamim
voltar. Mas, se devo perder meus filhos, que assim seja”.
\textit{\tiny 15}
Então os homens pegaram os presentes de Jacó e o dobro do dinheiro e
partiram com Benjamim. Por fim, chegaram ao Egito e se apresentaram a José.

\bigskip
\textit{\tiny 16}
Quando José viu Benjamim com eles, disse ao administrador de sua casa: “Estes
homens almoçarão comigo ao meio-dia. Leve-os ao palácio, mate um animal e
prepare um grande banquete”. 
\textit{\tiny 17}
O homem fez conforme José ordenou e os levou
ao palácio de José.
\textit{\tiny 18}
Quando os irmãos viram que estavam sendo levados à casa de José, ficaram
apavorados. “É por causa do dinheiro que alguém colocou de volta nos sacos da
outra vez que estivemos aqui”, disseram uns aos outros. “Ele planeja nos acusar de
roubo e, depois, nos prender, nos tornar escravos e tomar nossos jumentos.”

\bigskip
\textit{\tiny 19}
À entrada do palácio, os irmãos se dirigiram ao administrador de José e lhe
disseram: 
\textit{\tiny 20}
“Ouça, senhor. Viemos ao Egito anteriormente para comprar
mantimentos. 
\textit{\tiny 21}
No caminho de volta para casa, paramos para pernoitar e
abrimos os sacos. Descobrimos que o dinheiro de cada um, a quantia exata que
havíamos pago, estava na boca do saco. Trouxemos o dinheiro de volta. Aqui está.
\textit{\tiny 22}
Também trouxemos mais dinheiro para comprar mantimentos. Não fazemos
ideia de quem colocou o dinheiro nos sacos”.

\bigskip
\textit{\tiny 23}
“Fiquem tranquilos”, disse o administrador. “Não tenham medo. Seu Deus, o
Deus de seu pai, deve ter colocado esse tesouro nos sacos. Tenho certeza de que
recebi seu pagamento.” Depois disso, soltou Simeão e o levou até onde eles
estavam.
\textit{\tiny 24}
Em seguida, o administrador os conduziu para dentro do palácio de José.
Deu-lhes água para lavar os pés e providenciou ração para seus jumentos.
\textit{\tiny 25}
Quando foram avisados que almoçariam lá, os irmãos prepararam os presentes
para a chegada de José ao meio-dia.
\textit{\tiny 26}
Assim que José chegou em casa, entregaram-lhe os presentes que haviam
trazido e curvaram-se até o chão diante dele. 

\bigskip
\textit{\tiny 27}
Depois de cumprimentá-los, José
quis saber: “Como está seu pai, o senhor idoso do qual me falaram? Ainda está
vivo?”.
\textit{\tiny 28}
“Sim”, responderam eles. “Nosso pai, seu servo, ainda está vivo e vai bem.” E
curvaram-se mais uma vez.
\textit{\tiny 29}
Então José olhou para seu irmão Benjamim, o filho de sua mãe, e perguntou:
“Este é o irmão mais novo de que vocês me falaram?”. E disse a Benjamim: “Deus
seja bondoso com você, meu filho”. 
\textit{\tiny 30}
Muito emocionado por causa do irmão, José
saiu depressa da sala. Foi para o quarto, onde chorou. 

\bigskip
\textit{\tiny 31}
Depois de lavar o rosto,
voltou mais controlado e ordenou: “Tragam a comida!”.
\textit{\tiny 32}
José foi servido em sua própria mesa, e seus irmãos, em uma mesa separada.
Os egípcios que comiam com José, por sua vez, foram servidos em outra mesa,
pois os egípcios desprezavam os hebreus e se recusavam a comer com eles. 

\bigskip
\textit{\tiny 33}
José
disse a cada um dos irmãos onde deviam sentar-se e, para espanto deles, colocou-
os ao redor da mesa em ordem de idade, do mais velho para o mais novo.
\textit{\tiny 34}
Mandou encher os pratos deles com comida de sua própria mesa, e deram a
Benjamim uma porção cinco vezes maior que a dos outros. E eles comeram e
beberam à vontade com José.

\bigskip   
\textbf{\large 44}
 Então José deu a seguinte ordem ao administrador do palácio: “Coloque
nos sacos que eles trouxeram todo o cereal que puderem carregar, e coloque o
dinheiro de cada um de volta no saco. 
\textit{\tiny 2} 
Depois, coloque meu copo de prata na
boca do saco de mantimento do mais novo, junto com o dinheiro dele”. O
administrador fez tudo conforme José ordenou. 

\bigskip   
\textit{\tiny 3}
Assim que amanheceu, os irmãos se levantaram e partiram com os jumentos
carregados. 
\textit{\tiny 4} 
Quando haviam percorrido apenas uma distância curta e mal haviam
saído da cidade, José disse ao administrador do palácio: “Vá atrás deles e detenha-
os. Quando os alcançar, diga-lhes: ‘Por que retribuíram o bem com o mal? 
\textit{\tiny 5} 
Por
que roubaram o copo de prata
 do meu senhor, que ele usa para prever o futuro?
Vocês agiram muito mal!’”. 

\bigskip   
\textit{\tiny 6}
Quando o administrador do palácio alcançou os homens, repetiu para eles as
palavras de José. 
\textit{\tiny 7}
“Do   que o senhor está falando?”, disseram os irmãos. “Somos seus servos e
jamais faríamos uma coisa dessas! 
\textit{\tiny 8}
Por acaso não devolvemos o dinheiro que
encontramos nos sacos? Nós o trouxemos de volta da terra de Canaã. Por que
roubaríamos ouro ou prata da casa do seu senhor? 
\textit{\tiny 9} 
Se encontrar o copo de prata
com um de nós, que morra quem estiver com ele! E nós, os restantes, seremos seus
escravos.”
\textit{\tiny 10}
“Sua proposta é justa”, respondeu ele. “Mas apenas aquele que roubou o copo
de prata se tornará meu escravo. Os outros estarão livres.”

\bigskip   
\textit{\tiny 11}
Sem demora, eles descarregaram os sacos e os abriram. 
\textit{\tiny 12}
O administrador
do palácio examinou a bagagem de cada um, começando pelo mais velho até o
mais novo. E o copo foi encontrado no saco de mantimento de Benjamim.
\textit{\tiny 13}
Quando os irmãos viram isso, rasgaram as roupas. Depois, colocaram a carga de
volta sobre os jumentos e retornaram à cidade.

\bigskip   
\subsubsection*{José e a reconciliação}
\textit{\tiny 14}
José ainda estava em seu palácio quando Judá e seus irmãos chegaram, e eles
se curvaram até o chão diante dele. 
\textit{\tiny 15}
“O que vocês fizeram?”, exigiu ele. “Não
sabem que um homem como eu é capaz de prever o que vai acontecer?”
\textit{\tiny 16}
Judá respondeu: “Meu senhor, o que podemos dizer? Que explicação
podemos dar? Como podemos provar nossa inocência? Deus está nos castigando
por causa de nossa maldade. Todos nós voltamos para ser seus escravos, todos
nós, e não apenas nosso irmão com quem foi encontrado o copo de prata”.
\textit{\tiny 17}
José, no entanto, disse: “Eu jamais faria uma coisa dessas! Apenas o homem
que roubou o copo será meu escravo. Os outros podem voltar em paz para a casa
de seu pai”.



----------------------------------------------------------------------
