\section*{Parashat Tzav 6:1–8:36}
\subsubsection*{A oferta queimada e o alimento}  
\textit{\tiny 8} 
Então o SENHOR disse a Moisés: 
\textit{\tiny 9} 
“Dê a Arão e a seus filhos as seguintes
instruções para os holocaustos. 

\smallskip
Os holocaustos serão deixados sobre o altar até a
manhã seguinte, e o fogo sobre o altar será mantido aceso a noite toda. 
\textit{\tiny 10}
Pela
manhã, depois que o sacerdote de serviço tiver vestido suas roupas oficiais de
linho e as roupas de baixo, também de linho, limpará as cinzas do holocausto e as
colocará ao lado do altar. 
\textit{\tiny 11}
Em seguida, removerá as roupas de linho, vestirá suas
roupas habituais e levará as cinzas para fora do acampamento, até um lugar
cerimonialmente puro. 

\smallskip
\textit{\tiny 12}
Enquanto isso, o fogo do altar será mantido aceso;
nunca deverá se apagar. A cada manhã, o sacerdote acrescentará mais lenha ao
fogo, arrumará sobre ele o holocausto e queimará nele a gordura das ofertas de
paz. 
\textit{\tiny 13}
Lembrem-se de que o fogo deverá ser mantido aceso no altar o tempo todo;
nunca deverá se apagar.”

\bigskip
\subsubsection*{A oferta de cereal e o alimento}  
\textit{\tiny 14}
“Estas são as instruções para as ofertas de cereal. Os filhos de Arão
apresentarão esta oferta ao SENHOR diante do altar. 
\textit{\tiny 15}
O sacerdote de serviço
pegará da oferta de cereal um punhado de farinha da melhor qualidade,
umedecida com azeite, junto com todo o incenso, e queimará essa porção
memorial no altar como aroma agradável ao SENHOR. 

\smallskip
\textit{\tiny 16}
Arão e seus filhos poderão
comer o restante da farinha, mas deverão assá-la sem fermento e comê-la num
lugar sagrado dentro do pátio da tenda do encontro. 
\textit{\tiny 17}
Lembrem-se de que essa
oferta nunca deverá ser preparada com fermento. Eu a dei aos sacerdotes como
porção das ofertas especiais apresentadas a mim. É santíssima, como a oferta pelo
pecado e a oferta pela culpa. 
\textit{\tiny 18}
Todos os homens descendentes de Arão poderão
comer das ofertas especiais apresentadas ao SENHOR. É seu direito permanente, de
geração em geração. Qualquer pessoa ou objeto que tocar nessas ofertas se
tornará santo”.

\bigskip
\textit{\tiny 19}
Então o SENHOR disse a Moisés: 
\textit{\tiny 20}
“No dia em que Arão e seus filhos forem
ungidos, apresentarão ao SENHOR a oferta padrão de cereal de dois litros de
farinha da melhor qualidade; metade pela manhã e metade à tarde. 
\textit{\tiny 21}
Misture-a
cuidadosamente com azeite e cozinhe-a numa assadeira. Corte em fatias a oferta
de cereal e apresente-a como aroma agradável ao SENHOR. 
\textit{\tiny 22}
A cada geração, o
sacerdote ungido que suceder a Arão preparará essa mesma oferta. Ela pertence
ao SENHOR e será totalmente queimada. Essa é uma lei permanente. 

\smallskip
\textit{\tiny 23}
Todas as
ofertas de cereal do sacerdote serão totalmente queimadas. Nenhuma parte
poderá ser consumida como alimento”.

\bigskip
\subsubsection*{A oferta do pecado e o alimento}  
\textit{\tiny 24}
O SENHOR também disse a Moisés: 
\textit{\tiny 25}
“Dê a Arão e a seus filhos as seguintes
instruções para a oferta pelo pecado. O animal apresentado como oferta pelo
pecado é oferta santíssima e será morto diante do SENHOR, onde são mortos os
animais para os holocaustos. 

\smallskip
\textit{\tiny 26}
O sacerdote que apresentar o sacrifício como
oferta pelo pecado comerá sua porção num lugar sagrado dentro do pátio da
tenda do encontro. 
\textit{\tiny 27}
Qualquer pessoa ou objeto que tocar a carne do sacrifício se
tornará santo. Se o sangue do sacrifício respingar na roupa de alguém, a peça
manchada será lavada num lugar sagrado. 
\textit{\tiny 28}
Se for usada uma panela de barro
para cozinhar a carne do sacrifício, terá de ser quebrada em seguida. Se for usada
uma panela de bronze, terá de ser esfregada e bem enxaguada com água.
\textit{\tiny 29}
Qualquer homem da família dos sacerdotes poderá comer dessa oferta. É oferta
santíssima. 

\smallskip
\textit{\tiny 30}
Mas a oferta pelo pecado não poderá ser comida se o sangue for
levado à tenda do encontro como oferta para fazer expiação no lugar santo. Nesse
caso, será totalmente queimada no fogo.”
   
\bigskip
\subsubsection*{A oferta de culpa e o alimento}  
\textbf{\large 7} “Estas são as instruções para a oferta pela culpa. É oferta santíssima. 
\textit{\tiny 2} 
O animal sacrificado como oferta pela culpa será morto onde são mortos os animais
para os holocaustos, e seu sangue será derramado em todos os lados do altar. 

\smallskip
\textit{\tiny 3} 
Em seguida, o sacerdote oferecerá toda a gordura sobre o altar, incluindo a gordura
da parte gorda da cauda, a gordura que envolve os órgãos internos, 
\textit{\tiny 4} 
os dois rins, a gordura ao redor deles perto dos lombos e o lóbulo do fígado. Ele removerá todas
essas partes junto com os rins 
\textit{\tiny 5} 
e queimará tudo no altar como oferta especial
apresentada ao SENHOR. É a oferta pela culpa. 

\smallskip
\textit{\tiny 6} 
Qualquer homem da família dos
sacerdotes poderá comer a carne. Deverá comê-la num lugar sagrado, pois é
santíssima. 
\textit{\tiny 7} 
“As mesmas instruções se aplicam tanto à oferta pela culpa como à oferta pelo
pecado. Ambas pertencem ao sacerdote que as utiliza para fazer expiação. 
\textit{\tiny 8} 
No caso dos holocaustos, o sacerdote poderá ficar com o couro do animal
sacrificado. 

\smallskip
\textit{\tiny 9} 
Toda oferta de cereal assada no forno, preparada numa panela ou
cozida numa assadeira, pertence ao sacerdote que a apresenta. 
\textit{\tiny 10}
Todas as outras
ofertas de cereal, preparadas com farinha seca ou farinha umedecida com azeite,
deverão ser divididas em partes iguais entre todos os sacerdotes, os descendentes
de Arão.”

\bigskip
\subsubsection*{A oferta de paz e o alimento}  
\textit{\tiny 11}
“Estas são as instruções sobre os diferentes tipos de oferta de paz que podem
ser apresentados ao SENHOR. 

\smallskip
\textit{\tiny 12}
Se alguém apresentar sua oferta de paz para
expressar gratidão, o animal que normalmente é oferecido será acompanhado de
bolos sem fermento misturados com azeite, pães finos sem fermento untados
com azeite e bolos feitos de farinha da melhor qualidade misturada com azeite.
\textit{\tiny 13}
Essa oferta de paz para expressar gratidão também será acompanhada de pães
preparados com fermento. 
\textit{\tiny 14}
Um pão de cada tipo será apresentado como oferta
para o SENHOR. Os pães serão do sacerdote que derramar o sangue da oferta de
paz no altar. 
\textit{\tiny 15}
A carne da oferta de paz para expressar gratidão será comida no
mesmo dia em que for oferecida. Nada poderá ser guardado até a manhã
seguinte.
   
\smallskip
\textit{\tiny 16}
“Se alguém apresentar uma oferta como cumprimento de um voto ou como
oferta voluntária, a carne será comida no mesmo dia em que o sacrifício for
oferecido, mas o que restar poderá ser comido no dia seguinte. 
\textit{\tiny 17}
A carne que
restar até o terceiro dia deverá ser totalmente queimada. 
\textit{\tiny 18}
Se alguma porção da
carne da oferta de paz for comida no terceiro dia, a pessoa que a trouxe não será
aceita pelo SENHOR e a oferta não terá valor. A essa altura, a carne estará
contaminada, e quem a comer será castigado por causa de seu pecado.
   
\smallskip
\textit{\tiny 19}
“A carne que tocar qualquer coisa cerimonialmente impura não poderá ser
comida; deverá ser totalmente queimada. Mas a carne do sacrifício poderá ser
comida por quem estiver cerimonialmente puro. 
\textit{\tiny 20}
Se alguém estiver
cerimonialmente impuro e comer a carne da oferta de paz apresentada ao SENHOR,
será eliminado do meio do povo. 

\smallskip
\textit{\tiny 21}
Se tocar em algo impuro, seja contaminação
humana, de um animal impuro ou de qualquer outra coisa impura e detestável, e
depois comer a carne de uma oferta de paz apresentada ao SENHOR, será
eliminado do meio do povo”.

\bigskip
\subsubsection*{A gordura e o sangue}  
\textit{\tiny 22}
Então o SENHOR disse a Moisés: 
\textit{\tiny 23}
“Dê as seguintes instruções ao povo de Israel.
Jamais comam gordura, seja de boi, carneiro ou cabrito. 
\textit{\tiny 24}
A gordura de um
animal encontrado morto ou despedaçado por animais selvagens jamais deverá
ser comida, embora possa ser usada para outros fins. 
\textit{\tiny 25}
Quem comer a gordura de
um animal apresentado como oferta especial para o SENHOR será eliminado do
meio do povo. 

\smallskip
\textit{\tiny 26}
Onde quer que morarem, jamais consumam o sangue de
qualquer ave ou animal. 
\textit{\tiny 27}
Quem consumir sangue será eliminado do meio do
povo”.

\bigskip
\subsubsection*{A oferta de paz e a oferta especial}  
\textit{\tiny 28}
O SENHOR disse a Moisés: 
\textit{\tiny 29}
“Dê as seguintes instruções ao povo de Israel.
Quando apresentarem uma oferta de paz ao SENHOR, levem uma parte dela como
oferta para o SENHOR. 
\textit{\tiny 30}
Apresentem-na com suas próprias mãos como oferta
especial para o SENHOR. Levem a gordura do animal junto com o peito e movam o
peito para o alto como oferta especial para o SENHOR. 
\textit{\tiny 31}
Em seguida, o sacerdote
queimará a gordura no altar, mas o peito será de Arão e seus descendentes.
\textit{\tiny 32}
Entreguem como oferta ao sacerdote a coxa direita da oferta de paz. 
\textit{\tiny 33}
A coxa
direita será sempre a porção entregue ao sacerdote que apresentar o sangue e a
gordura da oferta de paz. 
\textit{\tiny 34}
Pois reservei para os sacerdotes o peito da oferta
especial e a coxa direita da oferta sagrada. Arão e seus descendentes têm o direito
permanente de participar das ofertas de paz que os israelitas apresentarem.
\textit{\tiny 35}
Essa é sua porção por direito das ofertas especiais apresentadas ao SENHOR,
reservada para Arão e seus descendentes desde o dia em que eles foram
separados para servir ao SENHOR como sacerdotes. 
\textit{\tiny 36}
No dia em que foram
ungidos, o SENHOR ordenou que os israelitas entregassem essas partes aos
sacerdotes como sua porção permanente, de geração em geração”.
   
\bigskip
\textit{\tiny 37}
Essas são as instruções para o holocausto, a oferta de cereal, a oferta pelo
pecado e a oferta pela culpa e também para a oferta de consagração e a oferta de
paz. 
\textit{\tiny 38}
O SENHOR deu essas instruções a Moisés no monte Sinai, quando ordenou
que os israelitas apresentassem suas ofertas ao SENHOR no deserto do Sinai.
A consagração dos sacerdotes
   
\bigskip
\subsubsection*{Os sacerdotes e a consagração - parte 1}  
\textbf{\large 8} Então o SENHOR disse a Moisés: 
\textit{\tiny 2} 
“Traga Arão e seus filhos, as roupas sagradas,
o óleo da unção, o novilho para a oferta pelo pecado, os dois carneiros e o cesto de
pães sem fermento, 
\textit{\tiny 3} 
e reúna toda a comunidade à entrada da tenda do encontro”. 
\textit{\tiny 4} 
Moisés   seguiu as instruções do SENHOR, e toda a comunidade se reuniu à
entrada da tenda do encontro. 
\textit{\tiny 5} 
“É isto que o SENHOR ordenou que façamos!”,
anunciou Moisés. 

\smallskip
\textit{\tiny 6} 
Em seguida, apresentou Arão e seus filhos e os lavou com água. 
\textit{\tiny 7} 
Colocou a túnica oficial em Arão e amarrou o cinturão ao redor de sua cintura.
Vestiu-o com o manto, sobre o qual colocou o colete sacerdotal, que prendeu
firmemente com o cinturão decorativo. 
\textit{\tiny 8} 
Colocou em Arão o peitoral e, dentro
dele, o Urim e o Tumim. 
\textit{\tiny 9} 
Pôs na cabeça de Arão o turbante e, na parte da frente
do turbante, prendeu a tiara sagrada, o emblema de santidade, conforme o
SENHOR havia ordenado.   

\smallskip
\textit{\tiny 10}
Depois, Moisés pegou o óleo da unção e ungiu o tabernáculo e tudo que nele
havia, a fim de consagrá-los. 
\textit{\tiny 11}
Aspergiu o altar com óleo sete vezes para ungi-lo,
bem como todos os seus utensílios, a bacia e seu suporte, para também consagrá-
los. 
\textit{\tiny 12}
Derramou um pouco do óleo sobre a cabeça de Arão, para ungi-lo e
consagrá-lo. 
\textit{\tiny 13}
Em seguida, Moisés apresentou os filhos de Arão. Vestiu-os com as
túnicas, amarrou neles o cinturão e pôs-lhes na cabeça o turbante especial,
conforme o SENHOR tinha ordenado.
   
\smallskip
\textit{\tiny 14}
Então Moisés apresentou o novilho para a oferta pelo pecado. Arão e seus
filhos colocaram as mãos sobre a cabeça do novilho, 
\textit{\tiny 15}
e Moisés o matou. Pegou
um pouco do sangue e, com o dedo, colocou-o nas quatro pontas do altar, a fim de
purificá-lo. O restante do sangue ele derramou na base do altar. Desse modo,
consagrou o altar e fez expiação por ele.
\textit{\tiny 16}
Depois, pegou toda a gordura que
envolvia os órgãos internos, o lóbulo do fígado, os dois rins e a gordura ao redor
deles e os queimou no altar. 
\textit{\tiny 17}
Pegou o restante do novilho, incluindo o couro, a
carne e os excrementos, e o queimou num fogo fora do acampamento, conforme
o SENHOR tinha ordenado.
   
\smallskip
\textit{\tiny 18}
Então Moisés apresentou o carneiro para o holocausto. Arão e seus filhos
colocaram as mãos sobre a cabeça do animal, 
\textit{\tiny 19}
e Moisés o matou. Pegou o sangue
do carneiro e o derramou em todos os lados do altar. 
\textit{\tiny 20}
Cortou o carneiro em
pedaços e o queimou no altar, junto com a cabeça e a gordura. 
\textit{\tiny 21}
Depois de lavar
os órgãos internos e as pernas com água, queimou todo o carneiro sobre o altar
como holocausto. Foi um aroma agradável, uma oferta especial apresentada ao
SENHOR, conforme o SENHOR tinha ordenado.
   
\smallskip
\textit{\tiny 22}
Então Moisés apresentou o outro carneiro, o carneiro da consagração. Arão e
seus filhos colocaram as mãos sobre a cabeça do animal, 
\textit{\tiny 23}
e Moisés o matou.
Pegou um pouco do sangue e o colocou na ponta da orelha direita, no polegar da
mão direita e no polegar do pé direito de Arão. 
\textit{\tiny 24}
Depois, apresentou os filhos de
Arão e colocou um pouco do sangue na ponta da orelha direita, no polegar da mão
direita e no polegar do pé direito deles. O restante do sangue ele derramou em
todos os lados do altar.
\textit{\tiny 25}
Em seguida, pegou a gordura, incluindo a gordura da parte gorda da cauda, a
gordura que envolve os órgãos internos, o lóbulo do fígado, os dois rins e a
gordura ao redor deles, bem como a coxa direita. 
\textit{\tiny 26}
Sobre essas partes colocou um
pão sem fermento, um pão de massa misturada com azeite e um pão fino untado
com azeite, que tirou do cesto de pães sem fermento que estava na presença do
SENHOR. 
\textit{\tiny 27}
Colocou tudo nas mãos de Arão e seus filhos e moveu os alimentos para
o alto como oferta especial para o SENHOR. 
\textit{\tiny 28}
Pegou as ofertas de volta das mãos
deles e as queimou no altar, sobre o holocausto. Essa foi a oferta de consagração.
Foi um aroma agradável, uma oferta especial apresentada ao SENHOR. 

\smallskip
\textit{\tiny 29}
Moisés
pegou o peito e o moveu para o alto como oferta especial para o SENHOR. Era a
porção de Moisés do carneiro da consagração, conforme o SENHOR tinha
ordenado.
   
\smallskip
\textit{\tiny 30}
Então Moisés pegou um pouco do óleo da unção e um pouco do sangue que
estava sobre o altar e aspergiu sobre Arão e suas roupas e sobre seus filhos e suas
roupas. Desse modo, consagrou Arão, seus filhos e suas roupas.
\textit{\tiny 31}
Por fim, Moisés disse a Arão e a seus filhos: “Cozinhem o restante da carne
das ofertas à entrada da tenda do encontro e comam-na ali, junto com os pães que
estão no cesto de ofertas para a consagração, conforme ordenei quando disse:
‘Arão e seus filhos os comerão’. 
\textit{\tiny 32}
Queimem qualquer carne ou pão que sobrar.
\textit{\tiny 33}
Não saiam da entrada da tenda do encontro por sete dias, pois só então estará
concluída a cerimônia de consagração. 
\textit{\tiny 34}
Tudo que fizemos hoje foi ordenado
pelo SENHOR a fim de fazer expiação por vocês. 
\textit{\tiny 35}
Agora, permaneçam à entrada
da tenda do encontro dia e noite por sete dias e cumpram todas as exigências do
SENHOR. Se não o fizerem, morrerão, pois foi isso que o SENHOR me ordenou”.

\smallskip
\textit{\tiny 36}
Arão e seus filhos fizeram tudo que o SENHOR tinha ordenado por meio de
Moisés.



----------------------------------------------------------------------
