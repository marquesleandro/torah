\section*{Parashat Lekh Lekha 12:1 - 17:27}

\subsubsection*{Abrão e a promessa}
\textbf{\large 12}
 O SENHOR tinha dito a Abrão: “Deixe sua terra natal, seus parentes e a
família de seu pai e vá à terra que eu lhe mostrarei. 
\textit{\tiny 2}
Farei de você uma grande
nação, o abençoarei e o tornarei famoso, e você será uma bênção para outros.
\textit{\tiny 3}
Abençoarei os que o abençoarem e amaldiçoarei os que o amaldiçoarem. Por
meio de você, todas as famílias da terra serão abençoadas”.

\bigskip
\textit{\tiny 4}
Então Abrão partiu, como o SENHOR havia instruído, e Ló foi com ele. Abrão
tinha 75 anos quando saiu de Harã. 
\textit{\tiny 5}
Tomou sua mulher, Sarai, seu sobrinho Ló e
todos os seus bens, os rebanhos e os servos que havia agregado à sua casa em
Harã, e seguiu para a terra de Canaã. 

\bigskip
Quando chegaram a Canaã, 
\textit{\tiny 6}
Abrão
atravessou a terra até Siquém, onde acampou junto ao carvalho de Moré. Naquele
tempo, os cananeus habitavam a região.
\textit{\tiny 7}
Então o SENHOR apareceu a Abrão e disse: “Darei esta terra a seus
descendentes”. Abrão construiu um altar ali e o dedicou ao SENHOR, que lhe havia
aparecido. 
\textit{\tiny 8}
Dali, Abrão viajou para o sul e acampou na região montanhosa, entre
Betel, a oeste, e Ai, a leste. Construiu ali mais um altar dedicado ao SENHOR e
invocou o nome do SENHOR. 
\textit{\tiny 9}
Abrão prosseguiu em sua jornada para o sul,
acampando ao longo do caminho em direção ao Neguebe.

\bigskip
\subsubsection*{Abrão e o Egito}
\textit{\tiny 10}
Naquele tempo, uma fome terrível atingiu a terra de Canaã, e Abrão foi
obrigado a descer ao Egito, onde viveu como estrangeiro. 
\textit{\tiny 11}
Aproximando-se da
fronteira do Egito, Abrão disse a Sarai, sua mulher: “Você é muito bonita.
\textit{\tiny 12}
Quando os egípcios a virem, dirão: ‘É mulher dele. Vamos matá-lo para ficarmos
com ela’. 
\textit{\tiny 13}
Diga, portanto, que é minha irmã. Eles pouparão minha vida e, por sua
causa, me tratarão bem”.

\bigskip
\textit{\tiny 14}
De fato, chegando Abrão ao Egito, todos notaram a grande beleza de sua
mulher. 
\textit{\tiny 15}
Quando os oficiais do palácio a viram, falaram maravilhas dela ao faraó
e a levaram para o palácio. 
\textit{\tiny 16}
Por causa de Sarai, o faraó deu muitos presentes a
Abrão: ovelhas, bois, jumentos e jumentas, servos e servas, e camelos.

\bigskip
\textit{\tiny 17}
Mas, por causa de Sarai, mulher de Abrão, o SENHOR enviou pragas terríveis
sobre o faraó e sobre os membros de sua casa. 
\textit{\tiny 18}
Por isso, o faraó mandou chamar
Abrão e disse: “O que você fez comigo? Por que não me disse que ela era sua
mulher? 
\textit{\tiny 19}
Por que disse que era sua irmã e permitiu que eu a tomasse como
esposa? Aqui está sua mulher. Tome-a e vá embora daqui!”. 
\textit{\tiny 20}
O faraó ordenou
que alguns de seus homens escoltassem Abrão, com sua mulher e todos os seus
bens, para fora de sua terra.

\bigskip
\subsubsection*{Abrão e Ló se separam}
\textbf{\large 13}
 Abrão saiu do Egito e subiu para o Neguebe, junto com sua mulher, com Ló
e com tudo que possuíam. 
\textit{\tiny 2}
(Abrão era muito rico e tinha muitos rebanhos, prata e
ouro.) 
\textit{\tiny 3}
Do Neguebe, prosseguiram em sua jornada, acampando ao longo do
caminho em direção a Betel. Por fim, armaram as tendas entre Betel e Ai, onde
haviam acampado anteriormente, 
\textit{\tiny 4}
e onde Abrão havia construído um altar. Ali,
Abrão invocou o nome do SENHOR outra vez.

\bigskip
\textit{\tiny 5}
Ló, que viajava com Abrão, também havia enriquecido e possuía rebanhos de
ovelhas, gado e muitas tendas. 
\textit{\tiny 6}
Os recursos da terra, porém, não eram suficientes para sustentar Abrão e Ló, com todos os seus rebanhos, vivendo tão próximos um
do outro. 
\textit{\tiny 7}
Logo, surgiram desentendimentos entre os pastores de Abrão e os de
Ló. (Naquele tempo, os cananeus e os ferezeus também viviam na terra.)
\textit{\tiny 8}
Então Abrão disse a Ló: “Não haja conflito entre nós, ou entre nossos pastores.
Afinal, somos parentes próximos! 
\textit{\tiny 9}
A região inteira está à sua disposição. Escolha
a parte da terra que desejar e nos separaremos. Se você escolher as terras à
esquerda, ficarei com as terras à direita. Se preferir as terras à direita, ficarei com
as terras à esquerda”.
\textit{\tiny 10}
Ló olhou demoradamente para as planícies férteis do vale do Jordão, na
direção de Zoar. A região toda era bem irrigada, como o jardim do SENHOR, ou
como a terra do Egito. (Isso foi antes de o SENHOR destruir Sodoma e Gomorra.)
\textit{\tiny 11}
Ló escolheu para si todo o vale do Jordão a leste de onde estavam. Partiu para lá
e se separou de seu tio Abrão. 
\textit{\tiny 12}
Assim, Abrão continuou na terra de Canaã, e Ló
mudou suas tendas para um lugar próximo de Sodoma e se estabeleceu entre as
cidades da planície. 
\textit{\tiny 13}
O povo dessa região, porém, era extremamente perverso e
vivia pecando contra o SENHOR.

\bigskip
\subsubsection*{Abrão e a primeira repetição da promessa}
\textit{\tiny 14}
Depois que Ló partiu, o SENHOR disse a Abrão: “Olhe até onde sua vista
alcançar, em todas as direções: norte e sul, leste e oeste. 
\textit{\tiny 15}
Toda esta terra que
você está vendo, até onde sua vista alcança, eu dou a você e a seus descendentes
como propriedade para sempre. 
\textit{\tiny 16}
Eu lhe darei tantos descendentes quanto o pó
da terra, de modo que, se fosse possível contar o pó da terra, seria possível contar
seus descendentes! 
\textit{\tiny 17}
Vá e percorra a terra em todas as direções, porque eu a dou
a você”.
\textit{\tiny 18}
Então Abrão mudou seu acampamento para Hebrom e se estabeleceu junto
ao bosque de carvalhos que pertencia a Manre. Ali, construiu mais um altar ao
SENHOR.


\bigskip
\subsubsection*{Ló é capturado}
\textbf{\large 14}
 Por esse tempo, houve guerra na região. Anrafel, rei da Babilônia, Arioque, rei de Elasar, Quedorlaomer, rei de Elão, e Tidal, rei de Goim, 
\textit{\tiny 2}
lutaram contra Bera, rei de Sodoma, Birsa, rei de Gomorra, Sinabe, rei de Admá,
Semeber, rei de Zeboim, e contra o rei de Belá (também chamada Zoar).
\textit{\tiny 3}
Esse segundo grupo de reis reuniu suas tropas no vale de Sidim (ou seja, no
vale do mar Morto). 
\textit{\tiny 4}
Por doze anos, estiveram sob o domínio do rei
Quedorlaomer, mas no décimo terceiro ano se rebelaram contra ele.
\textit{\tiny 5}
Um   ano depois, Quedorlaomer e seus aliados vieram e derrotaram os refains
em Asterote-Carnaim, os zuzins em Hã, os emins em Savé-Quiriatim,
\textit{\tiny 6}
e os horeus
no monte Seir, até El-Parã, à beira do deserto. 
\textit{\tiny 7}
Em seguida, voltaram e foram a En-
Mispate (hoje chamada Cades) e conquistaram o território dos amalequitas e dos
amorreus que viviam em Hazazom-Tamar.
\textit{\tiny 8}
Então os reis de Sodoma, Gomorra, Admá, Zeboim e Belá (também chamada
Zoar) se prepararam para a batalha no vale do mar Morto.
\textit{\tiny 9}
Lutaram contra
Quedorlaomer, rei de Elão, Tidal, rei de Goim, Anrafel, rei da Babilônia, e Arioque,
rei de Elasar, quatro reis contra cinco. 
\textit{\tiny 10}
Acontece que o vale do mar Morto era
cheio de poços de betume. Quando o exército dos reis de Sodoma e Gomorra
fugiu, alguns dos soldados caíram nos poços de betume, enquanto o restante
escapou para as montanhas. 
\textit{\tiny 11}
Os invasores vitoriosos saquearam Sodoma e
Gomorra e partiram para casa, levando consigo todos os espólios da guerra e os
mantimentos. 
\textit{\tiny 12}
Também capturaram Ló, o sobrinho de Abrão que morava em
Sodoma, e tudo que ele possuía.
\textit{\tiny 13}
Um dos homens de Ló, porém, conseguiu escapar e contou tudo a Abrão, o
hebreu, que morava junto ao bosque de carvalhos pertencente a Manre, o
amorreu. Manre e seus parentes, Escol e Aner, eram aliados de Abrão.
\textit{\tiny 14}
Quando Abrão soube que seu sobrinho Ló havia sido capturado, mobilizou os 318
 homens treinados que tinham nascido em sua casa. Perseguiu o exército de
Quedorlaomer até alcançá-los em Dã, 
\textit{\tiny 15}
onde dividiu os homens em grupos e
atacou durante a noite. O exército de Quedorlaomer fugiu, mas Abrão o perseguiu
até Hobá, ao norte de Damasco. 
\textit{\tiny 16}
Abrão recuperou todos os bens saqueados e
trouxe de volta Ló, seu sobrinho, com todos os seus bens, as mulheres e os outros
prisioneiros.

\bigskip
\subsubsection*{Abrão e Melquisedeque}
\textit{\tiny 17}
Depois que Abrão regressou vitorioso do conflito com Quedorlaomer e todos os
seus aliados, o rei de Sodoma saiu ao seu encontro no vale de Savé (conhecido
como vale do Rei).
\textit{\tiny 18}
Melquisedeque, rei de Salém e sacerdote do Deus Altíssimo, trouxe pão e
vinho 
\textit{\tiny 19}
e abençoou Abrão, dizendo:
  “Bendito seja Abrão pelo Deus Altíssimo,
    Criador dos céus e da terra.
\textit{\tiny 20}
E bendito seja o Deus Altíssimo, que derrotou seus inimigos por você”.
Então Abrão entregou a Melquisedeque um décimo de todos os bens que havia
recuperado.

\bigskip
\textit{\tiny 21}
O rei de Sodoma disse a Abrão: “Devolva-me apenas as pessoas que foram
capturadas. Fique com os bens que você recuperou”.
\textit{\tiny 22}
Abrão respondeu ao rei de Sodoma: “Juro solenemente diante do SENHOR, o
Deus Altíssimo, Criador dos céus e da terra, 
\textit{\tiny 23}
que não ficarei com coisa alguma
do que é seu, nem sequer um fio ou uma correia de sandália. Do contrário, o rei
poderia dizer: ‘Fui eu que enriqueci Abrão’. 
\textit{\tiny 24}
Aceito apenas aquilo que meus
jovens guerreiros comeram e peço que dê uma parte justa dos bens a Aner, Escol e
Manre, meus aliados”.

\bigskip
\subsubsection*{Abrão e a segunda repetição da promessa}
\textbf{\large 15}
 Algum tempo depois, o SENHOR falou a Abrão em uma visão e lhe disse:
“Não tenha medo, Abrão, pois eu serei seu escudo, e sua recompensa será muito
grande”.
\textit{\tiny 2}
Abrão, porém, respondeu: “Ó SENHOR Soberano, de que me adiantam todas as
tuas bênçãos se eu nem mesmo tenho um filho? Uma vez que não me deste filhos,
Eliézer de Damasco, servo em minha casa, herdará toda a minha riqueza. 
\textit{\tiny 3}
Não me
deste nenhum descendente próprio e, por isso, um dos meus servos será meu
herdeiro”.

\bigskip
\textit{\tiny 4}
O SENHOR lhe disse: “Não, não será esse o seu herdeiro; você terá seu próprio
filho, e ele será seu herdeiro”. 
\textit{\tiny 5}
Em seguida, levou Abrão para fora e lhe disse:
“Olhe para o céu e conte as estrelas, se for capaz. Este é o número de
descendentes que você terá”.
\textit{\tiny 6}
Abrão creu no SENHOR, e assim foi considerado justo.

\bigskip
\textit{\tiny 7}
Então o SENHOR lhe disse: “Eu sou o SENHOR, que o tirei de Ur dos caldeus para
lhe dar esta terra como posse”.
\textit{\tiny 8}
Abrão perguntou: “Ó SENHOR Soberano, como posso ter certeza de que a
possuirei de fato?”.
\textit{\tiny 9}
O SENHOR respondeu: “Traga-me uma novilha, uma cabra e um carneiro, todos
com três anos, mais uma rolinha e um pombinho”. 
\textit{\tiny 10}
Abrão lhe apresentou todos
esses animais e os matou. Em seguida, cortou cada um deles ao meio e colocou as
metades lado a lado; as aves, porém, não cortou ao meio. 
\textit{\tiny 11}
Aves de rapina
mergulharam para comer as carcaças, mas Abrão as afugentou.
\textit{\tiny 12}
Enquanto  o sol se punha, Abrão caiu em sono profundo, e uma escuridão
apavorante desceu sobre ele. 

\bigskip
\textit{\tiny 13}
Então o SENHOR disse a Abrão: “Esteja certo de que
seus descendentes serão forasteiros em terra alheia, onde sofrerão opressão
como escravos por quatrocentos anos. 
\textit{\tiny 14}
Mas eu castigarei a nação que os
escravizar e, por fim, eles sairão de lá com grande riqueza. 
\textit{\tiny 15}
(Você, por sua vez,
morrerá em paz e será sepultado em idade avançada.) 
\textit{\tiny 16}
Depois de quatro
gerações, seus descendentes voltarão a esta terra, pois a maldade dos amorreus
ainda não chegou ao ponto de provocar meu castigo”.

\bigskip
\textit{\tiny 17}
Quando o sol se pôs e veio a escuridão, Abrão viu um fogareiro fumegante e
uma tocha ardente passarem por entre as metades das carcaças. 
\textit{\tiny 18}
Então o SENHOR
fez uma aliança com Abrão naquele dia e disse: “Dei esta terra a seus
descendentes, desde a fronteira com o Egito até o grande rio Eufrates, 
\textit{\tiny 19}
a terra
hoje ocupada pelos queneus, quenezeus, cadmoneus, 
\textit{\tiny 20}
hititas, ferezeus, refains,
\textit{\tiny 21}
amorreus, cananeus, girgaseus e jebuseus”.

\bigskip
\subsubsection*{Ismael}
\textbf{\large 16}
 Sarai, mulher de Abrão, não havia conseguido lhe dar filhos. Tinha, porém,
uma serva egípcia chamada Hagar. 
\textit{\tiny 2}
Sarai disse a Abrão: “O SENHOR me impediu de
ter filhos. Vá e deite-se com minha serva. Talvez, por meio dela, eu consiga ter
uma família”. Abrão aceitou a proposta de Sarai. 

\bigskip
\textit{\tiny 3}
Então Sarai, mulher de Abrão,
tomou Hagar, a serva egípcia, e a entregou a Abrão como mulher. (Isso aconteceu
dez anos depois que Abrão havia se estabelecido na terra de Canaã.)
\textit{\tiny 4}
Abrão teve relações com Hagar, e ela engravidou. Quando Hagar soube que
estava grávida, começou a tratar Sarai, sua senhora, com desprezo. 
\textit{\tiny 5}
Então Sarai
disse a Abrão: “Você é o culpado da vergonha que estou passando! Entreguei
minha serva a você, mas, agora que engravidou, ela me trata com desprezo. O
SENHOR mostrará quem está errado: você ou eu!”.
\textit{\tiny 6}
Abrão respondeu: “Hagar é sua serva. Faça com ela o que lhe parecer melhor”.
Então Sarai a tratou tão mal que, por fim, Hagar fugiu.

\bigskip
\textit{\tiny 7}
O anjo do SENHOR encontrou Hagar no deserto, perto de uma fonte de água
junto à estrada para Sur, 
\textit{\tiny 8}
e perguntou: “Hagar, serva de Sarai, de onde você vem e
para onde vai?”.
   “Estou fugindo de minha senhora, Sarai”, respondeu ela.
\textit{\tiny 9}
Então o anjo do SENHOR disse: “Volte para sua senhora e sujeite-se à autoridade
dela”. 
\textit{\tiny 10}
E acrescentou: “Eu lhe darei tantos descendentes que será impossível contá-los”.
\textit{\tiny 11}
O anjo do SENHOR também disse: “Você está grávida e dará à luz um filho. Dê
a ele o nome de Ismael, pois o SENHOR ouviu seu clamor angustiado. 
\textit{\tiny 12}
Seu filho
será um homem solitário e indomável, como um jumento selvagem. Levantará o
punho contra todos, e todos serão contra ele. Sim, ele viverá em franca oposição a
todos os seus parentes”.   
\textit{\tiny 13}
Então Hagar passou a usar outro nome para se referir ao SENHOR, que havia
falado com ela. Chamou-o de “Tu és o Deus que me vê”, pois tinha dito: “Aqui eu
vi aquele que me vê!”. 
\textit{\tiny 14}
Por isso, aquela fonte que fica entre Cades e Berede
recebeu o nome de Beer-Laai-Roi.   

\bigskip
\textit{\tiny 15}
Assim, Hagar deu um filho a Abrão, e Abrão o chamou de Ismael. 
\textit{\tiny 16}
Quando
Ismael nasceu, Abrão tinha 86 anos.

\bigskip
\subsubsection*{Abraão e a aliança (terceira repetição da promessa)}
\textbf{\large 17}
 Quando Abrão estava com 99 anos, o SENHOR lhe apareceu e disse: “Eu sou
o Deus Todo-poderoso. Seja fiel a mim e tenha uma vida íntegra. 
\textit{\tiny 2}
Farei uma
aliança com você e lhe darei uma descendência incontável”.
\textit{\tiny 3}
Ao ouvir essas palavras, Abrão se prostrou com o rosto no chão, e Deus lhe
disse: 
\textit{\tiny 4}
“Esta é a minha aliança com você: farei de você o pai de numerosas nações!
\textit{\tiny 5}
Além disso, mudarei seu nome. Você já não será chamado Abrão, mas sim
Abraão, pois será o pai de muitas nações. 
\textit{\tiny 6}
Eu o tornarei extremamente fértil.
Seus descendentes formarão muitas nações, e haverá reis entre eles.
\textit{\tiny 7}
“Confirmarei a minha aliança com você e seus descendentes, de geração em
geração. Esta é a aliança sem fim: serei sempre o seu Deus e o Deus de seus
descendentes. 
\textit{\tiny 8}
Darei a você e a seus descendentes toda a terra de Canaã, onde
hoje você vive como estrangeiro. Será propriedade deles para sempre, e eu serei o
seu Deus”.

\bigskip
\textit{\tiny 9}
Então Deus disse a Abraão: “É sua responsabilidade permanente, e de seus
descendentes, obedecer aos termos da aliança. 
\textit{\tiny 10}
Este é o sinal da aliança que você
e seus descendentes devem guardar: todo indivíduo do sexo masculino entre
vocês deve ser circuncidado. 
\textit{\tiny 11}
Cortem a carne do prepúcio como sinal da aliança
entre mim e vocês. 
\textit{\tiny 12}
Todo menino deve ser circuncidado no oitavo dia depois do
nascimento, de geração em geração. Isso se aplica não apenas aos membros de sua família, mas também aos servos nascidos em sua casa e aos servos
estrangeiros que você comprou. 
\textit{\tiny 13}
Quer sejam nascidos em sua casa, quer os
tenha comprado, todos devem ser circuncidados. Terão no corpo o sinal da minha
aliança sem fim. 
\textit{\tiny 14}
O indivíduo do sexo masculino que não for circuncidado será
excluído do seu povo, pois quebrou a minha aliança”.

\bigskip
\textit{\tiny 15}
Deus também disse a Abraão: “Quanto à sua mulher, não se chamará mais Sarai.
De agora em diante ela se chamará Sara. 
\textit{\tiny 16}
Eu a abençoarei e por meio dela darei
a você um filho! Sim, eu a abençoarei, e ela se tornará mãe de muitas nações.
Haverá reis de nações entre seus descendentes”.
\textit{\tiny 17}
Abraão se prostrou com o rosto no chão e riu consigo. Pensou: “Como eu, aos
100 anos, poderia ser pai? E como Sara, aos 90 anos, teria um filho?”. 
\textit{\tiny 18}
Então
Abraão disse a Deus: “Que Ismael viva sob a tua bênção!”.
\textit{\tiny 19}
Mas Deus respondeu: “Na verdade, Sara, sua mulher, lhe dará um filho. Você
o chamará Isaque, e eu confirmarei com ele e com seus descendentes, para
sempre, a minha aliança. 
\textit{\tiny 20}
Quanto a Ismael, também o abençoarei, como você
pediu. Eu o tornarei extremamente fértil e multiplicarei seus descendentes. Ele
será pai de doze príncipes, e farei dele uma grande nação. 
\textit{\tiny 21}
Minha aliança,
porém, será confirmada com Isaque, filho que Sara lhe dará por esta época, no
ano que vem”.

\bigskip 
\textit{\tiny 22}
Quando Deus terminou de falar, retirou-se da presença de
Abraão.
\textit{\tiny 23}
Naquele mesmo dia, Abraão tomou Ismael, seu filho, e todos os indivíduos do
sexo masculino em sua casa, tanto os nascidos ali como os comprados, e os
circuncidou, removendo o prepúcio, como Deus havia ordenado. 
\textit{\tiny 24}
Abraão tinha
99 anos quando foi circuncidado, 
\textit{\tiny 25}
e Ismael, seu filho, tinha 13 anos. 
\textit{\tiny 26}
Ambos
foram circuncidados naquele mesmo dia, 
\textit{\tiny 27}
junto com todos os outros homens e
meninos da casa, tanto os nascidos ali como os comprados. Todos foram
circuncidados com Abraão.

----------------------------------------------------------------------
