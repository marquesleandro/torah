\section*{Parashat Vayera 18:1 - 22:24}

\subsubsection*{Sara riu (quarta repetição da promessa)}

\textbf{\large 18}
 O SENHOR apareceu novamente a Abraão junto ao bosque de carvalhos que
pertencia a Manre. Abraão estava sentado à entrada de sua tenda na hora mais
quente do dia. 
\textit{\tiny 2}
Olhando para fora, viu três homens em pé, próximos à tenda.
Quando os viu, correu até onde estavam e lhes deu as boas-vindas, curvando-se
até o chão.

\bigskip   
\textit{\tiny 3}
Abraão disse: “Meu senhor, se assim desejar, pare aqui um pouco. 
\textit{\tiny 4}
Descanse à
sombra desta árvore enquanto mando trazer água para lavarem os pés. 
\textit{\tiny 5}
E, uma
vez que honraram seu servo com esta visita, prepararei uma refeição para
restaurar suas forças antes de seguirem viagem”.
   “Está bem”, responderam eles. “Faça como você disse.”
\textit{\tiny 6}
Abraão voltou correndo para a tenda e disse a Sara: “Rápido! Pegue três
medidas da melhor farinha, amasse-a e faça alguns pães”. 
\textit{\tiny 7}
Em seguida, Abraão
correu ao rebanho, escolheu um novilho tenro e o entregou a seu servo, que o
preparou rapidamente. 
\textit{\tiny 8}
Quando a comida estava pronta, Abraão pegou coalhada,
leite e a carne assada e os serviu aos visitantes. Enquanto comiam, Abraão
permaneceu à disposição deles, à sombra das árvores.

\bigskip   
\textit{\tiny 9}
“Onde está Sara, sua mulher?”, perguntaram os visitantes.
   “Está dentro da tenda”, respondeu Abraão.
\textit{\tiny 10}
Então um deles disse: “Voltarei a visitar você por esta época, no ano que vem,
e sua mulher, Sara, terá um filho”.
   Sara estava ouvindo a conversa de dentro da tenda. 
\textit{\tiny 11}
Abraão e Sara já eram
bem velhos, e Sara tinha passado, havia muito tempo, da idade de ter filhos. 
\textit{\tiny 12}
Por
isso, riu consigo e disse: “Como poderia uma mulher da minha idade ter esse
prazer, ainda mais quando meu senhor, meu marido, também é idoso?”.

\bigskip   
\textit{\tiny 13}
Então o SENHOR disse a Abraão: “Por que Sara riu? Por que disse: ‘Pode uma
mulher da minha idade ter um filho’? 
\textit{\tiny 14}
Existe alguma coisa difícil demais para o
SENHOR? Voltarei por esta época, no ano que vem, e Sara terá um filho”.
\textit{\tiny 15}
Sara teve medo e, por isso, mentiu: “Eu não ri”.
   Mas ele disse: “Não é verdade. Você riu”.

\bigskip   
\subsubsection*{Abraão e Sodoma}
\textit{\tiny 16}
Depois da refeição, os visitantes se levantaram e olharam em direção a Sodoma.
Quando partiram, Abraão os acompanhou para despedir-se deles.
\textit{\tiny 17}
Então o SENHOR disse: “Devo esconder meu plano de Abraão? 
\textit{\tiny 18}
Afinal,
Abraão certamente se tornará uma grande e poderosa nação, e todas as nações da
terra serão abençoadas por meio dele. 
\textit{\tiny 19}
Eu o escolhi para que ordene a seus
filhos e às famílias deles que guardem o caminho do SENHOR, praticando o que é
certo e justo. Então farei por Abraão tudo que prometi”.

\bigskip   
\textit{\tiny 20}
Portanto, o SENHOR disse a Abraão: “Ouvi um grande clamor vindo de Sodoma
e Gomorra, porque o pecado dessas duas cidades é extremamente grave.
\textit{\tiny 21}
Descerei para investigar se seus atos são, de fato, tão perversos quanto tenho
ouvido. Se não forem, quero saber”.

\bigskip   
\textit{\tiny 22}
Os outros visitantes partiram para Sodoma, mas Abraão permaneceu diante
do SENHOR. 
\textit{\tiny 23}
Aproximou-se dele e disse: “Exterminarás tanto os justos como os
perversos? 
\textit{\tiny 24}
Suponhamos que haja cinquenta justos na cidade. Mesmo assim os
exterminarás e não a pouparás por causa deles? 
\textit{\tiny 25}
Claro que não farias tal coisa:
destruir o justo com o perverso. Afinal, estarias tratando o justo e o perverso da
mesma maneira! Certamente não farias isso! Acaso o Juiz de toda a terra não faria
o que é certo?”.
\textit{\tiny 26}
O SENHOR respondeu: “Se eu encontrar cinquenta justos em Sodoma,
pouparei a cidade toda por causa deles”.

\bigskip   
\textit{\tiny 27}
Abraão voltou a falar: “Embora eu seja apenas pó e cinza, permita-me dizer
mais uma coisa ao meu Senhor. 
\textit{\tiny 28}
Suponhamos que haja apenas quarenta e cinco
justos, e não cinquenta. Destruirás a cidade toda por falta de cinco justos?”.
   O SENHOR disse: “Se encontrar ali quarenta e cinco justos, não a destruirei”.

\bigskip   
\textit{\tiny 29}
Abraão levou seu pedido ainda mais longe: “Suponhamos que haja apenas
quarenta”.
   O SENHOR respondeu: “Por causa dos quarenta, não a destruirei”.

\bigskip   
\textit{\tiny 30}
“Por favor, não fiques irado comigo, meu Senhor”, suplicou Abraão.
“Permita-me falar. Suponhamos que haja apenas trinta justos.”
   O SENHOR disse: “Se encontrar ali trinta justos, não a destruirei”.

\bigskip   
\textit{\tiny 31}
Abraão prosseguiu: “Uma vez que tive a ousadia de falar ao Senhor, permita-
me continuar. Suponhamos que haja apenas vinte”.
   O SENHOR respondeu: “Por causa dos vinte, não a destruirei”.

\bigskip   
\textit{\tiny 32}
Por fim, Abraão disse: “Senhor, não fiques irado comigo por eu falar mais
uma vez. Suponhamos que haja apenas dez”.
   O SENHOR respondeu: “Por causa dos dez, não a destruirei”.

\bigskip   
\textit{\tiny 33}
Quando terminou a conversa com Abraão, o SENHOR partiu, e Abraão voltou
para sua tenda.
   
\bigskip   
\subsubsection*{A destruição de Sodoma e Gomorra}
\textbf{\large 19}
 Ao anoitecer, os dois anjos chegaram à entrada da cidade de Sodoma. Ló
estava sentado ali. Ao avistá-los, levantou-se para recebê-los. Deu-lhes boas-vindas, curvou-se com o rosto no chão 
\textit{\tiny 2}
e disse: “Meus senhores, venham à minha
casa para lavar os pés e sejam meus hóspedes esta noite. Amanhã, poderão
levantar-se cedo e seguir viagem”.
   “Não”, responderam eles. “Passaremos a noite aqui, na praça da cidade.”
\textit{\tiny 3}
Mas Ló insistiu muito e, por fim, eles o acompanharam até sua casa. Ló lhes
preparou um banquete completo, com pão fresco sem fermento, e eles comeram.

\bigskip   
\textit{\tiny 4}
Ainda não tinham ido se deitar quando todos os homens de Sodoma, jovens e
velhos, chegaram de toda parte da cidade e cercaram a casa. 
\textit{\tiny 5}
Gritaram para Ló:
“Onde estão os homens que vieram passar a noite em sua casa? Traga-os aqui fora
para nós, para que tenhamos relações com eles!”.

\bigskip   
\textit{\tiny 6}
Ló saiu para conversar com os homens e fechou a porta atrás de si. 
\textit{\tiny 7}
“Por favor, meus irmãos, não cometam tamanha maldade”, suplicou. 
\textit{\tiny 8}
“Escutem, tenho
duas filhas virgens. Deixem-me trazê-las para fora, e vocês poderão fazer com
elas o que desejarem. Mas, por favor, deixem os homens em paz, pois são meus
hóspedes e estão sob minha proteção.”

\bigskip   
\textit{\tiny 9}
“Saia da frente!”, gritaram eles. “Esse sujeito é um estrangeiro que se mudou
para a cidade e, agora, age como se fosse nosso juiz! Faremos a você coisas bem
piores do que a seus hóspedes!” Então partiram para cima de Ló, tentando
arrombar a porta.

\bigskip   
\textit{\tiny 10}
Os dois anjos, porém, estenderam a mão, puxaram Ló para dentro da casa e
trancaram a porta. 
\textit{\tiny 11}
Depois, cegaram todos os homens, jovens e velhos, que
estavam à porta, de modo que eles se cansaram e desistiram de invadir a casa.
\textit{\tiny 12}
Os anjos perguntaram a Ló: “Você tem outros parentes na cidade? Tire-os
todos daqui: genros, filhos, filhas ou qualquer outro parente, 
\textit{\tiny 13}
pois estamos
prestes a destruir toda a cidade. O clamor contra ela é tão grande que chegou ao
SENHOR, e ele nos enviou para destruí-la”.

\bigskip   
\textit{\tiny 14}
Então Ló correu para avisar os noivos de suas filhas: “Saiam depressa da
cidade! O SENHOR está prestes a destruí-la”. Os rapazes, porém, pensaram que ele
estava brincando.

\bigskip   
\textit{\tiny 15}
No dia seguinte, ao amanhecer, os anjos insistiram: “Rápido! Tome sua
mulher e suas duas filhas que estão aqui! Saia agora mesmo, ou também morrerá
quando a cidade for castigada!”.
\textit{\tiny 16}
Visto que Ló ainda hesitava, os anjos o tomaram pela mão, e também sua
mulher e as duas filhas, e correram com eles para um lugar seguro, fora da cidade,
pois o SENHOR foi misericordioso. 

\bigskip   
\textit{\tiny 17}
Quando estavam em segurança, fora da cidade, um dos anjos ordenou: “Corram e salvem-se! Não olhem para trás nem
parem no vale! Fujam para as montanhas, ou serão destruídos!”.
\textit{\tiny 18}
Mas Ló suplicou: “Não, meu senhor! 
\textit{\tiny 19}
Os senhores foram muito bondosos
comigo, salvaram minha vida e mostraram grande compaixão. Não posso,
contudo, ir para as montanhas. A calamidade também me alcançaria ali, e bem
depressa eu morreria. 
\textit{\tiny 20}
Vejam, aqui perto há um vilarejo. É um lugar bem
pequeno. Por favor, deixem-me ir para lá, e minha vida será salva”.
\textit{\tiny 21}
“Está bem”, disse o anjo. “Atenderei a seu pedido. Não destruirei o vilarejo.
\textit{\tiny 22}
Mas vá logo! Fuja para ele, pois não posso fazer nada enquanto você não chegar
lá.” (Isso explica por que a vila era conhecida como Zoar.)

\bigskip   
\textit{\tiny 23}
Ló chegou a Zoar quando o sol aparecia no horizonte. 
\textit{\tiny 24}
Então o SENHOR fez
chover do céu fogo e enxofre sobre Sodoma e Gomorra. 
\textit{\tiny 25}
Destruiu-as
completamente, além de outras cidades e vilas da planície, e exterminou todos os
habitantes e toda a vegetação. 
\textit{\tiny 26}
A mulher de Ló, porém, olhou para trás enquanto
o seguia e se transformou numa coluna de sal.
\textit{\tiny 27}
Naquela manhã, Abraão se levantou cedo e correu para o lugar onde tinha
estado na presença do SENHOR. 
\textit{\tiny 28}
Olhou para a planície, em direção a Sodoma e
Gomorra, e viu colunas de fumaça subindo do lugar onde antes ficavam as
cidades, como fumaça de uma fornalha.
\textit{\tiny 29}
Contudo, Deus atendeu ao pedido de Abraão e salvou Ló, tirando-o do meio
da destruição que engoliu as cidades da planície.

\bigskip   
\subsubsection*{Moabitas e Anomitas}
\textit{\tiny 30}
Algum tempo depois, Ló deixou Zoar, pois tinha medo do povo de lá, e foi morar
numa caverna nas montanhas com suas duas filhas. 

\bigskip   
\textit{\tiny 31}
Certo dia, a filha mais velha
disse à irmã: “Nesta região não resta homem algum com quem possamos ter
relações, como fazem todas as pessoas. E logo nosso pai será velho demais para
ter filhos. 
\textit{\tiny 32}
Vamos embebedá-lo com vinho e então nos deitaremos com ele. Com
isso, preservaremos nossa descendência por meio de nosso pai”.
\textit{\tiny 33}
Naquela noite, portanto, embebedaram o pai com vinho, e a filha mais velha
teve relações com ele. E ele não percebeu quando ela se deitou nem quando se
levantou.

\bigskip   
\textit{\tiny 34}
Na manhã seguinte, a filha mais velha disse à irmã mais nova: “Ontem à noite,
tive relações com nosso pai. Vamos embebedá-lo com vinho outra vez hoje à
noite, e você terá relações com ele. Com isso, preservaremos nossa descendência
por meio de nosso pai”.
\textit{\tiny 35}
Naquela noite, portanto, voltaram a embebedar o pai com vinho, e a filha
mais nova teve relações com ele. Mais uma vez, ele não percebeu quando ela se
deitou nem quando se levantou.

\bigskip   
\textit{\tiny 36}
Como resultado, as duas filhas de Ló engravidaram do próprio pai. 

\bigskip   
\textit{\tiny 37}
Quando
a filha mais velha deu à luz um menino, chamou-o de Moabe. Ele se tornou o
antepassado do povo conhecido até hoje como moabitas. 
\textit{\tiny 38}
Quando a filha mais
nova deu à luz um menino, chamou-o de Ben-Ami. Ele se tornou o antepassado
do povo conhecido até hoje como amonitas.
Abraão mente para Abimeleque
   
\bigskip   
\subsubsection*{Abraão em Gerar}
\textbf{\large 20}
 Abraão se mudou para o Neguebe, ao sul. Permaneceu por algum tempo
entre Cades e Sur e depois seguiu até Gerar. Enquanto morava ali como
estrangeiro, 
\textit{\tiny 2}
Abraão apresentava Sara, sua mulher, dizendo: “Ela é minha irmã”.
Por isso, o rei Abimeleque, de Gerar, mandou buscar Sara para seu palácio.

\bigskip   
\textit{\tiny 3}
Naquela noite, Deus apareceu a Abimeleque num sonho e lhe disse: “Você vai
morrer! A mulher que tomou já é casada!”.
\textit{\tiny 4}
Abimeleque, porém, ainda não havia dormido com ela. Assim, disse: “Senhor,
castigarás uma nação inocente? 
\textit{\tiny 5}
Não foi Abraão quem me disse: ‘Ela é minha
irmã’? E ela própria afirmou: ‘Sim, ele é meu irmão’? Agi com total inocência.
Minhas mãos estão limpas!”.
\textit{\tiny 6}
No sonho, Deus respondeu: “Sim, eu sei que você é inocente. Por isso o impedi
de pecar e não deixei que a tocasse. 
\textit{\tiny 7}
Agora, devolva a mulher ao marido dela, e ele
orará por você, pois é profeta. Então você viverá. Mas, se não a devolver, esteja
certo de que você e todo o seu povo morrerão”.

\bigskip   
\textit{\tiny 8}
Na manhã seguinte, Abimeleque se levantou cedo e, sem demora, reuniu todos
os seus servos. Quando contou o que havia acontecido, seus homens se encheram
de medo. 
\textit{\tiny 9}
Então Abimeleque mandou chamar Abraão. “O que você fez conosco?”,
perguntou. “Que crime cometi para merecer este tratamento que nos torna, a
mim e ao meu reino, culpados deste grande pecado? O que você me fez não se faz
a ninguém! 
\textit{\tiny 10}
O que deu em você para agir desse jeito?”

\bigskip   
\textit{\tiny 11}
Abraão respondeu: “Pensei comigo: ‘Este é um lugar onde ninguém teme a
Deus, e vão me matar para ficarem com minha mulher’. 
\textit{\tiny 12}
Além do mais, ela é, de
fato, minha irmã por parte de pai, mas não de mãe, e eu me casei com ela.
\textit{\tiny 13}
Quando Deus me chamou para deixar a casa de meu pai e viajar de um lugar para outro, eu disse a ela: ‘Faça-me este favor: por onde formos, diga que eu sou
seu irmão’”.

\bigskip   
\textit{\tiny 14}
Então Abimeleque pegou ovelhas e bois, servos e servas, e os deu de presente
a Abraão. Também lhe devolveu Sara, sua mulher. 
\textit{\tiny 15}
Abimeleque disse: “Veja,
minha terra está à sua disposição. More onde lhe parecer melhor”. 
\textit{\tiny 16}
E disse a
Sara: “Estou dando a seu irmão mil peças de prata diante de todas estas
testemunhas para reparar qualquer dano que eu lhe tenha causado. Assim, todos
saberão que você é inocente”.

\bigskip   
\textit{\tiny 17}
Então Abraão orou a Deus, e Deus curou Abimeleque, sua mulher e suas
servas, de modo que pudessem ter filhos, 
\textit{\tiny 18}
pois o SENHOR havia tornado estéreis
todas as mulheres do harém de Abimeleque por causa do que tinha acontecido
com Sara, mulher de Abraão.
   
\bigskip   
\subsubsection*{Isaque}
\textbf{\large 21}
 O SENHOR agiu em favor de Sara e cumpriu o que lhe tinha prometido. 
\textit{\tiny 2}
Ela
engravidou e deu à luz um filho para Abraão na velhice dele, exatamente no
tempo indicado por Deus. 
\textit{\tiny 3}
Abraão deu o nome Isaque ao filho que Sara lhe deu.

\bigskip   
\textit{\tiny 4}
No oitavo dia depois do nascimento de Isaque, Abraão o circuncidou, como Deus
havia ordenado. 
\textit{\tiny 5}
Abraão tinha 100 anos quando Isaque nasceu.

\bigskip   
\textit{\tiny 6}
Sara declarou: “Deus me fez sorrir. Todos que ficarem sabendo do que
aconteceu vão rir comigo!”. 
\textit{\tiny 7}
E disse mais: “Quem diria a Abraão que sua mulher
amamentaria um bebê? E, no entanto, em sua velhice, eu lhe dei um filho!”.
Abraão expulsa Hagar e Ismael

\bigskip   
\subsubsection*{Isaque e Ismael}
\textit{\tiny 8}
Quando Isaque cresceu e estava para ser desmamado, Abraão preparou uma
grande festa para comemorar a ocasião. 
\textit{\tiny 9}
Sara, porém, viu Ismael, filho de Abraão
e da serva egípcia Hagar, caçoar de seu filho, Isaque,
\textit{\tiny 10}
e disse a Abraão: “Livre-
se da escrava e do filho dela! Ele jamais será herdeiro junto com meu filho,
Isaque!”.
\textit{\tiny 11}
Abraão ficou muito perturbado com isso, pois Ismael era seu filho. 

\bigskip   
\textit{\tiny 12}
Deus,
porém, lhe disse: “Não se perturbe por causa do menino e da serva. Faça tudo que
Sara lhe pedir, pois Isaque é o filho de quem depende a sua descendência.
\textit{\tiny 13}
Contudo, também farei uma nação dos descendentes do filho de Hagar, pois ele
é seu filho”.

\bigskip   
\textit{\tiny 14}
Na  manhã seguinte, Abraão se levantou cedo, preparou mantimentos e uma
vasilha cheia de água e os pôs sobre os ombros de Hagar. Então, mandou-a
embora com seu filho, e ela andou sem rumo pelo deserto de Berseba.
\textit{\tiny 15}
Quando acabou a água, Hagar colocou o menino à sombra de um arbusto 
\textit{\tiny 16}
e
foi sentar-se sozinha, uns cem metros adiante. “Não quero ver o menino
morrer”, disse ela, chorando sem parar.

\bigskip   
\textit{\tiny 17}
Mas Deus ouviu o choro do menino e, do céu, o anjo de Deus chamou Hagar:
“Que foi, Hagar? Não tenha medo! Deus ouviu o menino chorar, dali onde ele está.
\textit{\tiny 18}
Levante-o e anime-o, pois farei dos descendentes dele uma grande nação”.
\textit{\tiny 19}
Então Deus abriu os olhos de Hagar, e ela viu um poço cheio de água. Sem
demora, encheu a vasilha de água e deu para o menino beber.
\textit{\tiny 20}
Deus estava com o menino enquanto ele crescia no deserto. Ismael se tornou
flecheiro 
\textit{\tiny 21}
e se estabeleceu no deserto de Parã, e sua mãe conseguiu para ele uma
esposa egípcia.

\bigskip   
\subsubsection*{Abraão em Berseba}
\textit{\tiny 22}
Por esse tempo, Abimeleque, acompanhado de Ficol, comandante do seu
exército, foi visitar Abraão. “É evidente que Deus está com você, ajudando-o em
tudo que faz”, disse Abimeleque. 
\textit{\tiny 23}
“Jure, em nome de Deus, que não enganará
nem a mim, nem a meus filhos, nem a nenhum de meus descendentes. Tenho
sido leal a você, por isso jure que será leal a mim e a esta terra onde vive como
estrangeiro.”
\textit{\tiny 24}
Abraão respondeu: “Eu juro!”. 
\textit{\tiny 25}
Contudo, Abraão reclamou com Abimeleque
sobre um poço que os servos de Abimeleque lhe haviam tomado à força.
\textit{\tiny 26}
“Eu não sabia disso”, respondeu Abimeleque. “Não faço ideia de quem seja o
responsável. Você nunca se queixou a esse respeito.”
\textit{\tiny 27}
Então Abraão deu ovelhas e bois a Abimeleque, e os dois fizeram um acordo.

\bigskip   
\textit{\tiny 28}
Quando Abraão também separou do rebanho mais sete cordeirinhas,
\textit{\tiny 29}
Abimeleque lhe perguntou: “Por que você separou estas sete das demais?”.
\textit{\tiny 30}
Abraão respondeu: “Por favor, aceite estas sete cordeirinhas como
testemunho de que eu cavei este poço”. 
\textit{\tiny 31}
Por isso Abraão chamou o lugar de
Berseba, porque ali os dois fizeram o juramento.

\bigskip   
\textit{\tiny 32}
Depois de firmarem a aliança em Berseba, Abimeleque e Ficol, comandante
do seu exército, voltaram para a terra dos filisteus. 
\textit{\tiny 33}
Abraão plantou uma
tamargueira em Berseba e ali invocou o nome do SENHOR, o Deus Eterno.
\textit{\tiny 34}
E
Abraão morou na terra dos filisteus como estrangeiro por longo tempo.

\bigskip   
\subsubsection*{Abraão em Moriá (o juramento da promessa)}
\textbf{\large 22}
 Algum tempo depois, Deus pôs Abraão à prova. 

\bigskip   
“Abraão!”, Deus chamou.
   “Sim”, respondeu Abraão. “Aqui estou!”
\textit{\tiny 2}
Deus disse: “Tome seu filho, seu único filho, Isaque, a quem você tanto ama, e
vá à terra de Moriá. Lá, em um dos montes que eu lhe mostrarei, ofereça-o como
holocausto”.

\bigskip   
\textit{\tiny 3}
Na manhã seguinte, Abraão se levantou cedo e preparou seu jumento. Levou
consigo dois de seus servos e seu filho Isaque. Cortou lenha para o fogo do
holocausto e partiu para o lugar que Deus tinha indicado. 

\bigskip   
\textit{\tiny 4}
No terceiro dia da
viagem, Abraão levantou os olhos e viu o lugar de longe. 
\textit{\tiny 5}
“Fiquem aqui com o
jumento”, disse ele aos servos. “O rapaz e eu iremos mais adiante. Vamos adorar e
depois voltaremos.”
\textit{\tiny 6}
Abraão pôs a lenha para o holocausto nos ombros de Isaque, e ele próprio
levou o fogo e a faca. 
Enquanto os dois caminhavam juntos, 
\textit{\tiny 7}
Isaque se virou para
Abraão e disse: “Pai?”.
   “Sim, meu filho”, respondeu Abraão.
   “Temos fogo e lenha”, disse Isaque. “Mas onde está o cordeiro para o
holocausto?”
\textit{\tiny 8}
“Deus providenciará o cordeiro para o holocausto, meu filho”, respondeu
Abraão. E continuaram a caminhar juntos.

\bigskip   
\textit{\tiny 9}
Quando chegaram ao lugar que Deus havia indicado, Abraão construiu um
altar e arrumou a lenha sobre ele. Em seguida, amarrou seu filho Isaque e o
colocou no altar, sobre a lenha. 
\textit{\tiny 10}
Então, pegou a faca para sacrificar o filho.

\bigskip   
\textit{\tiny 11}
Nesse momento, o anjo do SENHOR o chamou do céu: “Abraão! Abraão!”.
   “Aqui estou!”, respondeu Abraão.
\textit{\tiny 12}
“Não toque no rapaz”, disse o anjo. “Não lhe faça mal algum. Agora sei que
você teme a Deus de fato. Não me negou nem mesmo seu filho, seu único filho!”

\bigskip   
\textit{\tiny 13}
Então Abraão levantou os olhos e viu um carneiro preso pelos chifres num
arbusto. Pegou o carneiro e o ofereceu como holocausto em lugar do filho.
\textit{\tiny 14}
Abraão chamou aquele lugar de Javé-Jiré. Até hoje, as pessoas usam esse nome
como provérbio: “No monte do SENHOR se providenciará”.

\bigskip   
\textit{\tiny 15}
Então o anjo do SENHOR chamou Abraão novamente do céu: 
\textit{\tiny 16}
“Assim diz o
SENHOR: Uma vez que você me obedeceu e não me negou nem mesmo seu filho,
seu único filho, juro pelo meu nome que 
\textit{\tiny 17}
certamente o abençoarei. Multiplicarei
grandemente seus descendentes, e eles serão como as estrelas no céu e a areia na
beira do mar. Seus descendentes conquistarão as cidades de seus inimigos 
\textit{\tiny 18}
e,
por meio deles, todas as nações da terra serão abençoadas. Tudo isso porque você
me obedeceu”.

\bigskip   
\textit{\tiny 19}
Então voltaram até onde estavam os servos e partiram para Berseba, onde
Abraão continuou a morar.

\bigskip   
\subsubsection*{Descendentes de Naor (avô de Rebeca)}
\textit{\tiny 20}
Pouco tempo depois, Abraão ficou sabendo que Milca, mulher de Naor,
irmão dele, lhe tinha dado filhos. 
\textit{\tiny 21}
O mais velho recebeu o nome de Uz, o
segundo mais velho, Buz, seguido de Quemuel (antepassado dos arameus),
\textit{\tiny 22}
Quésede, Hazo, Pildás, Jidlafe e Betuel 
\textit{\tiny 23}
(que foi o pai de Rebeca). Esses foram
os oito filhos que Milca deu a Naor, irmão de Abraão. 
\textit{\tiny 24}
Além desses, Reumá, sua
concubina, lhe deu quatro filhos: Tebá, Gaã, Taás e Maaca.

----------------------------------------------------------------------
