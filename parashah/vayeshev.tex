\section*{Parashat Vayeshev 37:1 - 40:23}


\subsubsection*{José e os sonhos}
\textbf{\large 37} Jacó passou a morar na terra de Canaã, onde seu pai tinha vivido como
estrangeiro. 

\bigskip
\textit{\tiny 2}
Este é o relato de Jacó e sua família. Quando José tinha 17 anos, cuidava dos
rebanhos de seu pai. Trabalhava com seus meios-irmãos, os filhos de Bila e Zilpa,
mulheres de seu pai, e contava para seu pai algumas das coisas erradas que seus
irmãos faziam. 
\textit{\tiny 3}
Jacó
 amava José mais que a qualquer outro de seus filhos, pois José havia
nascido quando Jacó era idoso. Por isso, certo dia Jacó encomendou um presente
especial para José: uma linda túnica. 
\textit{\tiny 4}
Os irmãos de José, por sua vez, o odiavam,
pois o pai deles o amava mais que a todos os outros filhos. Não eram capazes de
lhe dizer uma única palavra amigável. 

\bigskip
\textit{\tiny 5}
Certa noite, José teve um sonho e, quando o contou a seus irmãos, eles o
odiaram ainda mais. 
\textit{\tiny 6}
“Ouçam este sonho que tive”, disse ele. 
\textit{\tiny 7}
“Estávamos no campo, amarrando feixes de trigo. De repente, meu feixe se levantou e ficou em
pé, e seus feixes se juntaram ao redor do meu e se curvaram diante dele!” 
\textit{\tiny 8}
Seus irmãos responderam: “Você imagina que será nosso rei? Pensa mesmo
que nos governará?”. E o odiaram ainda mais por causa de seus sonhos e da
maneira como os contava. 

\bigskip
\textit{\tiny 9}
Pouco tempo depois, José teve outro sonho e, mais uma vez, contou-o a seus
irmãos. “Ouçam, tive outro sonho”, disse ele. “O sol, a lua e onze estrelas se
curvavam diante de mim!”
\textit{\tiny 10}
Dessa vez, contou o sonho não apenas aos irmãos, mas também ao pai, que o
repreendeu, dizendo: “Que sonho é esse? Por acaso eu, sua mãe e seus irmãos
viremos e nos curvaremos até o chão diante de você?”. 
\textit{\tiny 11}
Os irmãos de José
ficaram com inveja dele, mas seu pai se perguntou qual seria o significado dos
sonhos.

\bigskip
\subsubsection*{José e a escravidão}
\textit{\tiny 12}
Pouco depois, os irmãos de José levaram os rebanhos de seu pai para pastar
junto de Siquém. 
\textit{\tiny 13}
Então Jacó disse a José: “Seus irmãos estão cuidando das
ovelhas em Siquém. Apronte-se, e eu o enviarei até eles”.
   “Estou pronto para ir”, respondeu José.
\textit{\tiny 14}
“Vá ver como estão seus irmãos e os rebanhos”, disse Jacó. “E traga-me
notícias deles.” Jacó o enviou, e José viajou de sua casa no vale de Hebrom até
Siquém.
   
\bigskip
\textit{\tiny 15}
Quando José chegou a Siquém, um homem da região notou que ele andava
perdido pelos campos. “O que você está procurando?”, perguntou o homem.
\textit{\tiny 16}
“Estou procurando meus irmãos”, respondeu José. “O senhor sabe onde eles
estão cuidando dos rebanhos?”
\textit{\tiny 17}
O homem lhe disse: “Sim, eles foram embora daqui, mas eu os ouvi dizer:
‘Vamos a Dotã’”. Então José foi atrás de seus irmãos e os encontrou em Dotã.
   
\bigskip
\textit{\tiny 18}
Quando os irmãos de José o viram, o reconheceram de longe. Antes que ele se
aproximasse, planejaram uma forma de matá-lo. 
\textit{\tiny 19}
“Lá vem o sonhador!”,
disseram uns aos outros. 
\textit{\tiny 20}
“Vamos matá-lo e jogá-lo numa dessas cisternas.
Diremos a nosso pai: ‘Um animal selvagem o devorou’. Então veremos o que será
dos seus sonhos!”
\textit{\tiny 21}
Mas, quando Rúben ouviu o plano, tratou de livrar José. “Não o matemos”,
disse ele. 
\textit{\tiny 22}
“Por que derramar sangue? Joguem-no nesta cisterna vazia aqui no
deserto e não toquemos nele.” Rúben planejava resgatar José e levá-lo de volta ao
pai.

\bigskip
\textit{\tiny 23}
Assim, quando José chegou, os irmãos lhe arrancaram a linda túnica que ele
estava usando, 
\textit{\tiny 24}
o agarraram e o jogaram na cisterna vazia, ou seja, sem água.

\bigskip
\textit{\tiny 25}
Mais tarde, quando se sentaram para comer, viram ao longe uma caravana de
camelos vindo em sua direção. Era um grupo de negociantes ismaelitas, que
transportavam especiarias, bálsamo e mirra de Gileade para o Egito.
\textit{\tiny 26}
Judá disse a seus irmãos: “O que ganharemos se matarmos nosso irmão e
encobrirmos o crime?
\textit{\tiny 27}
Em vez de matá-lo, vamos vendê-lo aos negociantes
ismaelitas. Afinal, ele é nosso irmão, sangue do nosso sangue!”. Seus irmãos
concordaram. 
\textit{\tiny 28}
Então, quando os ismaelitas, que eram negociantes midianitas, se
aproximaram, os irmãos de José o tiraram da cisterna e o venderam para eles por
vinte peças de prata. E os negociantes o levaram para o Egito.
   
\bigskip
\textit{\tiny 29}
Algum tempo depois, Rúben voltou para tirar José da cisterna. Quando
descobriu que seu irmão não estava lá, rasgou as roupas. 
\textit{\tiny 30}
Voltou a seus irmãos e
lamentou-se: “O menino sumiu! E agora, o que farei?”.
\textit{\tiny 31}
Então os irmãos mataram um bode e mergulharam a túnica de José no
sangue do animal. 
\textit{\tiny 32}
Enviaram a linda túnica para o pai, com a seguinte
mensagem: “Veja o que encontramos. Não é a túnica de seu filho?”.

\bigskip
\textit{\tiny 33}
O pai a reconheceu de imediato e disse: “Sim, é a túnica de meu filho. Um
animal selvagem o deve ter devorado. Com certeza José morreu despedaçado!”.
\textit{\tiny 34}
Jacó rasgou suas roupas e vestiu-se de pano de saco. Por longo tempo, lamentou
profundamente a morte do filho. 
\textit{\tiny 35}
A família toda tentou consolá-lo, mas ele se
recusava. “Descerei à sepultura
lamentando a morte de meu filho”, dizia, e
continuou a lamentar-se.
   
\bigskip
\textit{\tiny 36}
Enquanto isso, os negociantes midianitas
chegaram ao Egito, onde
venderam José a Potifar, oficial e capitão da guarda do faraó.


\bigskip
\subsubsection*{Tamar}
\textbf{\large 38}
   Por essa época, Judá saiu de casa e se mudou para Adulão, onde foi morar
na casa de um homem chamado Hira. 

\bigskip
\textit{\tiny 2} 
Ali, Judá viu uma mulher cananita, filha de
Suá, e se casou com ela. Teve relações com a mulher, 
\textit{\tiny 3} 
e ela engravidou e deu à luz
um filho, que ele chamou de Er. 
\textit{\tiny 4} 
Ela engravidou novamente e deu à luz outro
filho, que chamou de Onã. 
\textit{\tiny 5} 
Quando estavam morando em Quezibe, ela deu à luz
o terceiro filho e o chamou de Selá. 

\bigskip
\textit{\tiny 6} 
No   devido tempo, Judá arranjou o casamento de Er, seu filho mais velho, com
uma moça chamada Tamar. 
\textit{\tiny 7} 
Mas Er era um homem perverso aos olhos do
SENHOR, por isso o SENHOR lhe tirou a vida. 
\textit{\tiny 8} 
Então Judá disse a Onã, irmão de Er:
“Case-se com Tamar, como é exigido para com a viúva do irmão. Você deve gerar
um herdeiro para seu irmão”. 
\textit{\tiny 9} 
Onã, porém, não estava disposto a ter um filho que não seria seu herdeiro. Por
isso, cada vez que tinha relações com a mulher de seu irmão, derramava o sêmen
no chão. Desse modo, evitava que Tamar tivesse um filho que pertenceria ao
irmão dele. 
\textit{\tiny 10}
O SENHOR, porém, considerou maldade a sua atitude e, por isso,
também tirou a vida de Onã.
\textit{\tiny 11}
Então Judá disse à sua nora Tamar: “Volte para a casa de seus pais e
permaneça viúva até que meu filho Selá tenha idade suficiente para se casar com
você”. (Na verdade, Judá disse isso apenas porque temia que Selá também
morresse, como seus dois irmãos.) Assim, Tamar voltou para a casa do pai.

\bigskip      
\textit{\tiny 12}
Alguns anos depois, a mulher de Judá morreu. Quando terminou o período
de luto, Judá e seu amigo Hira, o adulamita, subiram a Timna para supervisionar a
tosquia das ovelhas de Judá. 
\textit{\tiny 13}
Alguém disse a Tamar: “Seu sogro está subindo a
Timna para tosquiar as ovelhas”.
\textit{\tiny 14}
Tamar sabia que Selá já era adulto, mas nenhuma providência havia sido
tomada para que ela se casasse com ele. Por isso, trocou suas roupas de viúva e,
para disfarçar-se, cobriu-se com um véu. Depois, foi sentar-se junto à entrada da
vila de Enaim, no caminho para Timna. 
\textit{\tiny 15}
Judá a viu e pensou que fosse uma
prostituta, pois ela estava com o rosto coberto. 
\textit{\tiny 16}
Ele parou à beira da estrada e,
sem saber que era sua própria nora, disse: “Quero me deitar com você”.
   “Quanto você me pagará para deitar-se comigo?”, perguntou Tamar.
\textit{\tiny 17}
“Eu lhe mandarei um cabrito do meu rebanho”, prometeu Judá.
   “Mas o que me dá como garantia de que mandará o cabrito?”, perguntou ela.
\textit{\tiny 18}
“Que tipo de garantia você quer?”, replicou ele.
   Ela disse: “Deixe comigo seu selo pessoal, junto com o cordão dele e o cajado
que você está segurando”. Judá entregou os objetos. Depois, teve relações com
Tamar, e ela engravidou. 

\bigskip
\textit{\tiny 19}
Em seguida, Tamar voltou para casa, tirou o véu e
tornou a vestir as roupas de viúva, como de costume.
\textit{\tiny 20}
Mais tarde, Judá pediu que seu amigo Hira, o adulamita, levasse o cabrito para
a mulher e pegasse de volta as coisas que ele havia deixado como garantia. Hira,
porém, não conseguiu encontrá-la. 
\textit{\tiny 21}
Perguntou aos homens que moravam lá:
“Onde posso encontrar a prostituta do templo que estava sentada junto à entrada
de Enaim?”.
   “Aqui nunca houve uma prostituta do templo”, responderam eles.
\textit{\tiny 22}
Então Hira voltou para onde Judá estava e lhe disse: “Não consegui encontrá-
la em lugar algum, e os homens da vila disseram que lá nunca houve uma
prostituta do templo”.
\textit{\tiny 23}
“Que ela fique com as minhas coisas”, disse Judá. “Mandei o cabrito como
tínhamos combinado, mas você não a encontrou. Se voltássemos para procurá-la,
o povo da vila zombaria de nós.”

\bigskip   
\textit{\tiny 24}
Uns três meses depois, disseram a Judá: “Sua nora, Tamar, se comportou
como prostituta e, por isso, está grávida”.
   Judá ordenou: “Tragam-na para fora e queimem-na!”.
\textit{\tiny 25}
Quando a estavam tirando de casa para matá-la, ela enviou a seguinte
mensagem a seu sogro: “Estou grávida do homem que é dono destes objetos. Olhe
com atenção. De quem são este selo, este cordão e este cajado?”.
\textit{\tiny 26}
Judá os reconheceu de imediato e disse: “Ela é mais justa que eu, pois não
tomei as providências para que ela se casasse com meu filho Selá”. E Judá nunca
mais teve relações com Tamar.
   
\bigskip   
\textit{\tiny 27}
Quando chegou a época de Tamar dar à luz, descobriu que teria gêmeos.
\textit{\tiny 28}
Durante o trabalho de parto, um dos bebês pôs a mão para fora. A parteira
segurou a mão do bebê, amarrou um fio vermelho no pulso e anunciou: “Este saiu
primeiro”. 
\textit{\tiny 29}
O bebê, porém, recolheu a mão, e seu irmão saiu. Então a parteira
disse: “Como você conseguiu sair primeiro?”. Por isso, ele recebeu o nome de
Perez.
\textit{\tiny 30}
Logo depois, o bebê com o fio vermelho no pulso nasceu e recebeu o
nome de Zerá.

\bigskip
\subsubsection*{José e Potifar}
\textbf{\large 39}
   Quando José foi levado para o Egito pelos negociantes ismaelitas, eles o
venderam a Potifar, um oficial egípcio. Potifar era capitão da guarda do faraó, o
rei do Egito. 

\bigskip
\textit{\tiny 2} 
O SENHOR estava com José, por isso ele era bem-sucedido em tudo que fazia no
serviço da casa de seu senhor egípcio. 
\textit{\tiny 3} 
Potifar percebeu que o SENHOR estava com
José e lhe dava sucesso em tudo que ele fazia. 
\textit{\tiny 4} 
Satisfeito com isso, nomeou José
seu assistente pessoal e o encarregou de toda a sua casa e de todos os seus bens. 
\textit{\tiny 5} 
A partir do dia em que José foi encarregado de toda a casa e de todas as
propriedades de Potifar, o SENHOR começou a abençoar a casa do egípcio por
causa de José. Tudo corria bem na casa, e as plantações e os animais prosperavam. 
\textit{\tiny 6} 
Assim, Potifar entregou tudo que possuía aos cuidados de José e, tendo-o como
administrador, não se preocupava com nada, exceto com o que iria comer.

\bigskip
   José era um rapaz muito bonito, de bela aparência, 
\textit{\tiny 7}
e logo a esposa de Potifar
começou a olhar para ele com desejo. “Venha e deite-se comigo”, ordenou ela. 
\textit{\tiny 8} 
José recusou e disse: “Meu senhor me confiou todos os bens de sua casa e não
precisa se preocupar com nada. 
\textit{\tiny 9} 
Ninguém aqui tem mais autoridade que eu. Ele
não me negou coisa alguma, exceto a senhora, pois é mulher dele. Como poderia
eu cometer tamanha maldade? Estaria pecando contra Deus!”.
   
\bigskip
\textit{\tiny 10}
A mulher continuava a assediar José diariamente, mas ele se recusava a
deitar-se com ela. 
\textit{\tiny 11}
Certo dia, porém, quando José entrou para fazer seu trabalho,
não havia mais ninguém na casa. 
\textit{\tiny 12}
Ela se aproximou, agarrou-o pelo manto e
exigiu: “Venha, deite-se comigo!”. José se desvencilhou e fugiu da casa, mas o
manto ficou na mão da mulher.
   
\bigskip
\textit{\tiny 13}
Quando ela viu que José tinha fugido, mas que o manto havia ficado na mão
dela, 
\textit{\tiny 14}
chamou seus servos. “Vejam!”, disse ela. “Meu marido trouxe esse escravo
hebreu para nos fazer de bobos! Ele entrou no meu quarto para me violentar, mas
eu gritei. 
\textit{\tiny 15}
Quando ele me ouviu gritar, saiu correndo e escapou, mas largou seu
manto comigo.”
\textit{\tiny 16}
Ela guardou o manto até o marido voltar para casa. 
\textit{\tiny 17}
Então, contou-lhe sua
versão da história. “O escravo hebreu que você trouxe para nossa casa tentou
aproveitar-se de mim”, disse ela. 
\textit{\tiny 18}
“Mas, quando eu gritei, ele saiu correndo e
largou seu manto comigo!”
\textit{\tiny 19}
Ao ouvir a mulher contar como José a havia tratado, Potifar se enfureceu.
\textit{\tiny 20}
Pegou José e o lançou na prisão onde ficavam os prisioneiros do rei, e ali José
permaneceu. 

\bigskip
\subsubsection*{José e a prisão}
\textit{\tiny 21}
Mas o SENHOR estava com ele na prisão e o tratou com bondade.
Fez José conquistar a simpatia do carcereiro, que, 
\textit{\tiny 22}
em pouco tempo, encarregou
José de todos os outros presos e de todas as tarefas da prisão. 
\textit{\tiny 23}
O carcereiro não
precisava mais se preocupar com nada, pois José cuidava de tudo. O SENHOR estava
com ele e lhe dava sucesso em tudo que ele fazia.

\bigskip
\textbf{\large 40}
   Algum tempo depois, o chefe dos copeiros e o chefe dos padeiros do faraó
ofenderam seu senhor, o rei do Egito. 
\textit{\tiny 2} 
O faraó se enfureceu com os dois oficiais 
\textit{\tiny 3} 
e os mandou para a prisão onde José estava, no palácio do capitão da guarda. 
\textit{\tiny 4}
Eles ficaram presos por um bom tempo, e o capitão da guarda os colocou sob a
responsabilidade de José, para que cuidasse deles. 

\bigskip
\textit{\tiny 5}
Certa noite, enquanto estavam presos, o copeiro e o padeiro tiveram, cada um,
um sonho, e cada sonho tinha o seu significado. 
\textit{\tiny 6} 
Quando José os viu no dia
seguinte, notou que os dois estavam perturbados 
\textit{\tiny 7}
e perguntou: “Por que vocês
estão preocupados?”. 
\textit{\tiny 8}
Eles responderam: “Esta noite, nós dois tivemos sonhos, mas ninguém sabe
nos dizer o que significam”.
   “A interpretação dos sonhos vem de Deus”, disse José. “Contem-me o que
sonharam.” 

\bigskip
\textit{\tiny 9}
O chefe dos copeiros foi o primeiro a relatar seu sonho a José. “Em meu
sonho, vi na minha frente uma videira”, disse ele. 
\textit{\tiny 10}
“Havia três ramos que
começaram a brotar e florescer e, em pouco tempo, produziram cachos de uvas.
\textit{\tiny 11}
Eu tinha na mão o copo do faraó. Tomei um dos cachos de uva, espremi o suco
na taça e a coloquei na mão do faraó.”
   
\bigskip
\textit{\tiny 12}
José disse: “Este é o significado do sonho: os três ramos representam três
dias. 
\textit{\tiny 13}
Dentro de três dias, o faraó o elevará de volta ao seu cargo de chefe dos
copeiros. 
\textit{\tiny 14}
Quando a situação estiver bem para você, peço que se lembre de mim.
Fale de mim ao faraó, para que ele me tire deste lugar, 
\textit{\tiny 15}
pois fui trazido à força da
minha terra natal, a terra dos hebreus, e agora estou nesta prisão, onde fui lançado
sem motivo justo”.
   
\bigskip
\textit{\tiny 16}
Ao ouvir a interpretação favorável de José para o primeiro sonho, o chefe dos
padeiros lhe disse: “Também tive um sonho. Nele, havia três cestos de pães
brancos empilhados sobre a minha cabeça. 
\textit{\tiny 17}
No cesto de cima, havia pães e
doces de todo tipo para o faraó, mas as aves vieram e comeram do cesto que
estava sobre a minha cabeça”.
   
\bigskip
\textit{\tiny 18}
José lhe disse: “Este é o significado do sonho: os três cestos também
representam três dias. 
\textit{\tiny 19}
Dentro de três dias, o faraó pendurará sua cabeça em um
poste, e as aves comerão sua carne”.
   
\bigskip
\textit{\tiny 20}
Três dias depois, era o aniversário do faraó, e ele preparou um banquete para
todos os seus oficiais e funcionários. Convocou o chefe dos copeiros e o chefe dos
padeiros para comparecerem à festa. 
\textit{\tiny 21}
Elevou o chefe dos copeiros de volta a seu
cargo, para que voltasse a entregar o copo ao faraó. 
\textit{\tiny 22}
Quanto ao chefe dos
padeiros, mandou enforcá-lo, como José havia previsto ao interpretar o sonho
dele. 
\textit{\tiny 23}
O chefe dos copeiros, porém, se esqueceu completamente de José e não
pensou mais nele.

----------------------------------------------------------------------
