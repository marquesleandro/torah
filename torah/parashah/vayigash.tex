\section*{Parashat Vayigash 44:18 - 47:27}


\bigskip
\textit{\tiny 18}
Então Judá deu um passo à frente e disse: “Por favor, meu senhor, permita que
seu servo lhe diga apenas uma palavra. Peço que não perca a paciência comigo,
embora o senhor seja tão poderoso quanto o próprio faraó.
\textit{\tiny 19}
“Meu senhor perguntou a nós, seus servos: ‘Vocês têm pai ou irmão?’. 
\textit{\tiny 20}
E nós
respondemos: ‘Sim, meu senhor, nosso pai é idoso e tem um filho mais novo,
nascido em sua velhice. O irmão desse filho, por parte de pai e mãe, morreu. Ele é
o único filho de sua mãe, e nosso pai o ama muito’.

\bigskip
\textit{\tiny 21}
“O senhor nos disse: ‘Tragam-no aqui para que eu possa vê-lo com os
próprios olhos’. 
\textit{\tiny 22}
E nós respondemos: ‘Meu senhor, o rapaz não pode deixar o
pai, pois, se o fizesse, o pai morreria’. 
\textit{\tiny 23}
Mas o senhor nos disse: ‘Vocês não me
verão novamente se não trouxerem seu irmão’.
\textit{\tiny 24}
“Assim, voltamos para seu servo, nosso pai, e contamos a ele o que o senhor
tinha dito. 

\bigskip
\textit{\tiny 25}
Passado algum tempo, quando ele disse: ‘Voltem e comprem mais
mantimentos’, 
\textit{\tiny 26}
nós respondemos: ‘Só poderemos voltar se nosso irmão mais
novo nos acompanhar. Não temos como ver o homem outra vez, a menos que
nosso irmão mais novo esteja conosco’.
\textit{\tiny 27}
“Então meu pai nos disse: ‘Como vocês sabem, minha mulher teve dois filhos,
\textit{\tiny 28}
e um deles foi embora e nunca mais voltou. Sem dúvida, foi despedaçado por
algum animal selvagem, e eu nunca mais o vi. 
\textit{\tiny 29}
Se agora vocês levarem de mim o
irmão dele e lhe acontecer algum mal, vocês me mandarão velho e infeliz para a
sepultura’.

\bigskip
\textit{\tiny 30}
“E agora, meu senhor, não posso voltar para a casa de meu pai sem o rapaz. A
vida de nosso pai está ligada à vida do rapaz. 
\textit{\tiny 31}
Quando ele vir que o rapaz não
está conosco, morrerá. Nós, seus servos, seremos, de fato, responsáveis por
mandar para a sepultura seu servo, nosso pai, em profunda tristeza. 
\textit{\tiny 32}
Meu
senhor, garanti a meu pai que levaria o rapaz de volta. Disse-lhe: ‘Se não o trouxer
de volta, carregarei a culpa para sempre’.
\textit{\tiny 33}
“Por isso, peço ao senhor que me permita ficar aqui como escravo no lugar
do rapaz e que o deixe voltar com os irmãos dele. 
\textit{\tiny 34}
Pois, como poderei voltar a
meu pai sem o rapaz? Não suportaria ver a angústia que isso lhe causaria!”.

\bigskip
\textbf{\large 45} José não conseguiu mais se conter. Havia muita gente na sala, e ele disse a
seus assistentes: “Saiam todos daqui!”. Assim, ficou a sós com seus irmãos e lhes
revelou sua identidade. 
\textit{\tiny 2} 
José se emocionou e começou a chorar. Chorou tão alto
que os egípcios o ouviram, e logo a notícia chegou ao palácio do faraó. 

\bigskip
\textit{\tiny 3}
“Sou eu, José!”, disse a seus irmãos. “Meu pai ainda está vivo?” Mas seus irmãos
ficaram espantados ao se dar conta de que o homem diante deles era José e
perderam a fala. 

\bigskip
\textit{\tiny 4} 
“Cheguem mais perto”, disse José. Quando eles se aproximaram,
José continuou: “Eu sou José, o irmão que vocês venderam como escravo ao Egito. 
\textit{\tiny 5} 
Agora, não fiquem aflitos ou furiosos uns com os outros por terem me vendido
para cá. Foi Deus quem me enviou adiante de vocês para lhes preservar a vida. 
\textit{\tiny 6} 
A
fome que assola a terra há dois anos continuará por mais cinco anos, e não haverá
plantio nem colheita. 
\textit{\tiny 7} 
Deus me enviou adiante para salvar a vida de vocês e de
suas famílias, e para salvar muitas vidas. 
\textit{\tiny 8} 
Portanto, foi Deus quem me mandou para cá, e não vocês! E foi ele quem me fez conselheiro
do faraó, administrador
de todo o seu palácio e governador de todo o Egito. 

\bigskip
\textit{\tiny 9}
“Agora, voltem depressa a meu pai e digam-lhe: ‘Assim diz seu filho José: Deus
me fez senhor de toda a terra do Egito. Venha para cá sem demora! 
\textit{\tiny 10}
O senhor
poderá viver na região de Gósen, onde estará perto de mim com todos os seus
filhos e netos, rebanhos e gado, e todos os seus bens. 
\textit{\tiny 11}
Ali eu cuidarei do senhor,
pois ainda haverá cinco anos de escassez. Do contrário, o senhor e toda a sua
família perderão tudo que têm’”.

\bigskip
\textit{\tiny 12}
José acrescentou: “Vejam! Vocês podem comprovar com seus próprios olhos,
e também meu irmão Benjamim, que sou eu mesmo, José, que falo com vocês!
\textit{\tiny 13}
Contem a meu pai a posição de honra que ocupo aqui no Egito. Descrevam para
ele tudo que viram e tragam-no para cá o mais rápido possível”. 
\textit{\tiny 14}
Chorando de
alegria, ele abraçou Benjamim, e Benjamim também o abraçou e chorou. 
\textit{\tiny 15}
Então
José beijou cada um de seus irmãos e chorou com eles; depois os irmãos
conversaram à vontade com ele.

\bigskip
\textit{\tiny 16}
A notícia não demorou a chegar ao palácio do faraó: “Os irmãos de José estão
aqui!”. O faraó e seus oficiais se alegraram muito quando souberam disso.
\textit{\tiny 17}
O faraó disse a José: “Diga a seus irmãos: ‘Coloquem as cargas em seus
animais e voltem depressa à terra de Canaã. 
\textit{\tiny 18}
Tragam seu pai e todas as suas
famílias para cá. Eu lhes darei a melhor terra do Egito, e vocês comerão do que
esta terra produz de melhor’”.
\textit{\tiny 19}
O faraó prosseguiu: “Diga a seus irmãos: ‘Levem carruagens do Egito para
transportar as crianças pequenas, as mulheres e também seu pai. 
\textit{\tiny 20}
Não se
preocupem com seus pertences, pois o melhor de toda a terra do Egito será de
vocês’”.

\bigskip
\textit{\tiny 21}
Os filhos de Jacó
seguiram essas instruções. José providenciou carruagens,
conforme o faraó havia ordenado, e lhes deu mantimentos para a viagem.
\textit{\tiny 22}
Também presenteou cada irmão com um traje novo, mas a Benjamim deu cinco
roupas novas e trezentas peças de prata.
\textit{\tiny 23}
E, a seu pai, enviou dez jumentos
carregados com os melhores produtos do Egito e dez jumentas carregadas com
cereais, pães e outros mantimentos para a viagem.
\textit{\tiny 24}
Depois, José se despediu de seus irmãos e, enquanto partiam, disse a eles:
“Não briguem no caminho por causa do que aconteceu”. 

\bigskip
\textit{\tiny 25}
Eles saíram do Egito e
voltaram a seu pai, Jacó, na terra de Canaã.
\textit{\tiny 26}
“José ainda está vivo!”, eles disseram a seu pai. “É o governador de toda a terra
do Egito!” Jacó ficou atônito com a notícia. Não podia acreditar. 
\textit{\tiny 27}
Quando, porém,
repetiram para Jacó tudo que José lhes tinha dito, e quando ele viu as carruagens
que José havia mandado para levá-lo, encheu-se de ânimo.
\textit{\tiny 28}
Então Jacó exclamou: “Deve ser verdade! Meu filho José está vivo! Preciso ir e
vê-lo antes que eu morra!”.

\bigskip
\subsubsection*{Israel em Berseba}
 \textbf{\large  46} Jacó  partiu para o Egito com todos os seus bens. Quando chegou a
Berseba, ofereceu sacrifícios ao Deus de Isaque, seu pai. 

\bigskip
\textit{\tiny 2} 
Durante a noite, Deus
lhe falou numa visão. “Jacó! Jacó!”, chamou ele.
   “Aqui estou!”, respondeu Jacó. 
\textit{\tiny 3} 
“Eu sou Deus, o Deus de seu pai”, disse a voz. “Não tenha medo de descer ao
Egito, pois lá farei de sua família uma grande nação. 
\textit{\tiny 4} 
Descerei com você ao Egito
e certamente o trarei de volta. E José estará ao seu lado quando você morrer.” 

\bigskip
\textit{\tiny 5} 
Então Jacó saiu de Berseba, e seus filhos o levaram para o Egito.
Transportaram o pai, as crianças e as mulheres nas carruagens que o faraó lhes
havia providenciado. 
\textit{\tiny 6} 
Também levaram todos os seus rebanhos e os bens que
haviam adquirido na terra de Canaã. Assim, Jacó e toda a sua família foram para o
Egito: 
\textit{\tiny 7} 
filhos e netos, filhas e netas, todos os seus descendentes. 

\bigskip
\subsubsection*{Descendentes de Israel}
\textit{\tiny 8} 
Estes são os nomes dos descendentes de Israel, os filhos de Jacó, que foram ao
Egito:
Rúben foi o filho mais velho de Jacó. 
\textit{\tiny 9} 
Os filhos de Rúben foram: Enoque, Palu,
 Hezrom e Carmi.
\textit{\tiny 10}
Os filhos de Simeão foram: Jemuel, Jamim, Oade, Jaquim, Zoar e Saul. (A mãe de
 Saul era cananita.)
\textit{\tiny 11}
Os filhos de Levi foram: Gérson, Coate e Merari.
\textit{\tiny 12}
Os filhos de Judá foram: Er, Onã, Selá, Perez e Zerá (embora Er e Onã tivessem
 morrido na terra de Canaã). Os filhos de Perez foram: Hezrom e Hamul.
\textit{\tiny 13}
Os filhos de Issacar foram: Tolá, Puá,
Jasube
e Sinrom.
\textit{\tiny 14}
Os filhos de Zebulom foram: Serede, Elom e Jaleel.
\textit{\tiny 15}
Esses foram os filhos de Lia e Jacó nascidos em Padã-Arã, além de sua filha
Diná. Por meio de Lia, Jacó teve 33
 descendentes, tanto homens quanto mulheres.

\bigskip
\textit{\tiny 16}
Os filhos de Gade foram: Zefom, Hagi, Suni, Esbom, Eri, Arodi e Areli.
\textit{\tiny 17}
Os filhos de Aser foram:  Imná, Isvá, Isvi e Berias. A irmã deles se chamava Sera.
 Os filhos de Berias foram: Héber e Malquiel.
\textit{\tiny 18}
Esses foram os filhos de Zilpa, serva dada a Lia por Labão, seu pai. Por meio de
Zilpa, Jacó teve dezesseis descendentes.

\bigskip
\textit{\tiny 19}
Os filhos de Raquel, mulher de Jacó, foram: José e Benjamim.
\textit{\tiny 20}
Os filhos de José, nascidos no Egito, foram: Manassés e Efraim. Sua mãe foi
 Azenate, filha de Potífera, sacerdote de Om.
\textit{\tiny 21}
Os filhos de Benjamim foram: Belá, Bequer, Asbel, Gera, Naamã, Eí, Rôs, Mupim,
 Hupim e Arde.
\textit{\tiny 22}
Esses foram os filhos de Raquel e Jacó. Por meio de Raquel, Jacó teve catorze
descendentes.

\bigskip
\textit{\tiny 23}
O filho de Dã foi Husim.
\textit{\tiny 24}
Os filhos de Naftali foram: Jazeel, Guni, Jezer e Silém.
\textit{\tiny 25}
Esses   foram os filhos de Bila, serva dada a Raquel por Labão, seu pai. Por meio
de Bila, Jacó teve sete descendentes.

\bigskip
\textit{\tiny 26}
No total, 66 descendentes diretos de Jacó foram com ele para o Egito, sem
contar as esposas de seus filhos. 
\textit{\tiny 27}
Além deles, José teve dois filhos
que
nasceram no Egito, totalizando setenta 164 membros da família de Jacó no Egito.

\bigskip
\subsubsection*{Israel no Egito}
\textit{\tiny 28}
Quando estavam quase chegando, Jacó enviou Judá adiante para encontrar-se
com José e pedir-lhe informações sobre o caminho para Gósen. 
\textit{\tiny 29}
José mandou
preparar sua carruagem e partiu para Gósen, a fim de encontrar-se com seu pai,
Jacó. Quando José chegou, abraçou fortemente seu pai e, sem soltá-lo, chorou por
longo tempo. 
\textit{\tiny 30}
Por fim, Jacó disse a José: “Agora estou pronto para morrer, pois vi
seu rosto novamente e sei que você está vivo”.

\bigskip
\textit{\tiny 31}
José disse a seus irmãos e a toda a família de seu pai: “Irei ao faraó e lhe direi:
‘Meus irmãos e toda a família de meu pai chegaram da terra de Canaã. 
\textit{\tiny 32}
Eles são
pastores e criadores de gado. Trouxeram consigo seus rebanhos, seu gado e todos
os seus bens’”.
\textit{\tiny 33}
Disse também: “Quando o faraó mandar chamá-los e perguntar-lhes em que
vocês trabalham, 
\textit{\tiny 34}
digam o seguinte: ‘Durante toda a vida, nós, seus servos,
criamos rebanhos e gado, como sempre fizeram nossos antepassados’. Quando
lhe disserem isso, ele permitirá que vivam aqui na região de Gósen, pois os
egípcios desprezam os pastores”.

\bigskip
  \textbf{\large 47} José foi ver o faraó e lhe disse: “Meus pais e meus irmãos chegaram da
terra de Canaã. Trouxeram seus rebanhos, seu gado e todos os seus bens, e agora
estão na região de Gósen”. 
\textit{\tiny 2} 
José levou consigo cinco de seus irmãos e os apresentou ao faraó. 
\textit{\tiny 3} 
“Em que
vocês trabalham?”, o faraó perguntou aos irmãos.
   Eles responderam: “Nós, seus servos, somos pastores, como nossos
antepassados. 
\textit{\tiny 4} 
Viemos morar no Egito por algum tempo, pois não há pastagem
para nossos rebanhos em Canaã. A fome é terrível naquela região. Por isso,
pedimos sua permissão para morar na região de Gósen”. 

\bigskip
\textit{\tiny 5} 
Então o faraó disse a José: “Agora que seu pai e seus irmãos estão com você, 
\textit{\tiny 6} 
escolha qualquer lugar em todo o Egito para morarem. Dê-lhes a melhor terra do
Egito. Que vivam na região de Gósen. Se você descobrir entre eles homens
capazes, coloque-os para cuidar de meus rebanhos”. 

\bigskip
\textit{\tiny 7} 
Em seguida, José trouxe seu pai, Jacó, e o apresentou ao faraó. E Jacó abençoou
o faraó. 
\textit{\tiny 8} 
“Quantos anos o senhor tem?”, perguntou o faraó. 
\textit{\tiny 9} 
Jacó respondeu: “Tenho andado por este mundo há 130
 árduos anos.
Comparada à vida de meus antepassados, minha vida foi curta”. 
\textit{\tiny 10}
Então Jacó
abençoou o faraó novamente antes de deixar a corte.

\bigskip
\textit{\tiny 11}
José deu a seu pai e a seus irmãos a melhor terra do Egito, a região de
Ramessés, e os acomodou ali, conforme o faraó havia ordenado. 
\textit{\tiny 12}
José também
providenciou mantimentos para seu pai e seus irmãos, em quantidades
proporcionais ao número de seus dependentes, incluindo as crianças pequenas.

\bigskip
\subsubsection*{José enriquece Faraó}
\textit{\tiny 13}
A essa altura, a escassez era tanta que se esgotaram todos os mantimentos, e
havia gente passando fome em toda a terra do Egito e de Canaã. 

\bigskip
\textit{\tiny 14}
Com o tempo,
vendendo cereais para o povo, José arrecadou todo o dinheiro do Egito e de Canaã
e o depositou no tesouro do faraó. 

\bigskip
\textit{\tiny 15}
Quando acabou o dinheiro do povo do Egito
e de Canaã, todos os egípcios foram implorar a José: “Não temos mais dinheiro!
Mas, por favor, dê-nos alimento, ou morreremos de fome diante dos seus olhos!”.
\textit{\tiny 16}
José respondeu: “Visto que seu dinheiro acabou, tragam-me seus animais. Eu
lhes darei alimento em troca”. 
\textit{\tiny 17}
Então eles entregaram seus animais a José em
troca de alimento. José lhes forneceu mantimentos para mais um ano em troca de
seus cavalos, rebanhos de ovelhas, bois e jumentos.

\bigskip
\textit{\tiny 18}
Mas aquele ano chegou ao fim e, no ano seguinte, o povo voltou a José,
dizendo: “Não podemos esconder a verdade. Nosso dinheiro acabou, e nossos
rebanhos e gado lhe pertencem. Não nos resta coisa alguma para oferecer além de
nosso corpo e nossas terras. 
\textit{\tiny 19}
Por que morreríamos de fome diante dos seus
olhos? Compre nossas terras em troca de mantimento; oferecemos nossas
propriedades e a nós mesmos como servos do faraó. Dê-nos cereais para que
vivamos e não morramos, e para que a terra não fique vazia e desolada”.
\textit{\tiny 20}
Assim, José comprou toda a terra do Egito para o faraó. Todos os egípcios
venderam seus campos, pois a fome era terrível, e em pouco tempo todas as terras
passaram a ser propriedade do faraó. 
\textit{\tiny 21}
Quanto ao povo, José os tornou todos
escravos,
de uma extremidade do Egito à outra. 
\textit{\tiny 22}
As únicas terras que ele não
comprou foram as dos sacerdotes. Eles recebiam do faraó uma porção regular de
mantimentos, por isso não precisaram vender suas terras.

\bigskip
\textit{\tiny 23}
Então José disse ao povo: “Hoje eu comprei vocês e suas terras para o faraó.
Em troca, fornecerei sementes para cultivarem os campos. 
\textit{\tiny 24}
Quando vocês os
ceifarem, um quinto da colheita será do faraó. Fiquem com os outros quatro
quintos e usem como alimento para vocês, para os membros de sua casa e para
suas crianças”.
\textit{\tiny 25}
“O senhor salvou nossa vida!”, exclamaram. “Permita-nos servir ao faraó.”
\textit{\tiny 26}
Então José mandou publicar um decreto que vale até hoje na terra do Egito,
segundo o qual um quinto de todas as colheitas pertence ao faraó. Apenas as
terras dos sacerdotes não foram entregues ao faraó.
\bigskip
\textit{\tiny 27}
Enquanto isso, o povo de Israel se estabeleceu na região de Gósen, no Egito.
Ali, adquiriram propriedades e tiveram muitos filhos, e sua população cresceu
rapidamente. 

----------------------------------------------------------------------
