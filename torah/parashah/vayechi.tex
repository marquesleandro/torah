\section*{Parashat Vayechi 47:28 - 50:26}


\bigskip
\textit{\tiny 28}
Depois de chegar ao Egito, Jacó viveu mais dezessete anos;
portanto, viveu ao todo 147 anos.

\bigskip
\textit{\tiny 29}
Quando se aproximava a hora de sua morte, Jacó
chamou seu filho José e
lhe disse: “Peço que me faça um favor. Coloque sua mão debaixo da minha coxa e
jure que mostrará sua bondade e lealdade a mim atendendo a este último desejo:
não me sepulte no Egito. 
\textit{\tiny 30}
Quando eu morrer, leve meu corpo para fora do Egito
e sepulte-me com meus antepassados”.
   José prometeu: “Farei como o senhor me pede”.

\bigskip
\textit{\tiny 31}
“Jure que o fará”, insistiu. José fez o juramento, e Jacó se curvou
humildemente à cabeceira de sua cama.

\bigskip
\subsubsection*{Israel, Efraim e Manassés}
  \textbf{\large 48} Certo dia, não muito tempo depois, avisaram José: “Seu pai está bastante
doente”. José foi visitá-lo e levou consigo seus dois filhos, Manassés e Efraim. 

\bigskip
\textit{\tiny 2} 
Quando José chegou, anunciaram a Jacó: “Seu filho José está aqui para vê-lo”.
Com as forças que lhe restavam, Jacó
se sentou na cama. 

\bigskip
\textit{\tiny 3} 
Jacó disse a José: “O Deus Todo-poderoso
me apareceu em Luz, na terra de
Canaã, e me abençoou. 
\textit{\tiny 4} 
Ele me disse: ‘Eu o tornarei fértil e multiplicarei seus
descendentes. Farei de você muitas nações e darei esta terra de Canaã a seus
descendentes como propriedade permanente’. 
\textit{\tiny 5} 
“Agora, tomo para mim, como meus próprios filhos, seus dois rapazes, Efraim
e Manassés, nascidos aqui na terra do Egito antes de minha chegada. Eles serão
meus filhos, como são Rúben e Simeão. 

\bigskip
\textit{\tiny 6} 
Os filhos que você tiver depois deles,
porém, serão seus e herdarão propriedades dentro do território dos irmãos deles,
Efraim e Manassés. 

\bigskip
\textit{\tiny 7} 
“Muito tempo atrás, quando eu voltava de Padã-Arã,
Raquel morreu na terra
de Canaã. Ainda estávamos viajando, a certa distância de Efrata (ou seja, Belém).
Com grande tristeza, sepultei-a ali mesmo, junto ao caminho para Efrata” 

\bigskip
\textit{\tiny 8} 
Em seguida, Jacó olhou para os dois rapazes e perguntou: “Quem são estes?”. 
\textit{\tiny 9} 
José respondeu: “Estes são os filhos que Deus me deu aqui no Egito”.
  Jacó disse: “Traga-os mais perto, para que eu os abençoe”.
\textit{\tiny 10}
Os olhos de Jacó estavam enfraquecidos por causa da idade, e ele quase não
conseguia enxergar. José levou os rapazes para perto dele, e Jacó os beijou e
abraçou. 

\bigskip
\textit{\tiny 11}
Então Jacó disse a José: “Nunca imaginei que voltaria a ver seu rosto,
mas agora Deus me permitiu ver também seus filhos!”.
\textit{\tiny 12}
José tirou os rapazes de junto dos joelhos do avô e se curvou com o rosto no
chão. 

\bigskip
\textit{\tiny 13}
Em seguida, colocou os rapazes na frente de Jacó. Com a mão direita,
colocou Efraim diante da mão esquerda de Jacó e, com a mão esquerda, colocou
Manassés sob a mão direita de Jacó. 
\textit{\tiny 14}
Mas, ao estender as mãos para colocá-las
sobre a cabeça dos rapazes, Jacó cruzou os braços. Pôs a mão direita sobre a
cabeça de Efraim, embora fosse o mais novo, e a mão esquerda sobre a cabeça de
Manassés, embora fosse o mais velho. 

\bigskip
\textit{\tiny 15}
Em seguida, abençoou José, dizendo:
    “Que o Deus diante do qual andaram
      meu avô, Abraão, e meu pai, Isaque,
    o Deus que tem sido meu pastor
       toda a minha vida, até o dia de hoje,    
\textit{\tiny 16}
o Anjo que me resgatou de todo o mal,
       abençoe estes rapazes.
    Que eles preservem meu nome
       e o nome de Abraão e Isaque,
    e seus descendentes se multipliquem
       grandemente na terra”.

\bigskip
          José, porém, não se agradou quando viu o pai colocar a mão direita sobre a
\textit{\tiny 17}
cabeça de Efraim. Por isso, levantou-a para passá-la da cabeça de Efraim para a
cabeça de Manassés. 

\bigskip
\textit{\tiny 18}
“Não, meu pai”, disse ele. “Este é o mais velho; coloque a
mão direita sobre a cabeça dele.”
\textit{\tiny 19}
Mas seu pai se recusou e disse: “Eu sei, meu filho; eu sei. Manassés também
se tornará um grande povo, mas seu irmão mais novo será ainda maior. E seus
descendentes se tornarão muitas nações”.

\bigskip
\textit{\tiny 20}
Assim, Jacó abençoou os rapazes naquele dia com a seguinte bênção: “O povo
de Israel usará seus nomes quando pronunciarem uma bênção. Dirão: ‘Deus os
faça prosperar como Efraim e Manassés!’”. Desse modo, Jacó pôs Efraim adiante
de Manassés.

\bigskip
\textit{\tiny 21}
Então Jacó disse a José: “Morrerei em breve, mas Deus estará com vocês e os
levará de volta a Canaã, a terra de seus antepassados. 
\textit{\tiny 22}
Em razão de sua
autoridade sobre seus irmãos, eu lhe dou uma porção a mais da terra,
que
tomei dos amorreus com a minha espada e o meu arco”.

\bigskip
\subsubsection*{Israel, os doze e a benção}
  \textbf{\large 49} Então Jacó mandou chamar todos os seus filhos e lhes disse: “Reúnam-se
ao meu redor, e eu direi o que acontecerá a cada um de vocês nos dias que virão. 
\textit{\tiny 2} 
“Venham e ouçam, filhos de Jacó,
       ouçam Israel, seu pai! 

\bigskip
\textit{\tiny 3} 
“Rúben, você é meu filho mais velho, minha força,
      o filho da minha juventude vigorosa;
      é o primeiro em importância e o primeiro em poder. 
\textit{\tiny 4} 
É, contudo, impetuoso como uma enchente,
      e não será mais o primeiro.
    Pois deitou-se em minha cama,  desonrou meu leito conjugal. 

\bigskip
\textit{\tiny 5} 
“Simeão e Levi são iguais em tudo;
  suas armas são instrumentos de violência. 
\textit{\tiny 6} 
Que eu jamais esteja presente em suas reuniões
  e nunca participe de seus planos.
Pois, em sua ira, mataram homens
  e, por diversão, aleijaram bois. 
\textit{\tiny 7} 
Maldita seja sua ira, pois é feroz;
  maldita sua fúria, pois é cruel.
Eu os espalharei entre os descendentes de Jacó,
  eu os dispersarei por todo o Israel. 

\bigskip
\textit{\tiny 8} 
“Judá, seus irmãos o louvarão;
   você agarrará seus inimigos pelo pescoço,
   e todos os seus parentes se curvarão à sua frente. 
\textit{\tiny 9} 
Judá, meu filho, é um leão novo
   que acabou de comer sua presa.
Como o leão, ele se agacha, e como a leoa, se deita;
   quem tem coragem de acordá-lo?
\textit{\tiny 10}
O cetro não se afastará de Judá,
   nem o bastão de autoridade de seus descendentes,
até que venha aquele a quem pertence,
   aquele que todas as nações honrarão.
\textit{\tiny 11}
Ele amarra seu potro a uma videira,
   seu jumentinho a uma videira seleta.
Lava suas roupas em vinho,
   suas vestes, no sangue das uvas.
\textit{\tiny 12}
Seus olhos são mais escuros que o vinho,
   seus dentes, mais brancos que o leite.

\bigskip
\textit{\tiny 13}
“Zebulom se estabelecerá à beira-mar
  e será um porto para os navios;
  suas fronteiras se estenderão até Sidom.

\bigskip
\textit{\tiny 14}
“Issacar é um jumento forte,
  que descansa entre dois sacos de carga.
Quando vir como o campo é bom
  e como a terra é agradável,
curvará seus ombros para a carga
  e se sujeitará a trabalhos forçados.

\bigskip
\textit{\tiny 16}
“Dã governará seu povo,
  como qualquer outra tribo de Israel.
\textit{\tiny 17}
Dã será uma serpente à beira da estrada,
  uma víbora junto ao caminho
que morde o calcanhar do cavalo
  e faz o cavaleiro cair.
\textit{\tiny 18}
Ó SENHOR, espero pelo teu livramento!

\bigskip
\textit{\tiny 19}
“Gade será atacado por bandos de saqueadores,
  mas os atacará quando baterem em retirada.

\bigskip
\textit{\tiny 20}
“Aser se alimentará de comidas deliciosas
  e produzirá iguarias dignas de reis.

\bigskip
\textit{\tiny 21}
“Naftali é uma gazela solta
  que dá à luz lindos filhotes.

\bigskip
\textit{\tiny 22}
“José é árvore frutífera,
  árvore frutífera junto à fonte;
  seus ramos se estendem por cima do muro.
\textit{\tiny 23}
Arqueiros o atacaram brutalmente;
  atiraram nele e o atormentaram.
\textit{\tiny 24}
Seu arco, porém, permaneceu esticado,
  e seus braços foram fortalecidos
pelas mãos do Poderoso de Jacó,
  pelo Pastor, a Rocha de Israel.
\textit{\tiny 25}
Que o Deus de seu pai o ajude;
  o Todo-poderoso o abençoe
com bênçãos dos altos céus,
  bênçãos das profundezas das águas,
  bênçãos dos seios e do ventre.
\textit{\tiny 26}
Que as bênçãos de seu pai ultrapassem
  as bênçãos de meus antepassados
  e alcancem as alturas das antigas colinas.
Que essas bênçãos descansem sobre a cabeça de José,
       que é príncipe entre seus irmãos.
    
\bigskip
\textit{\tiny 27}
“Benjamim é um lobo voraz;
       pela manhã devora seus inimigos,
       ao entardecer divide o despojo”.
    
\bigskip
\textit{\tiny 28}
Essas são as doze tribos de Israel, e foi isso que seu pai disse ao despedir-se
de seus filhos. Deu a cada um deles a bênção que lhe era adequada.


\bigskip
\subsubsection*{A morte de Israel}
\textit{\tiny 29}
Em seguida, Jacó lhes deu a seguinte instrução: “Em breve morrerei e me
reunirei a meus antepassados. Sepultem-me com meu pai e com meu avô na
caverna no campo de Efrom, o hitita. 
\textit{\tiny 30}
É a caverna de Macpela, perto de Manre,
em Canaã, que Abraão comprou do hitita como sepultura permanente. 
\textit{\tiny 31}
Ali estão
sepultados Abraão e sua mulher, Sara. Ali também estão sepultados Isaque e sua
mulher, Rebeca. E ali sepultei Lia. 
\textit{\tiny 32}
É o campo e a caverna que meu avô, Abraão,
comprou dos hititas”.
\textit{\tiny 33}
Quando Jacó terminou de dar essa instrução a seus filhos, deitou-se em sua
cama, deu o último suspiro e, ao morrer, reuniu-se a seus antepassados.

\bigskip
  \textbf{\large 50} José atirou-se sobre seu pai, chorou sobre ele e o beijou. 
\textit{\tiny 2} 
Em seguida, deu
ordens aos médicos que o serviam para que embalsamassem o corpo de seu pai, e
Jacó
foi embalsamado. 
\textit{\tiny 3} 
O processo de embalsamamento levou os quarenta
dias habituais. E os egípcios lamentaram sua morte durante setenta dias. 

\bigskip
\textit{\tiny 4} 
Quando terminou o período de luto, José procurou os conselheiros do faraó e
lhes disse: “Por gentileza, peço que falem com o faraó em meu favor. 
\textit{\tiny 5} 
Digam-lhe
que meu pai me fez prestar um juramento. Disse: ‘Morrerei em breve. Leve meu
corpo de volta para a terra de Canaã e coloque-me na sepultura que preparei para
mim’. Portanto, peço que me deixe ir sepultar meu pai; depois, voltarei sem
demora”. 
\textit{\tiny 6} 
O faraó atendeu ao pedido de José e disse: “Vá e sepulte seu pai, como ele o fez
prometer”. 

\bigskip
\textit{\tiny 7} 
Então José partiu para sepultar seu pai. Foi acompanhado de todos os
oficiais do faraó, todos os membros mais importantes da casa do faraó e todos os
oficiais de alto escalão do Egito. 
\textit{\tiny 8} 
José também levou consigo toda a sua família,
seus irmãos e a família deles. As crianças pequenas, os rebanhos e o gado, porém,
deixaram na terra de Gósen. 
\textit{\tiny 9} 
Muitas carruagens e seus condutores
acompanharam José, formando um grande cortejo.  
\textit{\tiny 10}
Quando   chegaram à eira de Atade, perto do rio Jordão, realizaram uma
grande cerimônia fúnebre, com um período de sete dias de luto pelo pai de José.
\textit{\tiny 11}
Os cananeus que moravam na região os viram chorar na eira de Atade e
mudaram o nome do lugar (que fica próximo ao Jordão) para Abel-Mizraim,
pois disseram: “Este é um lugar de lamento profundo para esses egípcios”.

\bigskip
\textit{\tiny 12}
Assim, os filhos de Jacó fizeram o que ele lhes havia ordenado. 
\textit{\tiny 13}
Levaram
seu corpo para a terra de Canaã e o sepultaram na caverna no campo de Macpela,
perto de Manre. Essa é a caverna que Abraão havia comprado de Efrom, o hitita,
como sepultura permanente.

\bigskip
\subsubsection*{José e os irmão (parte 3)}
\textit{\tiny 14}
Depois de sepultar Jacó, José voltou para o Egito com seus irmãos e com todos
que o haviam acompanhado. 
\textit{\tiny 15}
Uma vez que seu pai estava morto, porém, os
irmãos de José ficaram temerosos e disseram: “Agora José mostrará sua ira e se
vingará de todo o mal que lhe fizemos”.
\textit{\tiny 16}
Por isso, enviaram a seguinte mensagem a José: “Antes de morrer, nosso pai
mandou 
\textit{\tiny 17}
que lhe disséssemos: ‘Por favor, perdoe seus irmãos pelo grande mal
que eles lhe fizeram, pelo pecado que cometeram ao tratá-lo com tanta
crueldade’. Por isso, nós, servos do Deus de seu pai, suplicamos que você perdoe
nosso pecado”. Quando José recebeu a mensagem, começou a chorar. 

\bigskip
\textit{\tiny 18}
Depois,
seus irmãos chegaram e se curvaram com o rosto no chão diante de José. “Somos
seus escravos!”, disseram eles.
\textit{\tiny 19}
José, porém, respondeu: “Não tenham medo de mim. Por acaso sou Deus para
castigá-los? 
\textit{\tiny 20}
Vocês pretendiam me fazer o mal, mas Deus planejou tudo para o
bem. Colocou-me neste cargo para que eu pudesse salvar a vida de muitos. 
\textit{\tiny 21}
Não
tenham medo. Continuarei a cuidar de vocês e de seus filhos”. Desse modo, ele os
tranquilizou ao tratá-los com bondade.

\bigskip
\subsubsection*{A morte de José}
\textit{\tiny 22}
José, seus irmãos e suas famílias continuaram a viver no Egito. José viveu 110 anos. 

\bigskip
\textit{\tiny 23}
Chegou a ver três gerações de descendentes de seu filho Efraim e o
nascimento dos filhos de Maquir, filho de Manassés, os quais ele tomou para si
como se fossem seus.

\bigskip
\textit{\tiny 24}
José disse a seus irmãos: “Em breve morrerei, mas certamente Deus os
ajudará e os tirará desta terra. Ele os levará de volta para a terra que prometeu
solenemente dar a Abraão, Isaque e Jacó”.
\textit{\tiny 25}
Então  José fez os filhos de Israel prestarem um juramento e disse: “Quando
Deus vier ajudá-los e conduzi-los de volta, levem meus ossos com vocês”. 

\bigskip
\textit{\tiny 26}
José
morreu com 110
anos. Os egípcios o embalsamaram e o colocaram em um caixão
no Egito.

----------------------------------------------------------------------
