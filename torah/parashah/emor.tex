\section*{Parashat Emor 21:1-24:23}

Instruções para os sacerdotes
   
\textit{\tiny 21}
 O SENHOR disse a Moisés: “Dê as seguintes instruções aos sacerdotes, os
filhos de Arão.
   “Nenhum sacerdote deverá se tornar cerimonialmente impuro por causa da
morte de alguém do povo.
 
\textit{\tiny 2} As únicas exceções são seus parentes mais próximos:
mãe ou pai, filho ou filha, irmão
 
\textit{\tiny 3} ou irmã virgem que dependa dele, uma vez que
não tem marido. Nesse caso, poderá contaminar-se.
 
\textit{\tiny 4} O sacerdote não deverá
contaminar-se e tornar-se cerimonialmente impuro por causa de algum parente
de sua esposa.
   
\textit{\tiny 5} “Os sacerdotes não rasparão a cabeça, não rasparão a barba rente à pele, nem
farão cortes no corpo.
 
\textit{\tiny 6} Serão consagrados ao seu Deus e jamais desonrarão o
nome de Deus, pois são eles que apresentam as ofertas especiais para o SENHOR,
ofertas de alimento para o seu Deus.
   
\textit{\tiny 7} “Os sacerdotes não se casarão com uma mulher contaminada pela
prostituição, nem se casarão com uma mulher divorciada do marido, pois o
sacerdote é consagrado ao seu Deus.
 
\textit{\tiny 8} Tratem-no como santo, pois ele traz as
ofertas de alimento perante o seu Deus. Considerem-no santo, pois eu, o SENHOR,
sou santo e santifico vocês.
   
\textit{\tiny 9} “Se a filha de um sacerdote se tornar prostituta e, desse modo, se contaminar,
também contamina a santidade de seu pai e deverá morrer queimada.
   
\textit{\tiny 10}
“O sumo sacerdote ocupa a posição mais elevada entre todos os sacerdotes. O
óleo da unção foi derramado sobre sua cabeça, e ele foi consagrado para vestir as
roupas sacerdotais. Nunca deixará o cabelo despenteado
 nem rasgará suas
roupas em sinal de luto. 
\textit{\tiny 11}
Não se contaminará por aproximar-se de um cadáver.
Não se tornará cerimonialmente impuro nem mesmo por causa de seu pai ou de
sua mãe. 
\textit{\tiny 12}
Não deixará o santuário, nem contaminará o santuário do seu Deus,
pois foi consagrado pelo óleo da unção de seu Deus. Eu sou o SENHOR.
   
\textit{\tiny 13}
“O sumo sacerdote somente se casará com uma virgem. 
\textit{\tiny 14}
Não se casará com
uma viúva, nem com uma mulher divorciada, nem com uma mulher contaminada
pela prostituição. Sua esposa deverá ser uma virgem de seu próprio clã, 
\textit{\tiny 15}
para
que ele não desonre seus descendentes entre o povo, pois eu sou o SENHOR, que o
santifico”.
   
\textit{\tiny 16}
Então o SENHOR disse a Moisés: 
\textit{\tiny 17}
“Dê as seguintes instruções a Arão. Nas
gerações futuras, nenhum de seus descendentes portador de algum defeito físico
estará qualificado para trazer ofertas de alimento ao seu Deus. 
\textit{\tiny 18}
Nenhum
homem que tenha algum defeito estará qualificado, seja ele cego, aleijado,
mutilado ou deformado, 
\textit{\tiny 19}
ou tenha o pé ou braço quebrado, 
\textit{\tiny 20}
ou seja corcunda,
ou anão, ou tenha um olho defeituoso, ou feridas na pele ou sarna, ou testículos
defeituosos. 
\textit{\tiny 21}
Nenhum descendente de Arão que tenha algum defeito se
aproximará do altar para apresentar ofertas especiais para o SENHOR. Uma vez que
tem defeito, não poderá se aproximar do altar para trazer ofertas de alimento ao
seu Deus. 
\textit{\tiny 22}
No entanto, poderá comer do alimento oferecido a Deus, das ofertas
santas e das ofertas santíssimas. 
\textit{\tiny 23}
Mas, por causa de seu defeito físico, não
passará adiante da cortina interna nem se aproximará do altar, pois contaminaria
meus lugares santos. Eu sou o SENHOR, que santifico esses lugares”.
   
\textit{\tiny 24}
Moisés deu essas instruções a Arão, a seus filhos e a todos os israelitas.

   
\textit{\tiny 22}
 O SENHOR disse a Moisés:
 
\textit{\tiny 2} “Diga a Arão e a seus filhos que tenham muito
respeito pelas ofertas sagradas que os israelitas consagrarem a mim, a fim de não
desonrarem meu santo nome. Eu sou o SENHOR.
 
\textit{\tiny 3} Dê a eles as seguintes instruções.
   “Nas gerações futuras, se algum de seus descendentes estiver cerimonialmente
impuro ao se aproximar das ofertas sagradas que os israelitas consagrarem ao
SENHOR, ele será eliminado de minha presença. Eu sou o SENHOR.
   
\textit{\tiny 4} “Se algum dos descendentes de Arão tiver lepra
 ou qualquer tipo de fluxo
que o torne cerimonialmente impuro, não comerá das ofertas sagradas enquanto
não for declarado puro. Também se tornará impuro se tocar num cadáver, expelir
sêmen,
 
\textit{\tiny 5} tocar num animal que rasteja pelo chão e seja impuro ou tocar em
alguém que, por qualquer motivo, esteja cerimonialmente impuro.
 
\textit{\tiny 6} Quem se
contaminar de alguma dessas formas ficará impuro até o entardecer. Não comerá
das ofertas sagradas enquanto não tiver se banhado com água.
 
\textit{\tiny 7} Depois do pôr do
sol, estará cerimonialmente puro outra vez e poderá comer das ofertas sagradas,
pois são seu alimento.
 
\textit{\tiny 8} Não comerá um animal que morreu de forma natural ou
que foi despedaçado por animais selvagens, pois se contaminaria. Eu sou o
SENHOR.
  
\textit{\tiny 9} “Os   sacerdotes obedecerão fielmente às minhas ordens. Do contrário, serão
culpados de pecado e morrerão, pois menosprezaram o que lhes ordenei. Eu sou o
SENHOR, que os santifica.
   
\textit{\tiny 10}
“Ninguém de fora da família do sacerdote comerá das ofertas sagradas. Nem
mesmo hóspedes e empregados da casa do sacerdote poderão comê-las. 
\textit{\tiny 11}
Mas, se
o sacerdote comprar um escravo, esse escravo poderá comer das ofertas sagradas.
E, se o escravo tiver filhos, eles também poderão comer de seu alimento. 
\textit{\tiny 12}
Se a
filha do sacerdote se casar com alguém de fora da família sacerdotal, não poderá
mais comer das ofertas sagradas. 
\textit{\tiny 13}
Se, contudo, ficar viúva ou divorciar-se, sem
ter filhos para sustentá-la, e voltar a morar na casa do pai, como quando era
jovem, poderá comer novamente do alimento do pai. Com exceção desses casos,
ninguém de fora da família do sacerdote comerá das ofertas sagradas.
   
\textit{\tiny 14}
“Se alguém não autorizado comer das ofertas sagradas por engano, pagará ao
sacerdote aquilo que comeu, mais um quinto do valor. 
\textit{\tiny 15}
Os sacerdotes não
contaminarão as ofertas sagradas apresentadas pelos israelitas ao SENHOR,

\textit{\tiny 16}
permitindo que sejam consumidas por pessoas não autorizadas. Elas se
tornariam culpadas e teriam de fazer reparação. Eu sou o SENHOR, que os
santifica”.
Ofertas dignas e indignas

\textit{\tiny 17}
O SENHOR também disse a Moisés: 
\textit{\tiny 18}
“Dê as seguintes instruções a Arão, a seus
filhos e a todo o povo de Israel. Elas se aplicam tanto aos israelitas de nascimento
como aos estrangeiros que vivem entre vocês.
   “Se alguém apresentar ao SENHOR um holocausto, seja como cumprimento de
um voto ou como oferta voluntária, 
\textit{\tiny 19}
só será aceito se o animal oferecido for um
macho sem defeito. Poderá ser um boi, um carneiro ou um bode. 
\textit{\tiny 20}
Não
apresentem um animal defeituoso, pois o SENHOR não o aceitará em favor de
vocês.
   
\textit{\tiny 21}
“Se alguém apresentar ao SENHOR uma oferta de paz, seja como cumprimento
de um voto ou como oferta voluntária, escolha do gado ou do rebanho um animal
perfeito, sem defeito algum. 
\textit{\tiny 22}
Não ofereçam um animal cego, aleijado, ferido, ou
que tenha um quisto, um ferimento na pele, ou sarna. Nunca ofereçam nenhum
desses animais no altar como ofertas especiais para o SENHOR. 
\textit{\tiny 23}
Se um boi
\textit{\tiny 34}
 ou
um cordeiro tiver uma perna mais comprida ou mais curta que as outras, poderá
ser apresentado como oferta voluntária, mas não como cumprimento de um voto.

\textit{\tiny 24}
Se um animal tiver testículos danificados ou for castrado, não poderá ser
oferecido ao SENHOR. Nunca façam isso em sua própria terra 
\textit{\tiny 25}
e não recebam de
estrangeiros animais como esses em pagamento, para depois oferecê-los como
sacrifício a Deus. Não serão aceitos em seu favor, pois são mutilados ou
defeituosos”.
   
\textit{\tiny 26}
E o SENHOR disse a Moisés: 
\textit{\tiny 27}
“Quando nascer um bezerro, cordeiro ou
cabrito, ficará sete dias com a mãe. Do oitavo dia em diante, será aceitável como
oferta especial para o SENHOR. 
\textit{\tiny 28}
Não matem a mãe e sua cria no mesmo dia, seja
uma vaca, uma ovelha ou uma cabra. 
\textit{\tiny 29}
Quando levarem uma oferta de gratidão ao
SENHOR, sacrifiquem-na corretamente para que sejam aceitos. 
\textit{\tiny 30}
Comam todo o
animal sacrificado no dia em que for apresentado. Não deixem parte alguma do
animal até a manhã seguinte. Eu sou o SENHOR.
   
\textit{\tiny 31}
“Guardem fielmente meus mandamentos pondo-os em prática, pois eu sou o
SENHOR. 
\textit{\tiny 32}
Não desonrem meu santo nome, pois demonstrarei minha santidade no
meio dos israelitas. Eu sou o SENHOR, que os santifica. 
\textit{\tiny 33}
Eu os libertei da terra do
Egito para ser o seu Deus. Eu sou o SENHOR”.
As festas de Israel
   
\textit{\tiny 23}
 O SENHOR disse a Moisés:
 
\textit{\tiny 2} “Dê as seguintes instruções ao povo de Israel.
Estas são as festas que o SENHOR estabeleceu e que vocês proclamarão como
reuniões sagradas.
   
\textit{\tiny 3} “Vocês têm seis dias na semana para fazer os trabalhos habituais, mas o
sétimo dia é o sábado, o dia de descanso absoluto e de reunião sagrada. Não
façam trabalho algum, pois é o sábado do SENHOR e deve ser guardado onde quer
que morarem.
   
\textit{\tiny 4} “Além do sábado, estas são as festas que o SENHOR estabeleceu, as reuniões
sagradas que serão celebradas anualmente no devido tempo.”
A Páscoa e a Festa dos Pães sem Fermento
 
\textit{\tiny 5} “A Páscoa do SENHOR começa ao entardecer do décimo quarto dia do primeiro
mês.
 
\textit{\tiny 6} No dia seguinte, o décimo quinto dia, comecem a celebrar a Festa dos
Pães sem Fermento. Essa celebração em homenagem ao SENHOR continuará por
sete dias e, durante esse tempo, o pão que comerem será preparado sem
fermento.
 
\textit{\tiny 7} No primeiro dia da festa, todos suspenderão seus trabalhos habituais
e realizarão uma reunião sagrada.
 
\textit{\tiny 8} Durante sete dias, apresentarão ofertas especiais para o SENHOR. No sétimo dia, suspenderão novamente seus trabalhos
habituais para realizar uma reunião sagrada”.
A Celebração do Início da Colheita
 
\textit{\tiny 9} O SENHOR disse a Moisés: 
\textit{\tiny 10}
“Dê as seguintes instruções ao povo de Israel. Quando
entrarem na terra que eu lhes dou e começarem a primeira colheita, levem ao
sacerdote um feixe dos primeiros cereais que colherem. 
\textit{\tiny 11}
No dia depois do
sábado, o sacerdote moverá o feixe para o alto diante do SENHOR, para que seja
aceito em favor de vocês. 
\textit{\tiny 12}
Nesse mesmo dia, ofereçam um cordeiro de um ano,
sem defeito, como holocausto para o SENHOR. 
\textit{\tiny 13}
Junto com o sacrifício,
apresentem uma oferta de cereal de quatro litros
\textit{\tiny 36}
 de farinha da melhor
qualidade umedecida com azeite. Será uma oferta especial, um aroma agradável
ao SENHOR. Ofereçam também um litro
\textit{\tiny 37}
 de vinho como oferta derramada.

\textit{\tiny 14}
Nesse dia, não comam pão algum, nem cereal torrado ou fresco, enquanto não
apresentarem a oferta ao seu Deus. Essa é uma lei permanente para vocês e deve
ser cumprida de geração em geração, onde quer que morarem.”
A Festa da Colheita

\textit{\tiny 15}
“A partir do dia seguinte ao sábado, o dia em que levarem o feixe de cereal a fim
de ser movido para o alto como oferta especial, contem sete semanas completas.

\textit{\tiny 16}
Continuem contando até o dia depois do sétimo sábado, isto é, cinquenta dias
depois. Então apresentem uma oferta de cereal novo para o SENHOR. 
\textit{\tiny 17}
Onde quer
que morarem, levem dois pães que serão movidos para o alto como oferta especial
diante do SENHOR. Preparem os pães com quatro quilos de farinha da melhor
qualidade e assem-nos com fermento. Serão uma oferta para o SENHOR dos
primeiros frutos de sua colheita. 
\textit{\tiny 18}
Junto com o pão, apresentem sete cordeiros de
um ano e sem defeito, um novilho e dois carneiros como holocaustos para o
SENHOR. Esses holocaustos, junto com as ofertas de cereal e ofertas derramadas,
serão uma oferta especial, um aroma agradável ao SENHOR. 
\textit{\tiny 19}
Em seguida,
ofereçam um bode como oferta pelo pecado e dois cordeiros de um ano como
ofertas de paz.
  
\textit{\tiny 20}
“O sacerdote levantará os dois cordeiros como oferta especial para o SENHOR,
junto com os pães que representam os primeiros frutos de suas colheitas. Essas
ofertas, que são santas para o SENHOR, pertencem aos sacerdotes. 
\textit{\tiny 21}
Esse mesmo
dia será declarado dia de reunião sagrada, um dia em que não farão nenhum
trabalho habitual. Essa é uma lei permanente para vocês e deve ser cumprida de
geração em geração, onde quer que morarem.
\textit{\tiny 38}

   
\textit{\tiny 22}
“Quando fizerem a colheita da sua terra, não colham as espigas nos cantos
dos campos e não apanhem aquilo que cair das mãos dos ceifeiros. Deixem esses
grãos para os pobres e estrangeiros que vivem entre vocês. Eu sou o SENHOR, seu
Deus”.
A Festa das Trombetas

\textit{\tiny 23}
O SENHOR disse a Moisés: 
\textit{\tiny 24}
“Dê as seguintes instruções ao povo de Israel. No
primeiro dia do sétimo mês,
\textit{\tiny 39}
 tenham um dia de descanso absoluto. Será uma
reunião sagrada, uma celebração memorial comemorada com toques de
trombeta. 
\textit{\tiny 25}
Não façam nenhum trabalho habitual nesse dia, mas apresentem
ofertas especiais para o SENHOR”.
O Dia da Expiação

\textit{\tiny 26}
O SENHOR disse a Moisés: 
\textit{\tiny 27}
“Comemorem o Dia da Expiação no décimo dia do
mesmo sétimo mês.
\textit{\tiny 40}
 Celebrem-no como uma reunião sagrada, um dia para se
humilharem
\textit{\tiny 41}
 e apresentarem ofertas especiais para o SENHOR. 
\textit{\tiny 28}
Não façam
trabalho algum durante todo esse dia, pois é o Dia da Expiação, no qual se fará
expiação em seu favor diante do SENHOR, seu Deus. 
\textit{\tiny 29}
Todos aqueles que não se
humilharem nesse dia serão eliminados do meio do povo. 
\textit{\tiny 30}
Destruirei aqueles
que, dentre vocês, trabalharem em algo nesse dia. 
\textit{\tiny 31}
Não façam trabalho algum.
Essa é uma lei permanente para vocês e deve ser cumprida de geração em
geração, onde quer que morarem. 
\textit{\tiny 32}
Será um sábado de descanso absoluto para
vocês e, nesse dia, deverão se humilhar. O dia de descanso começará ao
entardecer do nono dia do mês e se estenderá até o entardecer do décimo dia”.
A Festa das Cabanas

\textit{\tiny 33}
O SENHOR também disse a Moisés: 
\textit{\tiny 34}
“Dê as seguintes instruções ao povo de
Israel. Comecem a celebrar a Festa das Cabanas
\textit{\tiny 42}
 no décimo quinto dia do sétimo
mês. Essa festa em homenagem ao SENHOR durará sete dias. 
\textit{\tiny 35}
O primeiro dia da
festa será declarado reunião sagrada, na qual não farão nenhum trabalho
habitual. 
\textit{\tiny 36}
Durante sete dias, vocês apresentarão ofertas especiais para o SENHOR.
No oitavo dia, haverá outra reunião sagrada, na qual apresentarão ofertas
especiais para o SENHOR. Será uma ocasião solene, e ninguém fará nenhum
trabalho habitual.
\textit{\tiny 37}
(“Essassão as festas que o SENHOR estabeleceu. Celebrem-nas a cada ano
como reuniões sagradas, apresentando para o SENHOR, no dia apropriado, as
ofertas especiais de sacrifícios queimados, ofertas de cereal, sacrifícios e ofertas
derramadas. 
\textit{\tiny 38}
Celebrem-nas além dos sábados habituais do SENHOR e apresentem
as ofertas além das ofertas pessoais que vocês trazem no cumprimento de votos e
das ofertas voluntárias para o SENHOR.)
   
\textit{\tiny 39}
“Lembrem-se de que essa festa de sete dias em homenagem ao SENHOR, a
Festa das Cabanas, começa no décimo quinto dia do sétimo mês, depois de terem
colhido tudo que a terra produziu. Celebrem a festa do SENHOR por sete dias. O
primeiro e o oitavo dia da festa serão de descanso absoluto. 
\textit{\tiny 40}
No primeiro dia,
recolham galhos das mais belas árvores,
\textit{\tiny 43}
 folhagens de palmeiras, ramos de
árvores verdejantes e de salgueiros que crescem junto dos riachos. Celebrem com
alegria diante do SENHOR, seu Deus, por sete dias. 
\textit{\tiny 41}
Comemorem essa festa em
homenagem ao SENHOR por sete dias a cada ano. Essa é uma lei permanente para
vocês e deve ser cumprida no sétimo mês, de geração em geração. 
\textit{\tiny 42}
Durante sete
dias, morarão ao ar livre em pequenas cabanas. Todos os israelitas de nascimento
morarão em cabanas. 
\textit{\tiny 43}
Desse modo, lembrarão cada nova geração de israelitas
que eu fiz seus antepassados morarem em cabanas quando os libertei da terra do
Egito. Eu sou o SENHOR, seu Deus”.
   
\textit{\tiny 44}
Assim, Moisés transmitiu aos israelitas essas instruções sobre as festas anuais
do SENHOR.
Óleo puro e pães sagrados
   
\textit{\tiny 24}
 O SENHOR disse a Moisés:
 
\textit{\tiny 2} “Ordene aos israelitas que tragam óleo puro de
azeitonas prensadas para a iluminação do candelabro, a fim de manter as
lâmpadas sempre acesas.
 
\textit{\tiny 3} É o candelabro que fica na tenda do encontro, em
frente à cortina interna que protege a arca da aliança.
 Arão manterá as
lâmpadas acesas na presença do SENHOR a noite toda. Essa é uma lei permanente
para vocês e deve ser cumprida de geração em geração.
 
\textit{\tiny 4} Arão e os sacerdotes
manterão sempre em ordem, na presença do SENHOR, as lâmpadas do candelabro
de ouro puro.
   
\textit{\tiny 5} “Asse doze pães de farinha da melhor qualidade usando quatro litros45 de
farinha para cada pão.
 
\textit{\tiny 6} Coloque os pães diante do SENHOR sobre a mesa de ouro
puro e arrume-os em duas fileiras, com seis pães em cada fileira.
 
\textit{\tiny 7} Coloque um
pouco de incenso sobre cada fileira como oferta memorial, uma oferta especial
apresentada ao SENHOR.
 
\textit{\tiny 8} A cada sábado, coloque regularmente diante do SENHOR
esses pães como oferta da parte dos israelitas; é uma expressão contínua da
aliança sem fim.
 
\textit{\tiny 9} Os pães pertencerão a Arão e seus descendentes, que os
comerão num lugar sagrado, pois são santíssimos. Os sacerdotes terão direito
permanente a essa porção das ofertas especiais apresentadas ao SENHOR”.
A pena para quem blasfemar

\textit{\tiny 10}
Certo dia, um homem, filho de uma israelita e de um egípcio, saiu de sua tenda e
se envolveu numa briga com um dos israelitas. 
\textit{\tiny 11}
Durante a briga, o filho da
israelita blasfemou o Nome com uma maldição. Por isso, foi levado a Moisés para
ser julgado. A mãe dele se chamava Selomite, filha de Dibri, da tribo de Dã. 
\textit{\tiny 12}
O
homem foi mantido preso até ficar clara a vontade do SENHOR a respeito de sua
situação.
   
\textit{\tiny 13}
Então o SENHOR disse a Moisés: 
\textit{\tiny 14}
“Leve o blasfemador para fora do
acampamento e diga a todos que ouviram a maldição que coloquem as mãos
sobre a cabeça dele. Depois, a comunidade toda o executará por apedrejamento.

\textit{\tiny 15}
Diga ao povo de Israel: Quem amaldiçoar o seu Deus será castigado por causa
do seu pecado. 
\textit{\tiny 16}
Quem blasfemar o nome do SENHOR será morto por
apedrejamento por toda a comunidade de Israel. Qualquer israelita de
nascimento ou estrangeiro entre vocês que blasfemar o Nome será morto.
   
\textit{\tiny 17}
“Quem tirar a vida de outra pessoa será morto.
   
\textit{\tiny 18}
“Quem matar um animal pertencente a outra pessoa a indenizará com um
animal vivo.
   
\textit{\tiny 19}
“Quem ferir outra pessoa será tratado de acordo com o ferimento que
causou: 
\textit{\tiny 20}
fratura por fratura, olho por olho, dente por dente. O dano que alguém
fizer a outra pessoa, será feito a ele.
   
\textit{\tiny 21}
“Quem matar um animal indenizará seu dono totalmente, mas quem matar
uma pessoa será morto.
   
\textit{\tiny 22}
“A mesma lei se aplica tanto aos israelitas de nascimento como aos
estrangeiros que vivem entre vocês. Eu sou o SENHOR, seu Deus”.
   
\textit{\tiny 23}
Depois que Moisés transmitiu todas essas instruções aos israelitas, eles
levaram o blasfemador para fora do acampamento e o executaram por
apedrejamento. Os israelitas fizeram exatamente conforme o SENHOR havia
ordenado a Moisés.

----------------------------------------------------------------------
